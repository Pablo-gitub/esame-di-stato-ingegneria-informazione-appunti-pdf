% Questo file contiene la domanda e risposta sul confronto TCP vs UDP.
% Sarà incluso da domande_ed_esercizi_reti.tex.

\subsection*{Domanda: Confronto tra TCP e UDP}

\textbf{Domanda}: Confronta il protocollo TCP con il protocollo UDP, evidenziando le loro caratteristiche principali, vantaggi, svantaggi e applicazioni tipiche.

\textbf{Risposta}:

Il \textbf{Livello di Trasporto} del modello OSI gestisce la comunicazione end-to-end tra processi. I protocolli principali sono TCP e UDP.
\paragraph{TCP (Transmission Control Protocol)}:
\begin{itemize}
    \item \textbf{Orientato alla Connessione}: Stabilisce una connessione (handshake a tre vie).
    \item \textbf{Affidabile}: Garantisce la consegna ordinata e senza perdite (numerazione, ACK, ritrasmissioni).
    \item \textbf{Controllo di Flusso}: Previene il sovraccarico del destinatario (finestra di ricezione).
    \item \textbf{Controllo di Congestione}: Riduce la velocità in reti congestionate.
    \item \textbf{Applicazioni Tipiche}: Web (HTTP), trasferimento file (FTP), email (SMTP), SSH.
\end{itemize}
\paragraph{UDP (User Datagram Protocol)}:
\begin{itemize}
    \item \textbf{Senza Connessione}: Invia datagrammi indipendenti senza handshake.
    \item \textbf{Inaffidabile (Best-Effort)}: Non garantisce consegna, ordine o assenza di duplicazioni.
    \item \textbf{Nessun Controllo}: Non implementa controllo di flusso o congestione.
    \item \textbf{Overhead Minimo}: Header piccolo, molto efficiente.
    \item \textbf{Applicazioni Tipiche}: Streaming multimediale (VoIP, video), DNS, giochi online (dove la velocità è prioritaria).
\end{itemize}