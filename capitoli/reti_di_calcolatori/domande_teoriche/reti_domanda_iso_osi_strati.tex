% Questo file contiene la domanda e risposta sulla struttura del Modello ISO/OSI.
% Sarà incluso da domande_ed_esercizi_reti.tex.

\subsection*{Domanda: Struttura a Sette Strati del Modello ISO/OSI}

\textbf{Domanda}: Qual è la struttura a sette strati del Modello ISO/OSI e quali sono le responsabilità di ciascun strato?

\textbf{Risposta}:

Il \textbf{modello ISO/OSI (Open Systems Interconnection)} è un modello concettuale che descrive come i sistemi di comunicazione in una rete interagiscono. È diviso in sette strati (layer), ciascuno con responsabilità specifiche:
\begin{enumerate}
    \item \textbf{Strato Fisico (Physical Layer)}: Gestisce la trasmissione di bit grezzi su un mezzo fisico (es. cavi Ethernet).
    \item \textbf{Strato di Collegamento Dati (Data Link Layer)}: Fornisce la trasmissione dati da nodo a nodo, rilevando errori e gestendo l'indirizzamento MAC (es. Ethernet).
    \item \textbf{Strato di Rete (Network Layer)}: Gestisce l'instradamento dei pacchetti (routing) e l'indirizzamento logico (IP).
    \item \textbf{Strato di Trasporto (Transport Layer)}: Fornisce la comunicazione end-to-end tra processi, assicurando la consegna affidabile dei dati, controllo di flusso e congestione (es. TCP, UDP).
    \item \textbf{Strato di Sessione (Session Layer)}: Stabilisce, gestisce e termina le sessioni di comunicazione tra applicazioni.
    \item \textbf{Strato di Presentazione (Presentation Layer)}: Si occupa della sintassi e semantica dei dati, inclusa crittografia/decrittografia e compressione.
    \item \textbf{Strato di Applicazione (Application Layer)}: Fornisce servizi di rete direttamente alle applicazioni dell'utente finale (es. HTTP, FTP).
\end{enumerate}