% Questo file conterrà le domande e gli esercizi per il capitolo "Reti di Calcolatori".
% Sarà incluso nel main.tex subito dopo il riassunto teorico del capitolo.

\section*{Domande e Esercizi} % Usiamo * per non numerare la sezione nell'indice
\addcontentsline{toc}{section}{Domande e Esercizi: Reti di Calcolatori} % Aggiunge la voce all'indice manualmente

\subsection*{Domande d'Esame Principali}
\addcontentsline{toc}{subsection}{Domande d'Esame Principali} % Aggiunge la voce all'indice

% --- Includi qui le singole domande e risposte da file separati ---
% Questo file contiene la domanda e risposta sullo Standard ISO/OSI e Livello di Trasporto.
% Sarà incluso da domande_ed_esercizi_reti.tex.

\subsection*{Domanda: Standard ISO/OSI e Livello di Trasporto}

\textbf{Domanda}: Il candidato dia una panoramica della standard architetturale per reti di calcolatori interoperabili ISO/OSI, approfondendo in particolare il livello di trasporto motivando e spiegando le scelte alla base di alcune tecniche di consegna affidabile del dato.

\paragraph{Risposta}:

\textbf{Panoramica sullo Standard Architetturale ISO/OSI}
Il \textbf{modello ISO/OSI (Open Systems Interconnection)} è un modello concettuale a sette strati (layer) che descrive come i sistemi di comunicazione in una rete interagiscono e cooperano. Ogni strato ha responsabilità specifiche, opera sopra lo strato precedente e fornisce servizi a quello successivo, facilitando l'interoperabilità tra sistemi eterogenei.
La struttura a sette strati è la seguente:
\begin{enumerate}
    \item \textbf{Strato Fisico (Physical Layer)}: Gestisce la trasmissione e ricezione di flussi di bit non strutturati su un mezzo fisico (es. cavi Ethernet, Wi-Fi). Definisce specifiche elettriche e meccaniche.
    \item \textbf{Strato di Collegamento Dati (Data Link Layer)}: Fornisce la trasmissione di dati da nodo a nodo, rilevando e potenzialmente correggendo errori. Gestisce l'indirizzamento MAC e il controllo del flusso (es. Ethernet, PPP).
    \item \textbf{Strato di Rete (Network Layer)}: Gestisce l'instradamento dei pacchetti attraverso la rete (routing) ed è responsabile dell'indirizzamento logico (IP) e della selezione del percorso migliore (es. IP, ICMP).
    \item \textbf{Strato di Trasporto (Transport Layer)}: Fornisce la comunicazione end-to-end tra processi su host diversi. Assicura la consegna affidabile dei dati, il controllo di flusso e il controllo della congestione (es. TCP, UDP).
    \item \textbf{Strato di Sessione (Session Layer)}: Stabilisce, gestisce e termina le sessioni di comunicazione tra applicazioni, gestendo sincronizzazione e dialogo.
    \item \textbf{Strato di Presentazione (Presentation Layer)}: Si occupa della sintassi e della semantica dei dati scambiati, traducendo i dati e gestendo crittografia/decrittografia e compressione.
    \item \textbf{Strato di Applicazione (Application Layer)}: Fornisce servizi di rete direttamente alle applicazioni dell'utente finale (es. HTTP, FTP, SMTP, DNS).
\end{enumerate}

\paragraph{Approfondimento sul Livello di Trasporto e Tecniche di Consegna Affidabile del Dato}
Il Livello di Trasporto è cruciale per stabilire una comunicazione logica tra applicazioni che risiedono su host diversi. I due protocolli principali a questo livello sono TCP (Transmission Control Protocol) e UDP (User Datagram Protocol), con TCP che si distingue per la sua capacità di fornire una consegna affidabile del dato.

\textbf{TCP (Transmission Control Protocol)}:
TCP è un protocollo orientato alla connessione, affidabile e con controllo di flusso e congestione. È la scelta preferita per applicazioni che richiedono garanzie di consegna dei dati.
Le \textbf{scelte alla base della consegna affidabile del dato} in TCP includono:
\begin{itemize}
    \item \textbf{Numerazione dei Segmenti e Riconoscimenti (ACK - Acknowledgement)}: Ogni segmento di dati inviato da TCP è numerato. Il mittente si aspetta un ACK dal destinatario per ogni segmento ricevuto. Se un ACK non viene ricevuto entro un determinato timeout, il segmento viene considerato perso e ritrasmesso. Questo meccanismo garantisce che tutti i dati arrivino a destinazione e nell'ordine corretto, gestendo perdite e duplicazioni.
    \item \textbf{Retrasmissione Selettiva o Go-Back-N}: Se vengono rilevate perdite (es. tramite timeout o ACK duplicati), TCP può ritrasmettere solo i segmenti persi (selettiva) o ritrasmettere dal primo segmento non riconosciuto (Go-Back-N), assicurando la completezza dei dati.
    \item \textbf{Checksum}: TCP calcola un checksum per ogni segmento e lo include nell'header. Il destinatario ricalcola il checksum e lo confronta: se non corrispondono, il segmento viene scartato, garantendo l'integrità dei dati.
    \item \textbf{Controllo di Flusso (Flow Control)}: Impedisce che un mittente veloce sovraccarichi il buffer di un destinatario lento. Il destinatario comunica al mittente la dimensione della sua "finestra di ricezione" (receive window), cioè la quantità di spazio buffer disponibile. Il mittente non invierà più dati di quanto il destinatario possa gestire, prevenendo la perdita di dati dovuta a buffer pieni.
    \item \textbf{Controllo di Congestione (Congestion Control)}: Evita che un mittente invii troppi dati in una rete congestionata, il che potrebbe portare a un collasso della rete. TCP rileva la congestione (es. tramite perdite di pacchetti o ritardi) e riduce dinamicamente la sua velocità di trasmissione, rallentando fino a quando la congestione non diminuisce. Implementa algoritmi come Slow Start, Congestion Avoidance, Fast Retransmit, e Fast Recovery.
\end{itemize}

\textbf{UDP (User Datagram Protocol)}:
In contrasto, UDP è un protocollo senza connessione e inaffidabile ("best-effort"). Non garantisce la consegna, l'ordine o l'assenza di duplicazioni. Ha un overhead minimo ed è utilizzato per applicazioni dove la velocità è più critica dell'affidabilità perfetta (es. streaming multimediale, VoIP, DNS, giochi online), e dove l'applicazione stessa può gestire eventuali perdite o ritrasmissioni.

In sintesi, il Livello di Trasporto offre una scelta tra un servizio affidabile e robusto (TCP) e uno veloce e a basso overhead (UDP), permettendo alle applicazioni di scegliere il protocollo più adatto alle proprie esigenze.

% --- Inserirai qui le future domande/esercizi per Reti di Calcolatori da file separati ---
\subsection*{Altre Possibili Domande}
\addcontentsline{toc}{subsection}{Altre Possibili Domande}
% Questo file contiene la domanda e risposta sul Modello Client/Server e HTTP.
% Sarà incluso da domande_ed_esercizi_reti.tex.

\subsection*{Domanda: Modello Client/Server e Applicazione in HTTP}

\textbf{Domanda}: Come funziona il Modello Client/Server, quali sono i suoi vantaggi e svantaggi, e come si applica nel contesto HTTP?

\textbf{Risposta}:

Il \textbf{modello Client/Server} è un'architettura di rete distribuita in cui i client (richiedenti servizi) e i server (fornitori di servizi) sono entità separate che comunicano su una rete.
\paragraph{Funzionamento}: Il Client invia richieste di servizi al Server, che le elabora e invia risposte. La comunicazione avviene tramite protocolli di rete (es. TCP/IP) su porte specifiche.
\paragraph{Vantaggi}:
\begin{itemize}
    \item \textbf{Centralizzazione}: Gestione centralizzata di dati e risorse (sicurezza, manutenzione).
    \item \textbf{Scalabilità}: Possibilità di scalare il server per gestire più richieste.
    \item \textbf{Manutenzione Facilitata}: Aggiornamenti sul server senza influenzare i client.
\end{itemize}
\paragraph{Svantaggi}:
\begin{itemize}
    \item \textbf{Single Point of Failure}: Se il server si blocca, tutti i client perdono l'accesso.
    \item \textbf{Collo di Bottiglia}: Un server sovraccarico rallenta l'intera rete.
    \item \textbf{Costo}: Server e manutenzione possono essere costosi.
\end{itemize}
\paragraph{Applicazione in HTTP}: L'\textbf{Hypertext Transfer Protocol (HTTP)} è il protocollo applicativo fondamentale per il World Wide Web, che opera nel modello client/server ed è intrinsecamente stateless. Il client (browser) invia richieste HTTP (GET, POST) a un server web, che risponde con lo stato e il contenuto richiesto (es. pagina HTML).
% Questo file contiene la domanda e risposta su Connessioni Stateless/Stateful in HTTP.
% Sarà incluso da domande_ed_esercizi_reti.tex.

\subsection*{Domanda: Connessioni Stateless e Stateful in HTTP}

\textbf{Domanda}: Qual è la differenza tra connessioni stateless e stateful nel contesto HTTP, e quali metodi sono utilizzati per gestire la persistenza dello stato in HTTP?

\textbf{Risposta}:

\paragraph{Connessioni Stateless e Stateful}
\begin{itemize}
    \item \textbf{Stateless Connection (Senza Stato)}: Ogni richiesta dal client al server è indipendente e autocontenuta; il server non ricorda le interazioni passate.
    \begin{itemize}
        \item \textbf{Vantaggi}: Alta scalabilità, resilienza, semplicità lato server.
        \item \textbf{Svantaggi}: Richiede informazioni ridondanti per ogni richiesta.
        \item \textbf{HTTP}: È intrinsecamente stateless.
    \end{itemize}
    \item \textbf{Stateful Connection (Con Stato)}: Il server mantiene e ricorda lo stato delle interazioni passate con un client per un periodo.
    \begin{itemize}
        \item \textbf{Vantaggi}: Meno ridondanza, semplifica logica per sequenze complesse.
        \item \textbf{Svantaggi}: Minore scalabilità, minore resilienza, maggiore complessità.
    \end{itemize}
\end{itemize}
\paragraph{Metodi per Gestire la Persistenza dello Stato in HTTP (per simulare Stateful)}:
Dato che HTTP è stateless, si utilizzano meccanismi come:
\begin{itemize}
    \item \textbf{Cookies}: Piccoli dati inviati dal server al browser, che li memorizza e li invia con richieste successive (per ID sessione, preferenze utente).
    \item \textbf{Session IDs}: Il server crea un ID univoco e lo invia al client (spesso tramite cookie); il server memorizza i dati della sessione.
    \item \textbf{URL Rewriting}: Lo stato incorporato direttamente nell'URL come parametri.
    \item \textbf{Hidden Form Fields}: Dati nascosti in moduli HTML inviati con richieste POST.
    \item \textbf{Web Storage} (Local Storage, Session Storage): API JavaScript per memorizzare dati nel browser del client.
\end{itemize}
% Questo file contiene la domanda e risposta sulla struttura del Modello ISO/OSI.
% Sarà incluso da domande_ed_esercizi_reti.tex.

\subsection*{Domanda: Struttura a Sette Strati del Modello ISO/OSI}

\textbf{Domanda}: Qual è la struttura a sette strati del Modello ISO/OSI e quali sono le responsabilità di ciascun strato?

\textbf{Risposta}:

Il \textbf{modello ISO/OSI (Open Systems Interconnection)} è un modello concettuale che descrive come i sistemi di comunicazione in una rete interagiscono. È diviso in sette strati (layer), ciascuno con responsabilità specifiche:
\begin{enumerate}
    \item \textbf{Strato Fisico (Physical Layer)}: Gestisce la trasmissione di bit grezzi su un mezzo fisico (es. cavi Ethernet).
    \item \textbf{Strato di Collegamento Dati (Data Link Layer)}: Fornisce la trasmissione dati da nodo a nodo, rilevando errori e gestendo l'indirizzamento MAC (es. Ethernet).
    \item \textbf{Strato di Rete (Network Layer)}: Gestisce l'instradamento dei pacchetti (routing) e l'indirizzamento logico (IP).
    \item \textbf{Strato di Trasporto (Transport Layer)}: Fornisce la comunicazione end-to-end tra processi, assicurando la consegna affidabile dei dati, controllo di flusso e congestione (es. TCP, UDP).
    \item \textbf{Strato di Sessione (Session Layer)}: Stabilisce, gestisce e termina le sessioni di comunicazione tra applicazioni.
    \item \textbf{Strato di Presentazione (Presentation Layer)}: Si occupa della sintassi e semantica dei dati, inclusa crittografia/decrittografia e compressione.
    \item \textbf{Strato di Applicazione (Application Layer)}: Fornisce servizi di rete direttamente alle applicazioni dell'utente finale (es. HTTP, FTP).
\end{enumerate}
% Questo file contiene la domanda e risposta sul confronto TCP vs UDP.
% Sarà incluso da domande_ed_esercizi_reti.tex.

\subsection*{Domanda: Confronto tra TCP e UDP}

\textbf{Domanda}: Confronta il protocollo TCP con il protocollo UDP, evidenziando le loro caratteristiche principali, vantaggi, svantaggi e applicazioni tipiche.

\textbf{Risposta}:

Il \textbf{Livello di Trasporto} del modello OSI gestisce la comunicazione end-to-end tra processi. I protocolli principali sono TCP e UDP.
\paragraph{TCP (Transmission Control Protocol)}:
\begin{itemize}
    \item \textbf{Orientato alla Connessione}: Stabilisce una connessione (handshake a tre vie).
    \item \textbf{Affidabile}: Garantisce la consegna ordinata e senza perdite (numerazione, ACK, ritrasmissioni).
    \item \textbf{Controllo di Flusso}: Previene il sovraccarico del destinatario (finestra di ricezione).
    \item \textbf{Controllo di Congestione}: Riduce la velocità in reti congestionate.
    \item \textbf{Applicazioni Tipiche}: Web (HTTP), trasferimento file (FTP), email (SMTP), SSH.
\end{itemize}
\paragraph{UDP (User Datagram Protocol)}:
\begin{itemize}
    \item \textbf{Senza Connessione}: Invia datagrammi indipendenti senza handshake.
    \item \textbf{Inaffidabile (Best-Effort)}: Non garantisce consegna, ordine o assenza di duplicazioni.
    \item \textbf{Nessun Controllo}: Non implementa controllo di flusso o congestione.
    \item \textbf{Overhead Minimo}: Header piccolo, molto efficiente.
    \item \textbf{Applicazioni Tipiche}: Streaming multimediale (VoIP, video), DNS, giochi online (dove la velocità è prioritaria).
\end{itemize}
% Questo file contiene la domanda e risposta sul Subnetting.
% Sarà incluso da domande_ed_esercizi_reti.tex.

\subsection*{Domanda: Concetto e Calcolo del Subnetting}

\textbf{Domanda}: Spiega il concetto di subnetting, i suoi obiettivi e illustra con un esempio pratico come si calcolano le sottoreti per una data rete IP.

\textbf{Risposta}:

Il \textbf{subnetting} è il processo di divisione di una rete IP più grande in sottoreti più piccole e gestibili, per migliorare efficienza, sicurezza e gestione degli indirizzi IP.
\paragraph{Obiettivi}:
\begin{itemize}
    \item \textbf{Efficienza nell'uso degli indirizzi IP}: Sfruttare meglio lo spazio IP.
    \item \textbf{Riduzione del traffico di rete}: Confinare il traffico di broadcast.
    \item \textbf{Miglioramento della Sicurezza}: Isolare segmenti di rete.
    \item \textbf{Facilitazione della Gestione}: Reti più piccole più facili da gestire.
\end{itemize}
Un indirizzo IP (IPv4) ha 32 bit (parte di rete + parte host). La \textbf{maschera di sottorete} (subnet mask) separa queste due parti. La notazione \textbf{CIDR} (\lstinline{IP/prefisso}) indica la lunghezza del prefisso di rete.
\paragraph{Calcolo del Subnetting (Esempio Pratico)}:
\textbf{Scenario}: Una rete \lstinline{192.168.1.0/24} deve essere divisa in 4 sottoreti per ospitare almeno 50 host per sottorete.
\begin{enumerate}
    \item \textbf{Bit per sottoreti}: $2^n \ge 4 \Rightarrow n=2$ bit.
    \item \textbf{Bit per host}: Maschera originale /24 (8 bit host). Usando 2 bit per sottoreti, rimangono $8-2=6$ bit per host. Max host per sottorete: $2^6-2=62$ host (sufficiente).
    \item \textbf{Nuova maschera}: $24+2=26$. Maschera \lstinline{/26} (\lstinline{255.255.255.192}).
    \item \textbf{Sottoreti valide} (intervalli di 64 nell'ultimo ottetto):
    \begin{itemize}
        \item Sottorete 1: \lstinline{192.168.1.0/26} (Host: .1 a .62, Broadcast: .63)
        \item Sottorete 2: \lstinline{192.168.1.64/26} (Host: .65 a .126, Broadcast: .127)
        \item Sottorete 3: \lstinline{192.168.1.128/26} (Host: .129 a .190, Broadcast: .191)
        \item Sottorete 4: \lstinline{192.168.1.192/26} (Host: .193 a .254, Broadcast: .255)
    \end{itemize}
\end{enumerate}
% \subsection*{Esercizi}
% \addcontentsline{toc}{subsection}{Esercizi}
% \input{capitoli/reti_di_calcolatori/esercizi/reti_esercizio_routing.tex}