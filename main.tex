\documentclass[a4paper,12pt,openright,twoside]{book} % Usiamo 'book' per la struttura con capitoli

% --- Caricamento dei tuoi file di stile personalizzati ---
\usepackage{stili/mio_stile_esame} % Contiene i pacchetti generali e le impostazioni di base
\usepackage{stili/impostazioni_codice} % Contiene le definizioni per listings (pseudocodice, SQL, Bash)

% --- Informazioni sul documento ---
\title{Appunti per l'Esame di Stato\\Ingegnere Junior - Settore Informazione}
\author{Paolo Pietrelli}
\date{\today} % Usa la data corrente
% \subject{Preparazione Integrata} % Campo soggetto per il PDF (opzionale, già in hyperref)

% --- Bibliografia (già gestita in mio_stile_esame.sty, qui solo il comando per stampare) ---
% \printbibliography % Questo comando andrà nella sezione \backmatter

\begin{document}
\sloppy % Rende la composizione un po' meno rigida, utile per evitare overfull hbox con testo lungo

\frontmatter % Pagine preliminari (numerazione romana, senza capitoli numerati)

\maketitle % Stampa titolo, autore, data

% Se vuoi un abstract, puoi inserirlo qui:
% \begin{abstract}
% Breve riassunto del contenuto complessivo dei tuoi appunti.
% \end{abstract}

\tableofcontents % Genera l'indice dei capitoli e delle sezioni
\listoffigures % Genera la lista delle figure
\lstlistoflistings % Genera la lista dei listati di codice (se usi newfloat=true in lstset, altrimenti usa listof=tof)

\mainmatter % Inizio del contenuto principale (numerazione araba, capitoli numerati)

\maketitle

\cleardoublepage
\chapter*{Collaboratori}
Questo documento è stato sviluppato con la collaborazione di:

\begin{itemize}
  \item Paolo Pietrelli – autore principale
 % \item Nome Cognome – contributo su Basi di Dati
 % \item Nome Cognome – contributo su Reti di Calcolatori
\end{itemize}

\cleardoublepage


% --- Inclusione dei Capitoli Principali e delle relative Domande/Esercizi ---
% Ogni riga \include farà iniziare un nuovo capitolo in una nuova pagina.
% Le domande e gli esercizi di quel capitolo verranno inclusi subito dopo, se il file esiste.

% Capitolo 1: Sistemi Operativi
\chapter{Sistemi Operativi}

Un \textbf{sistema operativo (SO)} è un software di sistema che gestisce le risorse hardware e software di un computer e fornisce servizi comuni per i programmi del computer e per l'utente. È l'interfaccia tra l'hardware e l'utente/applicazioni. La sua importanza risiede nell'astrazione dell'hardware, nella gestione efficiente delle risorse e nell'esecuzione controllata dei programmi.

\section{Struttura e Organizzazione del Sistema Operativo}
Un sistema operativo è un'entità complessa, ma può essere scomposto in componenti modulari che cooperano per fornire un ambiente funzionale per l'esecuzione dei programmi.

\subsection{Componenti Principali}
I principali componenti di un sistema operativo includono:
\begin{itemize}
    \item \textbf{Kernel}: Il cuore del SO, responsabile della gestione dei processi (creazione, scheduling, terminazione, comunicazione interprocesso), della memoria (allocazione, protezione, gestione della memoria virtuale), dei file system (gestione dei file e delle directory, allocazione dello spazio su disco) e dell'I/O (gestione dei dispositivi di input/output, driver).
    \item \textbf{Gestore dei Processi (Process Management)}: Si occupa della creazione, terminazione, sospensione e ripristino dei processi, e della gestione dei loro stati (pronto, in esecuzione, in attesa).
    \item \textbf{Gestore della Memoria (Memory Management)}: Responsabile dell'allocazione e deallocazione della memoria ai processi, della gestione della memoria virtuale (paginazione, segmentazione) e della protezione della memoria per evitare interferenze tra i processi.
    \item \textbf{File System Management}: Organizza e gestisce i dati su dispositivi di archiviazione, controllando l'accesso e la protezione dei file e delle directory.
    \item \textbf{Gestore I/O (I/O Management)}: Fornisce un'interfaccia standardizzata per interagire con i dispositivi hardware (stampanti, tastiere, dischi) tramite driver specifici.
    \item \textbf{Network Management}: Gestisce le comunicazioni di rete e i protocolli di comunicazione.
    \item \textbf{Security and Protection}: Implementa meccanismi per proteggere le risorse del sistema e i dati degli utenti da accessi non autorizzati o malfunzionamenti.
    \item \textbf{Interfaccia Utente (User Interface)}: Può essere una GUI (Graphical User Interface) con elementi visivi o una CLI (Command Line Interface) basata su testo, permettendo all'utente di interagire con il sistema.
\end{itemize}

\subsection{Modelli di Sistemi Operativi}
I sistemi operativi possono essere strutturati secondo diversi modelli architetturali:
\begin{itemize}
    \item \textbf{Monolitici}: Tutti i servizi del SO risiedono nello stesso spazio di indirizzamento (kernel space).
    \begin{itemize}
        \item \textbf{Vantaggi}: Alta performance grazie al minimo overhead di comunicazione.
        \item \textbf{Svantaggi}: Difficili da debuggare, poco flessibili, un crash di un componente può bloccare l'intero sistema.
        \item \textbf{Esempio}: Linux, Unix (tradizionali).
    \end{itemize}
    \item \textbf{Layered (a Strati)}: Il SO è diviso in strati, ognuno dei quali offre servizi allo strato superiore e utilizza servizi dallo strato inferiore.
    \begin{itemize}
        \item \textbf{Vantaggi}: Modularità, facilità di debug e manutenzione.
        \item \textbf{Svantaggi}: Performance ridotte a causa dell'overhead di comunicazione tra strati.
        \item \textbf{Esempio}: THE (Dijkstra).
    \end{itemize}
    \item \textbf{Microkernel}: Solo i servizi essenziali (gestione processi, gestione memoria base, comunicazione interprocesso) risiedono nel kernel (microkernel). Altri servizi (file system, driver, network) sono implementati come processi utente (server).
    \begin{itemize}
        \item \textbf{Vantaggi}: Modularità, robustezza (un crash di un server non blocca il sistema), flessibilità.
        \item \textbf{Svantaggi}: Performance potenzialmente più basse a causa di più cambi di contesto.
        \item \textbf{Esempio}: Mach (base per macOS), QNX.
    \end{itemize}
    \item \textbf{Modulari (o Ibridi)}: Un approccio intermedio che combina le migliori caratteristiche dei modelli monolitici e microkernel. Permettono il caricamento dinamico dei moduli kernel (es. driver) senza richiedere un riavvio completo del sistema.
    \begin{itemize}
        \item \textbf{Esempio}: Versioni moderne di Linux, Windows.
    \end{itemize}
\end{itemize}

\subsection{Processi e Thread}
Nei sistemi operativi moderni, l'esecuzione dei programmi è gestita attraverso due concetti principali: i processi e i thread.

\subsubsection{Processo (Processo "Pesante")}
Un \textbf{processo} è un'istanza di un programma in esecuzione. È un'unità di allocazione delle risorse del sistema operativo e include:
\begin{itemize}
    \item Il codice del programma.
    \item I dati (variabili globali, heap).
    \item Lo stack (per le chiamate di funzione e le variabili locali).
    \item Il Program Counter (PC) e i registri della CPU.
    \item Risorse del sistema operativo allocate (file aperti, segnali, memoria, ecc.).
\end{itemize}
Ogni processo ha il proprio spazio di indirizzamento virtuale separato, il che fornisce isolamento e protezione tra i processi. Se un processo crasha, di solito non influisce sugli altri processi. I processi sono considerati "pesanti" a causa dell'overhead associato alla loro creazione, alla distruzione e al cambio di contesto.

\subsubsection{Thread (Processo "Leggero")}
Un \textbf{thread} (o thread di esecuzione) è un'unità di esecuzione all'interno di un processo. Un singolo processo può contenere più thread. I thread all'interno dello stesso processo condividono lo stesso spazio di indirizzamento virtuale, il codice del programma, i dati globali e le risorse del sistema operativo (file aperti, ecc.). Ogni thread ha il proprio:
\begin{itemize}
    \item Program Counter (PC).
    \item Set di registri della CPU.
    \item Stack separato.
\end{itemize}
I thread sono considerati "leggeri" perché la loro creazione, distruzione e il cambio di contesto sono molto più veloci ed efficienti rispetto ai processi, dato che non richiedono la creazione di un nuovo spazio di indirizzamento e la copia di risorse.

\subsubsection{Punti in Comune e Differenze}
\begin{itemize}
    \item \textbf{Punti in Comune}: Sia processi che thread rappresentano unità di esecuzione che possono essere schedulate dalla CPU. Entrambi hanno un Program Counter, un set di registri e uno stack.
    \item \textbf{Differenze Principali}:
    \begin{itemize}
        \item \textbf{Isolamento delle Risorse}: I processi hanno spazi di indirizzamento separati e risorse isolate. I thread all'interno dello stesso processo condividono lo spazio di indirizzamento e le risorse.
        \item \textbf{Overhead}: I processi hanno un overhead maggiore per creazione, distruzione e cambio di contesto. I thread sono più "leggeri".
        \item \textbf{Comunicazione}: La comunicazione tra processi (IPC - Inter-Process Communication) è più complessa (richiede meccanismi espliciti come pipe, shared memory, message queues). La comunicazione tra thread (ITC - Inter-Thread Communication) è più semplice e diretta, poiché condividono la memoria.
        \item \textbf{Robustezza}: Un crash di un thread può compromettere l'intero processo (e tutti i suoi thread). Un crash di un processo non influisce sugli altri processi.
    \end{itemize}
\end{itemize}

\section{Record di Attivazione (Activation Record / Stack Frame)}
Il \textbf{Record di Attivazione} (o \textit{Stack Frame}) è una struttura dati creata sullo stack del programma ogni volta che una funzione (o procedura, o subroutine) viene chiamata. Contiene tutte le informazioni necessarie per la gestione dell'esecuzione di quella specifica chiamata di funzione. È una componente fondamentale del runtime di un linguaggio di programmazione e del supporto fornito dal sistema operativo per l'esecuzione di programmi che utilizzano stack per le chiamate a funzione.

\subsection{Contenuto di un Record di Attivazione}
Un tipico record di attivazione può contenere le seguenti informazioni, sebbene la loro esatta disposizione e i dettagli possano variare a seconda dell'architettura della CPU, del sistema operativo e del compilatore:
\begin{itemize}
    \item \textbf{Parametri Attuali (Actual Parameters)}: I valori degli argomenti passati alla funzione.
    \item \textbf{Indirizzo di Ritorno (Return Address)}: L'indirizzo di memoria della istruzione nel codice chiamante a cui il controllo deve tornare una volta che la funzione corrente ha terminato l'esecuzione.
    \item \textbf{Valore di Ritorno (Return Value)}: Uno spazio per memorizzare il valore restituito dalla funzione (se la funzione restituisce un valore).
    \item \textbf{Variabili Locali (Local Variables)}: Le variabili dichiarate all'interno della funzione e che esistono solo per la durata di quella specifica attivazione.
    \item \textbf{Stato dei Registri Salvati (Saved Register State)}: I valori dei registri della CPU che erano in uso dalla funzione chiamante e che devono essere ripristinati prima del ritorno.
    \item \textbf{Puntatore al Frame Precedente (Control Link / Dynamic Link)}: Un puntatore (indirizzo) al record di attivazione della funzione chiamante, permettendo al sistema di "risalire" lo stack.
    \item \textbf{Puntatore al Contesto Statico (Access Link / Static Link)}: (Solo per linguaggi con scope annidato statico, come Pascal o linguaggi funzionali) Un puntatore al record di attivazione della funzione che definisce lo scope lessicale della funzione corrente, permettendo l'accesso a variabili non locali.
\end{itemize}

\subsection{Ciclo di Vita di un Record di Attivazione}
Il ciclo di vita di un record di attivazione segue il flusso delle chiamate a funzione e la struttura dello stack:
\begin{enumerate}
    \item \textbf{Creazione (Chiamata di Funzione)}: Quando una funzione viene chiamata, il codice chiamante (o il runtime) crea un nuovo record di attivazione. Questo nuovo record viene "pushed" (inserito) in cima allo stack di esecuzione del processo. Il puntatore dello stack (Stack Pointer) viene aggiornato per puntare a questo nuovo frame.
    \item \textbf{Esecuzione}: La CPU salta all'indirizzo di inizio del codice della funzione chiamata. La funzione utilizza le informazioni nel suo record di attivazione (parametri, variabili locali, registri) durante la sua esecuzione.
    \item \textbf{Terminazione (Ritorno da Funzione)}: Quando la funzione termina (raggiunge un `return` o la fine del suo blocco di codice), il valore di ritorno (se presente) viene posizionato in una posizione designata (spesso un registro). Il record di attivazione corrente viene "popped" (rimosso) dallo stack, liberando lo spazio di memoria che occupava. Il puntatore dello stack viene ripristinato al record di attivazione precedente.
    \item \textbf{Ripristino del Contesto}: Il controllo del programma viene trasferito all'indirizzo di ritorno memorizzato nel record di attivazione appena rimosso, e i registri salvati vengono ripristinati per continuare l'esecuzione della funzione chiamante.
\end{enumerate}
Questo meccanismo a stack garantisce che le chiamate a funzione e i ritorni avvengano in modo LIFO (Last-In, First-Out), permettendo la corretta gestione delle funzioni annidate e ricorsive.

\section{Gestione della Memoria}
La \textbf{gestione della memoria} è una delle funzioni fondamentali del sistema operativo, responsabile di allocare e deallocare la memoria ai processi, di proteggere la memoria per evitare interferenze tra i processi e di fornire un'astrazione della memoria fisica per gli sviluppatori. L'obiettivo principale è massimizzare l'utilizzo della CPU mantenendo molti processi in memoria e fornendo un meccanismo efficiente per accedere ai dati.

\subsection{Memoria Virtuale}
La \textbf{memoria virtuale} è una tecnica che permette a un sistema operativo di compensare la carenza di memoria fisica (RAM) usando la memoria secondaria (spazio su disco) per simulare più RAM. Questo permette ai programmi di usare più memoria di quella fisicamente disponibile e facilita il multitasking, isolando lo spazio di indirizzamento di ogni processo. Le tecniche principali per implementare la memoria virtuale sono la paginazione e la segmentazione.

\subsubsection{Paginazione (Paging)}
La paginazione è una tecnica di gestione della memoria virtuale che suddivide lo spazio di indirizzamento logico di un processo in blocchi di dimensione fissa chiamati \textbf{pagine}. La memoria fisica è anch'essa divisa in blocchi della stessa dimensione, chiamati \textbf{frame}.
\begin{itemize}
    \item \textbf{Funzionamento}: Quando un processo viene caricato, le sue pagine possono essere caricate in frame non contigui della memoria fisica. La traduzione dell'indirizzo logico (virtuale) in indirizzo fisico avviene tramite una \textbf{Page Table} (Tabella delle Pagine). Ogni processo ha la propria Page Table, che mappa le pagine logiche ai frame fisici.
    \item \textbf{Traduzione dell'Indirizzo}: L'indirizzo logico è diviso in due parti: il numero di pagina (page number, p) e l'offset all'interno della pagina (offset, d). La Page Table usa il numero di pagina per trovare il frame corrispondente, e l'offset viene aggiunto al frame address per ottenere l'indirizzo fisico.
    \item \textbf{TLB (Translation Lookaside Buffer)}: Una cache hardware veloce che memorizza le mappature pagina-frame usate più frequentemente per accelerare il processo di traduzione degli indirizzi e compensare il fatto che ogni accesso alla memoria richiede due accessi (uno alla Page Table e uno al dato).
    \item \textbf{Vantaggi}:
    \begin{itemize}
        \item Elimina la \textbf{frammentazione esterna} (non ci sono buchi inutilizzati tra i blocchi allocati).
        \item Semplifica l'allocazione della memoria.
    \end{itemize}
    \item \textbf{Svantaggi}:
    \begin{itemize}
        \item Introduce la \textbf{frammentazione interna} (il frame finale potrebbe non essere completamente riempito dalla pagina, lasciando spazio inutilizzato).
        \item Overhead della Page Table (può essere grande e richiede memoria).
    \end{itemize}
\end{itemize}

\subsubsection{Segmentazione (Segmentation)}
La segmentazione è una tecnica di gestione della memoria che permette di visualizzare la memoria come una collezione di segmenti di dimensione variabile. Ogni segmento corrisponde a un'unità logica del programma (es. codice, dati, stack, subroutine).
\begin{itemize}
    \item \textbf{Funzionamento}: L'indirizzo logico è composto da un numero di segmento e un offset. La \textbf{Segment Table} (Tabella dei Segmenti) memorizza l'indirizzo base e la lunghezza di ciascun segmento in memoria fisica. La traduzione avviene verificando che l'offset non superi la lunghezza del segmento e aggiungendo l'offset all'indirizzo base del segmento.
    \item \textbf{Vantaggi}:
    \begin{itemize}
        \item Riflette la visione logica del programma (moduli, funzioni).
        \item Facilita la protezione e la condivisione di segmenti tra processi.
    \end{itemize}
    \item \textbf{Svantaggi}:
    \begin{itemize}
        \item Soffre di \textbf{frammentazione esterna} (spazi liberi tra i segmenti che possono essere troppo piccoli per nuove allocazioni).
        \item La gestione delle dimensioni variabili dei segmenti è più complessa.
    \end{itemize}
\end{itemize}

\subsection{Algoritmi di Allocazione della Memoria (per Blocchi Contigui)}
Quando la memoria fisica viene allocata in blocchi contigui (come nella segmentazione o senza memoria virtuale), il sistema operativo deve decidere quale blocco libero assegnare a una richiesta di memoria.
\begin{itemize}
    \item \textbf{First Fit}:
    \begin{itemize}
        \item \textbf{Descrizione}: Alloca il primo blocco libero che trova che sia abbastanza grande da soddisfare la richiesta.
        \item \textbf{Vantaggi}: Veloce da implementare, spesso porta a buoni risultati.
        \item \textbf{Svantaggi}: Può lasciare molti piccoli blocchi liberi all'inizio della lista, o frammentare blocchi grandi.
    \end{itemize}
    \item \textbf{Best Fit}:
    \begin{itemize}
        \item \textbf{Descrizione}: Alloca il blocco libero più piccolo che sia comunque sufficientemente grande a soddisfare la richiesta.
        \item \textbf{Vantaggi}: Lascia i blocchi liberi più grandi intatti per richieste future, riducendo la frammentazione interna.
        \item \textbf{Svantaggi}: Richiede una scansione completa della lista dei blocchi liberi (o una ricerca in una lista ordinata), quindi è più lento. Genera molti blocchi liberi molto piccoli.
    \end{itemize}
    \item \textbf{Worst Fit}:
    \begin{itemize}
        \item \textbf{Descrizione}: Alloca il blocco libero più grande disponibile.
        \item \textbf{Vantaggi}: Si cerca di lasciare un blocco residuo il più grande possibile dopo l'allocazione, teoricamente utile per richieste future.
        \item \textbf{Svantaggi}: Spesso è l'algoritmo peggiore in termini di frammentazione. Rende la ricerca più lenta e frammenta i blocchi più grandi.
    \end{itemize}
\end{itemize}

\section{Gestione dei File System}
La \textbf{gestione dei file system} è un componente fondamentale del sistema operativo, responsabile di organizzare, archiviare e recuperare i dati su dispositivi di memoria secondaria (come dischi rigidi, SSD). Fornisce un'interfaccia logica per l'utente e le applicazioni per interagire con i dati, astraendo i dettagli di basso livello dell'hardware di archiviazione.

\subsection{Organizzazione Fisica dei Dati su Disco}
La memoria secondaria è tipicamente organizzata in unità di archiviazione fisiche che il file system gestisce.
\begin{itemize}
    \item \textbf{Settori}: La più piccola unità fisica di memorizzazione su un disco.
    \item \textbf{Blocchi (o Cluster)}: L'unità di trasferimento logica più piccola riconosciuta dal file system. Un blocco è composto da uno o più settori contigui. La dimensione del blocco è un compromesso: blocchi grandi riducono l'overhead di I/O ma aumentano la frammentazione interna; blocchi piccoli aumentano l'overhead di I/O ma riducono la frammentazione interna.
    \item \textbf{Cilindri/Tracce}: Concetti fisici legati ai dischi rotanti (hard disk), dove le tracce sono anelli concentrici e i cilindri sono l'insieme di tracce alla stessa distanza dal centro su tutti i piatti.
\end{itemize}

\subsubsection{Struttura delle Directory}
Le directory sono strutture che organizzano i file e altre directory in una gerarchia.
\begin{itemize}
    \item \textbf{Struttura a Singolo Livello}: Tutti i file sono nella stessa directory. Semplice, ma difficile da gestire per molti utenti/file.
    \item \textbf{Struttura a Due Livelli}: Ogni utente ha la propria directory principale, separata dagli altri. Impedisce conflitti di nomi tra utenti, ma non offre organizzazione interna.
    \item \textbf{Struttura Ad Albero (Gerarchica)}: La struttura più comune, con una directory radice e sottodirectory. Offre flessibilità e organizzazione.
    \item \textbf{Grafo Aciclico Diretto (DAG)}: Permette la condivisione di file e directory tra diversi percorsi (tramite link hard o soft/symbolic).
\item \textbf{Implementazione Interna}: Le directory possono essere implementate come liste di voci (nome file, puntatore a i-node/FAT entry) o tabelle hash per una ricerca più veloce.
\end{itemize}

\subsection{Metodi di Allocazione dei File}
Il metodo di allocazione determina come i blocchi di un file sono memorizzati e gestiti sul disco, influenzando l'efficienza di accesso e la gestione dello spazio.

\subsubsection{Allocazione Contigua}
\begin{itemize}
    \item \textbf{Descrizione}: Ogni file è memorizzato come un blocco contiguo di blocchi sul disco. Il file system memorizza solo l'indirizzo del primo blocco e la lunghezza del file.
    \item \textbf{Vantaggi}:
    \begin{itemize}
        \item Semplice da implementare.
        \item Ottime prestazioni per l'accesso sequenziale e diretto (un solo seek per leggere un file intero, accesso diretto immediato a qualsiasi blocco).
    \end{itemize}
    \item \textbf{Svantaggi}:
    \begin{itemize}
        \item \textbf{Frammentazione Esterna}: Con il tempo, lo spazio libero sul disco si frammenta in piccoli buchi non contigui, rendendo difficile trovare blocchi contigui grandi per nuovi file, anche se c'è molto spazio totale disponibile.
        \item Difficile prevedere la dimensione finale dei file.
        \item Richiede compattazione periodica per recuperare spazio contiguo (operazione costosa).
    \end{itemize}
\end{itemize}

\subsubsection{Allocazione Collegata (Linked Allocation)}
\begin{itemize}
    \item \textbf{Descrizione}: Ogni file è una lista collegata di blocchi su disco. Ogni blocco contiene un puntatore al blocco successivo. Il file system memorizza solo l'indirizzo del primo blocco.
    \item \textbf{Vantaggi}:
    \begin{itemize}
        \item Nessuna frammentazione esterna.
        \item Allocazione dinamica e facile espansione dei file.
    \end{itemize}
    \item \textbf{Svantaggi}:
    \begin{itemize}
        \item Lento per l'accesso diretto/casuale (bisogna seguire la catena di puntatori).
        \item Spazio sprecato per i puntatori in ogni blocco (anche se questo è spesso mitigato da una File Allocation Table - FAT).
        \item Rischio di perdita dell'intera catena se un puntatore è corrotto.
    \end{itemize}
\end{itemize}

\subsubsection{Allocazione Indicizzata (Indexed Allocation)}
\begin{itemize}
    \item \textbf{Descrizione}: Ogni file ha un blocco indice dedicato, che è un array di puntatori ai blocchi di dati effettivi del file. Il file system memorizza solo l'indirizzo del blocco indice.
    \item \textbf{Vantaggi}:
    \begin{itemize}
        \item Combina i vantaggi dell'allocazione contigua e collegata: supporta sia l'accesso sequenziale che diretto.
        \item Nessuna frammentazione esterna.
        \item Facile espansione (aggiungendo puntatori al blocco indice o blocchi indice multipli/gerarchici).
    \end{itemize}
    \item \textbf{Svantaggi}:
    \begin{itemize}
        \item Overhead dello spazio per i blocchi indice.
        \item La dimensione del file è limitata dalla dimensione del blocco indice.
    \end{itemize}
    \item \textbf{Implementazione Comune}: \textbf{i-nodes (Unix/Linux)}: Ogni file ha un i-node che contiene metadati del file e un array di puntatori ai blocchi di dati, inclusi puntatori a blocchi indiretti per file di grandi dimensioni.
\textbf{FAT (File Allocation Table - MS-DOS/Windows)}: Una tabella sul disco che memorizza le catene di blocchi per ogni file. Non è esattamente un blocco indice per file singoli, ma un array centrale di puntatori.
\end{itemize}

\section{Sincronizzazione e Deadlock}
Nei sistemi operativi multi-programmati o multi-thread, dove più processi o thread condividono risorse e dati, è fondamentale garantire la \textbf{sincronizzazione} per mantenere la coerenza dei dati e prevenire condizioni di errore come le race condition. La mancanza di sincronizzazione può portare anche a situazioni di \textbf{deadlock}.

\subsection{Problemi di Sincronizzazione}
\begin{itemize}
    \item \textbf{Race Condition (Condizione di Corsa)}: Si verifica quando più processi o thread accedono e manipolano dati condivisi concorrentemente, e il risultato finale dell'esecuzione dipende dall'ordine in cui le operazioni dei processi/thread si intersecano. Il risultato non è predicibile.
    \item \textbf{Sezione Critica (Critical Section)}: Una sezione di codice in cui un processo o thread accede a risorse condivise (variabili, file, periferiche). L'obiettivo della sincronizzazione è garantire che, in un dato istante, al massimo un processo/thread sia nella sua sezione critica per quella risorsa.
    \item \textbf{Problema della Sezione Critica (Critical Section Problem)}: Progettare protocolli per garantire che i processi cooperanti accedano alle sezioni critiche in modo sicuro, rispettando le seguenti condizioni:
    \begin{itemize}
        \item \textbf{Mutua Esclusione (Mutual Exclusion)}: Se un processo è nella sua sezione critica, nessun altro processo può entrare nella propria sezione critica.
        \item \textbf{Progresso (Progress)}: Se nessun processo è nella sezione critica e alcuni processi vogliono entrare, solo quei processi che non sono nella loro sezione remainder possono partecipare alla decisione di quale processo entrerà nella sezione critica successiva, e questa decisione non deve essere ritardata indefinitamente.
        \item \textbf{Attesa Limitata (Bounded Waiting)}: Esiste un limite al numero di volte in cui altri processi possono entrare nelle loro sezioni critiche dopo che un processo ha richiesto di entrare nella sua sezione critica e prima che la sua richiesta venga soddisfatta. Questo previene la starvation.
    \end{itemize}
\end{itemize}

\subsection{Meccanismi di Sincronizzazione}
Diversi strumenti e meccanismi sono stati sviluppati per risolvere il problema della sezione critica:
\begin{itemize}
    \item \textbf{Lock}: Un meccanismo semplice che permette di acquisire e rilasciare un blocco su una risorsa. Prima di entrare nella sezione critica, un processo acquisisce il lock; dopo essere uscito, lo rilascia. Se il lock è già acquisito, il processo deve attendere.
    \item \textbf{Semafori}: Un tipo di variabile intera (generalmente non negativa) a cui si può accedere solo tramite due operazioni atomiche:
    \begin{itemize}
        \item \textbf{\lstinline{wait()}} (o \lstinline{P()}): Decrementa il valore del semaforo. Se il valore diventa negativo, il processo viene bloccato.
        \item \textbf{\lstinline{signal()}} (o \lstinline{V()}): Incrementa il valore del semaforo. Se ci sono processi bloccati sul semaforo, uno di essi viene sbloccato.
        \item \textbf{Semafori Binari (Mutex Lock)}: Valori 0 o 1. Usati per la mutua esclusione.
        \item \textbf{Semafori Contatori}: Valori interi. Usati per gestire l'accesso a un numero finito di risorse.
    \end{itemize}
    \item \textbf{Esempio di Problema Produttore-Consumatore con Semafori}:
    Un problema classico di sincronizzazione dove un produttore genera dati e li inserisce in un buffer, mentre un consumatore preleva dati dal buffer.
    \begin{lstlisting}[language=Pseudocode, numbers=none]
// Semafori:
// empty: contatore (inizializzato a N, dimensione buffer) -> numero di slot vuoti
// full: contatore (inizializzato a 0) -> numero di slot pieni
// mutex: binario (inizializzato a 1) -> mutua esclusione per accesso al buffer

Producer:
  LOOP
    produce item
    CALL wait(empty)
    CALL wait(mutex)
    add item to buffer
    CALL signal(mutex)
    CALL signal(full)
  END LOOP

Consumer:
  LOOP
    CALL wait(full)
    CALL wait(mutex)
    remove item from buffer
    CALL signal(mutex)
    CALL signal(empty)
    consume item
  END LOOP
    \end{lstlisting}
    \item \textbf{Monitor}: Un costrutto di sincronizzazione di alto livello che incapsula dati condivisi e le procedure che li manipolano. Garantisce che, in ogni momento, al massimo un processo possa essere attivo all'interno del monitor. Utilizza \textbf{variabili di condizione} con operazioni `wait()` e `signal()` per sospendere e riattivare i processi che devono attendere specifiche condizioni sui dati condivisi.
\end{itemize}

\subsection{Deadlock (Interblocco)}
Il \textbf{deadlock} è una situazione in cui due o più processi sono bloccati indefinitamente, in attesa di una risorsa detenuta da un altro processo bloccato.

\subsubsection{Condizioni Necessarie per il Deadlock (Condizioni di Coffman)}
Il deadlock può verificarsi se e solo se tutte e quattro le seguenti condizioni sono presenti contemporaneamente:
\begin{enumerate}
    \item \textbf{Mutua Esclusione (Mutual Exclusion)}: Almeno una risorsa deve essere tenuta in modalità non-condivisibile, cioè al massimo un processo può usarla alla volta.
    \item \textbf{Attesa e Mantenimento (Hold and Wait)}: Un processo deve detenere almeno una risorsa ed essere in attesa di acquisirne altre attualmente detenute da altri processi.
    \item \textbf{Non-Preemptive (Nessuna Preemption)}: Le risorse non possono essere sottratte a un processo che le detiene; possono essere rilasciate solo volontariamente dal processo che le detiene.
    \item \textbf{Attesa Circolare (Circular Wait)}: Deve esistere una catena circolare di due o più processi, in cui ogni processo in attesa nella catena sta aspettando una risorsa detenuta dal processo successivo nella catena.
\end{enumerate}

\subsubsection{Strategie di Gestione del Deadlock}
Per affrontare il deadlock, i sistemi operativi possono adottare diverse strategie:
\begin{itemize}
    \item \textbf{Prevenzione del Deadlock (Deadlock Prevention)}:
        \begin{itemize} % Inizio del sub-itemize
            \item Obiettivo: Garantire che almeno una delle quattro condizioni necessarie non si verifichi.
            \item Come: Negando una o più condizioni (es. non permettere l'attesa e mantenimento, assegnare tutte le risorse all'inizio).
            \item Svantaggi: Spesso porta a un basso utilizzo delle risorse e a un throughput ridotto.
        \end{itemize} % Fine del sub-itemize
    \item \textbf{Evitamento del Deadlock (Deadlock Avoidance)}:
        \begin{itemize} % Inizio del sub-itemize
            \item Obiettivo: Richiede che il sistema abbia informazioni a priori sulle risorse che un processo richiederà. Il sistema verifica se lo stato corrente è "sicuro" (se esiste una sequenza di esecuzione dei processi che eviterà il deadlock).
            \item Algoritmo Esempio: Algoritmo del banchiere.
            \item Vantaggi: Meno restrittivo della prevenzione, migliore utilizzo delle risorse.
            \item Svantaggi: Richiede conoscenza a priori, può essere computazionalmente costoso.
        \end{itemize} % Fine del sub-itemize
    \item \textbf{Rilevamento e Ripristino del Deadlock (Deadlock Detection and Recovery)}:
        \begin{itemize} % Inizio del sub-itemize
            \item Obiettivo: Permettere che il deadlock si verifichi, rilevarlo e poi ripristinare il sistema da esso.
            \item Rilevamento: Si usa un algoritmo di rilevamento del ciclo nel grafo di allocazione delle risorse.
            \item Ripristino:
                \begin{itemize} % Questo è un sub-sub-itemize
                    \item Terminazione del processo: Terminare uno o più processi coinvolti nel deadlock.
                    \item Preemption della risorsa: Sottrarre risorse a un processo e assegnarle a un altro.
                \end{itemize} % Fine del sub-sub-itemize
            \item Svantaggi: Comporta un overhead per il rilevamento e la perdita di lavoro per il ripristino.
        \end{itemize} % Fine del sub-itemize
    \item \textbf{Ignorare il Problema}:
        \begin{itemize} % Inizio del sub-itemize
            \item Questo è l'approccio più comune in molti sistemi operativi (es. Unix/Linux, Windows), assumendo che il deadlock sia un evento raro e che sia meno costoso lasciarlo gestire all'amministratore (riavvio del sistema) piuttosto che implementare algoritmi complessi.
        \end{itemize} % Fine del sub-itemize
\end{itemize}

\section{Scheduling della CPU}
Lo \textbf{scheduling della CPU} è l'attività di selezionare quale processo, tra quelli pronti per l'esecuzione, deve essere assegnato alla CPU in un dato momento. È una funzione fondamentale del sistema operativo che ha un impatto cruciale sulle performance complessive del sistema, influenzando parametri come il throughput (numero di processi completati per unità di tempo), il tempo di risposta (tempo tra richiesta e prima risposta), e l'equità nella distribuzione delle risorse della CPU tra i processi.

\subsection{Principali Problematiche dello Scheduling}
Lo scheduling deve affrontare diverse sfide e problematiche per bilanciare l'efficienza e l'equità:
\begin{itemize}
    \item \textbf{Ottimizzazione degli obiettivi}: Bilanciare metriche contrastanti come massimizzare il throughput, minimizzare il tempo di risposta, minimizzare il tempo di attesa e garantire l'equità tra i processi.
    \item \textbf{Contesto Switching (Cambio di Contesto)}: L'overhead di tempo necessario per salvare lo stato di un processo in esecuzione e caricare lo stato del prossimo processo da eseguire. Questo tempo è "sprecato" e non contribuisce all'esecuzione del lavoro utile.
    \item \textbf{Starvation (Inedia)}: Un processo a bassa priorità potrebbe non essere mai eseguito se processi a priorità più alta arrivano continuamente e monopolizzano la CPU.
    \item \textbf{Deadlock}: Sebbene sia una problematica più ampia della gestione della concorrenza, situazioni di deadlock possono emergere in sistemi con scheduling se le risorse non sono gestite correttamente, bloccando indefinitamente i processi.
    \item \textbf{Dipendenza dall'I/O}: Processi che trascorrono molto tempo in attesa di operazioni di I/O (I/O-bound) possono rendere inefficiente lo scheduling se la CPU rimane inattiva mentre attende il completamento di tali operazioni.
\end{itemize}

\subsection{Esempi di Algoritmi di Scheduling}
Diversi algoritmi sono stati sviluppati per affrontare le problematiche dello scheduling, ognuno con i propri compromessi tra efficienza, equità e complessità di implementazione.

\subsubsection{First-Come, First-Served (FCFS)}
\begin{itemize}
    \item \textbf{Descrizione}: Non preemptive. I processi vengono eseguiti nell'ordine in cui arrivano nella coda dei processi pronti. Una volta che un processo ottiene la CPU, la tiene fino al completamento o fino a quando non esegue un'operazione di I/O.
    \item \textbf{Vantaggi}: Semplice da implementare e comprendere.
    \item \textbf{Svantaggi}:
    \begin{itemize}
        \item \textbf{"Effetto Convoglio"}: Un processo con un tempo di esecuzione molto lungo può bloccare tutti i processi successivi più brevi, aumentando notevolmente il tempo medio di attesa.
        \item Tempo di risposta e throughput possono essere scarsi in scenari sfavorevoli.
    \end{itemize}
    \item \textbf{Esempio}: Consideriamo i processi P1 (burst time 24), P2 (burst time 3), P3 (burst time 3) che arrivano nell'ordine P1, P2, P3.
    \begin{lstlisting}[numbers=none, language=Pseudocode]
Gantt Chart (FCFS):
| P1 (24) | P2 (3) | P3 (3) |
0        24       27       30
Tempo di attesa:
P1 = 0
P2 = 24
P3 = 27
Tempo medio di attesa = (0 + 24 + 27) / 3 = 17
    \end{lstlisting}
\end{itemize}

\subsubsection{Shortest-Job-First (SJF)}
\begin{itemize}
    \item \textbf{Descrizione}: Può essere preemptive (chiamato Shortest-Remaining-Time-First, SRTF) o non preemptive. Assegna la CPU al processo che ha il tempo di esecuzione stimato più breve.
    \item \textbf{Vantaggi}: Ottimale per minimizzare il tempo medio di attesa per un dato insieme di processi.
    \item \textbf{Svantaggi}:
    \begin{itemize}
        \item \textbf{Difficoltà di stima}: È difficile conoscere a priori la durata esatta del "burst time" di un processo (si usano stime basate sulla storia passata).
        \item \textbf{Starvation}: Processi lunghi potrebbero non essere mai eseguiti se arrivano continuamente processi più brevi.
    \end{itemize}
    \item \textbf{Esempio (Non Preemptive)}: Processi P1 (burst 7), P2 (burst 4), P3 (burst 1), P4 (burst 4). Arrivano quasi contemporaneamente.
    \begin{lstlisting}[numbers=none, language=Pseudocode]
Gantt Chart (SJF Non-Preemptive):
| P3 (1) | P2 (4) | P4 (4) | P1 (7) |
0        1        5        9        16
Tempo di attesa:
P3 = 0
P2 = 1
P4 = 5
P1 = 9
Tempo medio di attesa = (0 + 1 + 5 + 9) / 4 = 3.75
    \end{lstlisting}
\end{itemize}

\subsubsection{Priority Scheduling}
\begin{itemize}
    \item \textbf{Descrizione}: Può essere preemptive o non preemptive. Ad ogni processo viene assegnata una priorità (un numero intero, dove un numero più basso può indicare una priorità più alta, o viceversa). La CPU viene assegnata al processo con la priorità più alta.
    \item \textbf{Vantaggi}: Permette di prioritizzare lavori critici o importanti del sistema.
    \item \textbf{Svantaggi}:
    \begin{itemize}
        \item \textbf{Starvation}: Processi a bassa priorità potrebbero non essere mai eseguiti se processi a priorità più alta arrivano continuamente.
        \item \textbf{Soluzione per Starvation (Aging)}: La priorità di un processo che aspetta da troppo tempo viene gradualmente aumentata.
    \end{itemize}
    \item \textbf{Esempio}: Processi P1 (burst 10, priority 3), P2 (burst 1, priority 1), P3 (burst 2, priority 4), P4 (burst 1, priority 5), P5 (burst 5, priority 2). (Priorità più basse = priorità più alte).
    \begin{lstlisting}[numbers=none, language=Pseudocode]
Gantt Chart (Priority Scheduling, Non-Preemptive):
| P2 (1) | P5 (5) | P1 (10) | P3 (2) | P4 (1) |
0        1        6         16       18       19
Tempo di attesa (processi ordinati per arrivo, non priorita'):
P1 = 6
P2 = 0
P3 = 16
P4 = 18
P5 = 1
Tempo medio di attesa = (6 + 0 + 16 + 18 + 1) / 5 = 8.2
    \end{lstlisting}
\end{itemize}

\subsubsection{Round Robin (RR)}
\begin{itemize}
    \item \textbf{Descrizione}: Preemptive. Progettato per sistemi time-sharing. Ogni processo ottiene una piccola porzione di tempo di CPU, chiamata "quantum" (o time slice), solitamente da 10 a 100 millisecondi. Se un processo non finisce entro il quantum, viene preempted e messo in coda alla fine della coda dei processi pronti.
    \item \textbf{Vantaggi}:
    \begin{itemize}
        \item \textbf{Equità}: Garantisce che nessun processo aspetti per un tempo eccessivamente lungo.
        \item \textbf{Buon Tempo di Risposta}: Adatto per processi interattivi, dando l'impressione che tutti i processi siano eseguiti contemporaneamente.
    \end{itemize}
    \item \textbf{Svantaggi}:
    \begin{itemize}
        \item L'overhead del context switching aumenta se il quantum è troppo piccolo (troppi cambi di contesto).
        \item Le performance degradano se il quantum è troppo grande (il RR tende a comportarsi come FCFS).
        \item Può essere meno efficiente se i processi hanno durate molto diverse.
    \end{itemize}
    \item \textbf{Esempio}: Processi P1 (burst 24), P2 (burst 3), P3 (burst 3). Quantum = 4.
    \begin{lstlisting}[numbers=none, language=Pseudocode]
Gantt Chart (Round Robin, Quantum=4):
| P1 (4) | P2 (3) | P3 (3) | P1 (4) | P1 (4) | P1 (4) | P1 (4) | P1 (4) |
0        4        7        10       14       18       22       26       30
Tempo di attesa:
P1 = (4+3+3+3+3) = 16 (somma dei tempi non consecutivi)
P2 = 4
P3 = 7
Tempo medio di attesa = (16 + 4 + 7) / 3 = 9
    \end{lstlisting}
\end{itemize}

\subsubsection{Multilevel Queue Scheduling}
\begin{itemize}
    \item \textbf{Descrizione}: I processi sono divisi in diverse code, ognuna con il proprio algoritmo di scheduling. Le code possono avere priorità fisse tra di loro, oppure essere gestite con time slices diversi per ogni coda.
    \item \textbf{Esempio}: Coda foreground (processi interattivi) con Round Robin; Coda background (processi batch) con FCFS. I processi non si muovono tra le code.
\end{itemize}

\subsubsection{Multilevel Feedback Queue Scheduling}
\begin{itemize}
    \item \textbf{Descrizione}: Permette ai processi di muoversi tra le code in base al loro comportamento. Se un processo usa troppo la CPU, viene spostato in una coda con priorità inferiore o un quantum più grande. Se un processo aspetta molto, può essere spostato in una coda con priorità più alta.
    \item \textbf{Vantaggi}: Altamente configurabile, può approssimare SJF senza conoscere i burst time, può prevenire la starvation (tramite l'aging), e offrire una buona risposta per processi interattivi.
    \item \textbf{Svantaggi}: Molto complesso da implementare e ottimizzare, richiede molti parametri (numero di code, algoritmo per ogni coda, quando spostare i processi tra code).
\end{itemize}
% Include le domande e gli esercizi per questo capitolo, se il file esiste.
% Questo file (domande_ed_esercizi_so.tex) conterrà al suo interno
% le varie \section o \subsection per ogni singola domanda/esercizio.
% Questo file conterrà le domande e gli esercizi per il capitolo "Sistemi Operativi".
% Sarà incluso nel main.tex subito dopo il riassunto teorico del capitolo.

\section*{Domande e Esercizi} % Usiamo * per non numerare la sezione nell'indice
\addcontentsline{toc}{section}{Domande e Esercizi} % Aggiunge la voce all'indice manualmente

% --- Includi qui le singole domande e risposte da file separati ---
% Ogni domanda/risposta sarà una \subsection o \section nel proprio file.
% Questo file contiene la domanda e risposta sulla struttura del SO e scheduling.
% Sarà incluso da domande_ed_esercizi_so.tex.

\subsection*{Domanda: Struttura e Organizzazione del Sistema Operativo, e Scheduling della CPU} % Usiamo * per non numerare la domanda, per pulizia

\textbf{Domanda}: Il candidato fornisca una panoramica sulla struttura e organizzazione di un sistema operativo, descrivendo i principali componenti e modelli di sistemi operativi. Si approfondisca, inoltre, il tema dello scheduling della CPU, evidenziando le principali problematiche che questo comporta e illustrando esempi di algoritmi.

\paragraph{Risposta}:

Un \textbf{sistema operativo (SO)} è un software di sistema che gestisce le risorse hardware e software di un computer, fungendo da interfaccia tra l'hardware e l'utente/applicazioni.

I suoi componenti principali includono:
\begin{itemize}
    \item \textbf{Kernel}: Il cuore del SO, che gestisce processi, memoria, file system e I/O.
    \item \textbf{Gestore dei Processi}: Si occupa della creazione, terminazione e scheduling dei processi.
    \item \textbf{Gestore della Memoria}: Responsabile dell'allocazione, protezione e gestione della memoria virtuale.
    \item \textbf{File System Management}: Organizzazione dei dati su storage secondario.
    \item \textbf{Gestore I/O}: Interazione con i dispositivi hardware.
    \item \textbf{Network Management}: Gestisce le comunicazioni di rete.
    \textbf{Security and Protection}: Protezione delle risorse.
\end{itemize}
I modelli architettonici dei SO possono essere:
\begin{itemize}
    \item \textbf{Monolitici}
    \item \textbf{Layered} (a strati)
    \item \textbf{Microkernel}
    \item \textbf{Modulari} (Ibridi)
\end{itemize}
Ciascuno con vantaggi e svantaggi in termini di performance, flessibilità e robustezza.

Lo \textbf{scheduling della CPU} è l'attività di selezionare quale processo, tra quelli pronti per l'esecuzione, deve essere assegnato alla CPU in un dato momento. Ha un impatto cruciale sulle performance complessive del sistema. Le principali problematiche dello scheduling includono l'ottimizzazione di obiettivi contrastanti (throughput, tempo di risposta, tempo di attesa, equità), l'overhead del contesto switching, la starvation (processi a bassa priorità che non vengono eseguiti) e la gestione implicita di situazioni di deadlock.
Diversi algoritmi di scheduling sono utilizzati:
\begin{itemize}
    \item \textbf{First-Come, First-Served (FCFS)}: Non preemptive; esegue i processi in ordine di arrivo. Semplice, ma soffre l'effetto convoglio.
    \item \textbf{Shortest-Job-First (SJF)}: Può essere preemptive o non preemptive; esegue il processo con il burst time stimato più breve. Ottimale per tempo medio di attesa, ma difficile stimare la durata.
    \item \textbf{Priority Scheduling}: Assegna la CPU al processo con priorità più alta. Permette di prioritizzare lavori critici, ma può causare starvation (risolvibile con l'aging).
    \item \textbf{Round Robin (RR)}: Preemptive; ogni processo ottiene un quantum di tempo. Equo e con buon tempo di risposta, ma con overhead di contesto switching.
    \item \textbf{Multilevel Queue Scheduling}: I processi sono divisi in diverse code, ognuna con il proprio algoritmo.
    \item \textbf{Multilevel Feedback Queue Scheduling}: Permette ai processi di muoversi tra le code per adattarsi al loro comportamento, bilanciando efficienza e fairness.
\end{itemize}
% Questo file contiene la domanda e risposta su Processi, Thread e Sincronizzazione.
% Sarà incluso da domande_ed_esercizi_so.tex.

\subsection*{Domanda: Processi e Thread, Sincronizzazione e Deadlock}

\textbf{Domanda}: Il Candidato introduca i concetti di Processo "pesante" e "leggero" (Thread), illustrandone i punti in comune e le differenze. Si illustrino poi brevemente i possibili meccanismi di sincronizzazione tra processi al fine di evitare deadlock, quali ad esempio il meccanismo di lock su risorse, i “semafori", etc. Il Candidato illustri poi i concetti tramite un breve esempio a sua scelta.

\paragraph{Risposta}:

\textbf{Processi ("Pesanti") e Thread ("Leggeri")}
Nei sistemi operativi moderni, l'esecuzione dei programmi è gestita attraverso i processi e i thread, che rappresentano unità di esecuzione con diverse caratteristiche di gestione delle risorse.

Un \textbf{processo} è un'istanza di un programma in esecuzione, che possiede il proprio spazio di indirizzamento virtuale isolato e risorse dedicate come codice, dati, stack, Program Counter (PC) e registri CPU, oltre a file aperti e segnali. I processi sono considerati "pesanti" a causa dell'overhead significativo nella loro creazione, distruzione e nel cambio di contesto. Questo isolamento garantisce che il crash di un processo non influenzi direttamente gli altri.

Un \textbf{thread} (o thread di esecuzione) è un'unità di esecuzione più "leggera" all'interno di un processo. Più thread possono esistere all'interno dello stesso processo e condividono lo stesso spazio di indirizzamento virtuale, il codice del programma, i dati globali e le risorse del sistema operativo (es. file aperti). Ogni thread, tuttavia, ha il proprio Program Counter, il proprio set di registri della CPU e uno stack separato. La natura "leggera" dei thread si traduce in una creazione, distruzione e un cambio di contesto molto più rapidi ed efficienti rispetto ai processi.

\textbf{Punti in Comune}:
\begin{itemize}
    \item Sia i processi che i thread sono unità di esecuzione che possono essere schedulate dalla CPU.
    \item Entrambi mantengono un Program Counter, un set di registri e uno stack (anche se lo stack è separato per ogni thread).
\end{itemize}
\textbf{Differenze Principali}:
\begin{itemize}
    \item \textbf{Isolamento delle Risorse}: I processi operano in spazi di indirizzamento separati e con risorse isolate, fornendo una maggiore protezione. I thread all'interno dello stesso processo condividono lo spazio di indirizzamento e le risorse, rendendo la comunicazione più semplice ma con meno isolamento.
    \item \textbf{Overhead}: I processi comportano un overhead maggiore per la gestione del loro ciclo di vita e il cambio di contesto. I thread sono più efficienti in queste operazioni.
    \item \textbf{Comunicazione}: La comunicazione tra processi (IPC) è più complessa e richiede meccanismi specifici. La comunicazione tra thread (ITC) è più diretta grazie alla memoria condivisa.
    \item \textbf{Robustezza}: Un errore critico in un thread può causare il crash dell'intero processo e di tutti i suoi thread. Un processo in crash solitamente non influisce sugli altri processi del sistema.
\end{itemize}

\paragraph{Meccanismi di Sincronizzazione e Deadlock}
Nei sistemi multi-programmati o multi-thread, la \textbf{sincronizzazione} è cruciale per prevenire \textbf{Race Condition} (risultati imprevedibili dovuti all'accesso concorrente a dati condivisi) e garantire la coerenza dei dati. La \textbf{sezione critica} è la porzione di codice in cui si accede a risorse condivise, e la mutua esclusione è l'obiettivo primario per garantire che un solo processo/thread vi acceda per volta.

I \textbf{meccanismi di sincronizzazione} includono:
\begin{itemize}
    \item \textbf{Lock}: Strumenti che permettono di acquisire un blocco su una risorsa prima di accedervi e rilasciarlo al termine, forzando l'attesa di altri processi.
    \item \textbf{Semafori}: Variabili intere gestite da operazioni atomiche `wait()` (decrementa e blocca se negativo) e `signal()` (incrementa e sblocca se ci sono processi in attesa). Esistono semafori binari (mutex) per la mutua esclusione e semafori contatori per risorse multiple.
        \textbf{Esempio (Produttore-Consumatore con Semafori)}: Un problema classico in cui un produttore aggiunge elementi a un buffer e un consumatore li preleva, utilizzando semafori (`empty` per slot vuoti, `full` per slot pieni, `mutex` per l'accesso al buffer) per coordinare l'accesso e prevenire la sovrascrittura/lettura di dati non validi.
        \begin{lstlisting}[language=Pseudocode, numbers=none]
// Semafori: empty (N), full (0), mutex (1)

Producer:
  LOOP
    produce item
    CALL wait(empty)
    CALL wait(mutex)
    add item to buffer
    CALL signal(mutex)
    CALL signal(full)
  END LOOP

Consumer:
  LOOP
    CALL wait(full)
    CALL wait(mutex)
    remove item from buffer
    CALL signal(mutex)
    CALL signal(empty)
    consume item
  END LOOP
        \end{lstlisting}
    \item \textbf{Monitor}: Costrutti di alto livello che incapsulano dati condivisi e le procedure che li manipolano, garantendo l'accesso esclusivo tramite variabili di condizione (`wait()` e `signal()`) per la sospensione/attivazione dei processi.
\end{itemize}

Il \textbf{deadlock} (interblocco) si verifica quando due o più processi sono bloccati indefinitamente, ciascuno in attesa di una risorsa detenuta da un altro processo bloccato. Le \textbf{quattro condizioni necessarie} per il deadlock sono:
\begin{enumerate}
    \item \textbf{Mutua Esclusione}: Risorse non condivisibili.
    \item \textbf{Attesa e Mantenimento}: Un processo detiene risorse e ne attende altre.
    \item \textbf{Non-Preemptive}: Le risorse non possono essere sottratte forzatamente.
    \item \textbf{Attesa Circolare}: Esiste una catena circolare di dipendenze tra processi.
\end{enumerate}
Le \textbf{strategie di gestione del deadlock} includono:
\begin{itemize}
    \item \textbf{Prevenzione}: Negare una o più delle condizioni necessarie (es. richiedere tutte le risorse all'inizio).
    \item \textbf{Evitamento}: Utilizzare informazioni a priori per decidere se uno stato è "sicuro" prima di allocare risorse (es. Algoritmo del Banchiere).
    \item \textbf{Rilevamento e Ripristino}: Permettere il deadlock, rilevarlo tramite algoritmi di ricerca di cicli e poi ripristinare il sistema (es. terminando processi o preemption di risorse).
    \item \textbf{Ignorare il Problema}: Assumere che il deadlock sia raro e gestirlo manualmente (es. riavvio del sistema).
\end{itemize}
% Questo file contiene la domanda e risposta sul Record di Attivazione.
% Sarà incluso da domande_ed_esercizi_so.tex.

\subsection*{Domanda: Record di Attivazione}

\textbf{Domanda}: Il Candidato illustri il concetto di record di attivazione, descriva cosa esso contenga ed il suo ciclo di vita. Presenti inoltre un esempio di record di attivazione per il caso di una funzione "pow2" che riceve in input un solo parametro "x" e lo ritorna elevato al quadrato.

\paragraph{Risposta}:

\textbf{Concetto di Record di Attivazione}
Il \textbf{Record di Attivazione} (o \textit{Stack Frame}) è una struttura dati fondamentale creata sullo stack di esecuzione di un processo ogni volta che una funzione (o procedura, o subroutine) viene chiamata. La sua funzione principale è contenere tutte le informazioni necessarie per la gestione dell'esecuzione di quella specifica chiamata di funzione, fornendo il contesto per il suo funzionamento e il suo corretto ritorno al punto di chiamata.

\paragraph{Contenuto di un Record di Attivazione}
Un record di attivazione è una struttura complessa la cui esatta composizione può variare leggermente a seconda dell'architettura della CPU, del sistema operativo e del compilatore, ma tipicamente include:
\begin{itemize}
    \item \textbf{Parametri Attuali}: I valori degli argomenti che vengono passati alla funzione durante la sua chiamata.
    \item \textbf{Indirizzo di Ritorno}: L'indirizzo di memoria dell'istruzione nel codice della funzione chiamante a cui il controllo del programma deve essere trasferito una volta che la funzione corrente ha terminato la sua esecuzione.
    \item \textbf{Valore di Ritorno}: Uno spazio riservato per memorizzare il risultato (se la funzione restituisce un valore) che verrà poi recuperato dalla funzione chiamante.
    \item \textbf{Variabili Locali}: Le variabili dichiarate all'interno del corpo della funzione. Queste variabili hanno scope e durata limitati all'attivazione corrente della funzione.
    \item \textbf{Stato dei Registri Salvati}: I valori dei registri della CPU che erano in uso dalla funzione chiamante e che vengono salvati per essere ripristinati al ritorno, garantendo che lo stato della CPU non sia corrotto dalla funzione chiamata.
    \item \textbf{Puntatore al Frame Precedente (Control Link / Dynamic Link)}: Un puntatore all'indirizzo del record di attivazione della funzione che ha effettuato la chiamata. Questo permette di "risalire" lo stack e ripristinare il contesto della funzione chiamante.
    \item \textbf{Puntatore al Contesto Statico (Access Link / Static Link)}: Un puntatore al record di attivazione della funzione che definisce lo scope lessicale della funzione corrente (rilevante in linguaggi con scope annidato statico, come Pascal, per accedere a variabili non locali).
\end{itemize}

\paragraph{Ciclo di Vita di un Record di Attivazione}
Il ciclo di vita di un record di attivazione è strettamente legato alla dinamica delle chiamate e dei ritorni delle funzioni, seguendo una logica LIFO (Last-In, First-Out) tipica delle strutture a stack:
\begin{enumerate}
    \item \textbf{Creazione (Chiamata di Funzione)}: Quando un processo chiama una funzione, un nuovo record di attivazione viene creato (popolato con le informazioni rilevanti) e "pushed" (inserito) in cima allo stack di esecuzione del processo. Il puntatore dello stack viene aggiornato per puntare a questo nuovo frame. Il controllo viene poi passato all'inizio della funzione chiamata.
    \item \textbf{Esecuzione}: La funzione utilizza i dati e le risorse definite all'interno del suo record di attivazione (parametri, variabili locali) per eseguire le sue operazioni. Durante l'esecuzione, può a sua volta chiamare altre funzioni, che a loro volta creeranno nuovi record di attivazione sul top dello stack.
    \item \textbf{Terminazione (Ritorno da Funzione)}: Una volta che la funzione completa la sua esecuzione (sia raggiungendo un'istruzione `return` che la fine del suo blocco di codice), il valore di ritorno (se presente) viene posizionato in un registro designato. Il record di attivazione corrente viene quindi "popped" (rimosso) dalla cima dello stack, liberando lo spazio di memoria che occupava.
    \item \textbf{Ripristino del Contesto}: Il sistema operativo (o il runtime) utilizza l'indirizzo di ritorno salvato nel record appena rimosso per trasferire il controllo alla funzione chiamante, e ripristina lo stato dei registri della CPU per consentire alla funzione chiamante di riprendere l'esecuzione dal punto in cui era stata interrotta.
\end{enumerate}

\paragraph{Esempio di Record di Attivazione per la funzione "pow2(x)"}
Consideriamo una semplice funzione `pow2(x)` che calcola il quadrato del suo input `x`.

\begin{lstlisting}[language=Pseudocode, numbers=none, caption={Funzione pow2(x)}]
FUNCTION pow2(x):
    RETURN x * x
END FUNCTION
\end{lstlisting}

Quando la funzione `pow2(5)` viene chiamata, viene creato il seguente record di attivazione (semplificato) sullo stack:

\begin{itemize}
    \item \textbf{Parametri Attuali}:
    \begin{itemize}
        \item `x`: 5
    \end{itemize}
    \item \textbf{Indirizzo di Ritorno}: L'indirizzo nel codice della funzione chiamante da cui `pow2(5)` è stata invocata (es. `0x00A0` nel chiamante).
    \item \textbf{Valore di Ritorno}: Spazio per il risultato (es. 25).
    \item \textbf{Variabili Locali}: Nessuna in questo esempio specifico (o temporanee usate dal compilatore).
    \item \textbf{Stato dei Registri Salvati}: Valori dei registri che devono essere preservati per la funzione chiamante.
    \item \textbf{Puntatore al Frame Precedente}: Indirizzo del record di attivazione della funzione chiamante.
\end{itemize}

\textbf{Ciclo di Vita Semplificato per `pow2(5)`}:
\begin{enumerate}
    \item \textbf{Chiamata}: La funzione chiamante spinge i parametri e l'indirizzo di ritorno sullo stack. Viene creato e spinto il record di attivazione per `pow2(5)`.
    \item \textbf{Esecuzione}: `pow2` prende il valore di `x` (5), calcola `5 * 5 = 25`.
    \item \textbf{Ritorno}: Il valore 25 viene posizionato dove la funzione chiamante lo recupererà. Il record di attivazione di `pow2` viene rimosso dallo stack.
    \item \textbf{Ripristino}: Il controllo torna all'indirizzo di ritorno nella funzione chiamante, che continua la sua esecuzione.
\end{enumerate}
Questo illustra come il record di attivazione gestisce il contesto di una singola invocazione di funzione, permettendo l'esecuzione modulare dei programmi.

\subsection*{Altre Possibili Domande}
\addcontentsline{toc}{subsection}{Altre Possibili Domande} % Aggiunge la voce all'indice

% --- Includi qui le domande e risposte aggiuntive, basate su concetti chiave ---
% Questo file contiene la domanda e risposta sulla memoria virtuale.
% Sarà incluso da domande_ed_esercizi_so.tex.

\subsection*{Domanda: Come la Memoria Virtuale Migliora la Gestione della Memoria nei Sistemi Operativi?}

\textbf{Domanda}: Come la memoria virtuale migliora la gestione della memoria nei sistemi operativi?

\textbf{Risposta}:

La \textbf{memoria virtuale} è una tecnica del sistema operativo che permette di usare lo spazio su disco (memoria secondaria) per simulare una RAM maggiore, consentendo ai programmi di utilizzare più memoria di quella fisica disponibile. Questo facilita il multitasking isolando lo spazio di indirizzamento di ogni processo, aumentandone la protezione e la flessibilità. Le tecniche principali per implementare la memoria virtuale sono la paginazione e la segmentazione.
\begin{itemize}
    \item \textbf{Paginazione}: Divide lo spazio logico in "pagine" di dimensione fissa e la memoria fisica in "frame", mappando le pagine ai frame tramite una Page Table. Questo elimina la frammentazione esterna ma introduce quella interna.
    \item \textbf{Segmentazione}: Vede la memoria come segmenti di dimensione variabile, corrispondenti a unità logiche del programma, facilitando protezione e condivisione ma soffrendo di frammentazione esterna.
\end{itemize}
Per l'allocazione della memoria (per blocchi contigui), il sistema operativo utilizza algoritmi come \textbf{First Fit} (alloca il primo blocco sufficiente), \textbf{Best Fit} (alloca il blocco più piccolo sufficiente) e \textbf{Worst Fit} (alloca il blocco più grande sufficiente), ognuno con diversi compromessi in termini di velocità e frammentazione.
% Questo file contiene la domanda e risposta sui problemi di sincronizzazione e gestione deadlock.
% Sarà incluso da domande_ed_esercizi_so.tex.

\subsection*{Domanda: Quali sono i Problemi Principali Legati alla Sincronizzazione dei Processi e Come Vengono Gestiti i Deadlock?}

\textbf{Domanda}: Quali sono i problemi principali legati alla sincronizzazione dei processi e come vengono gestiti i deadlock?

\textbf{Risposta}:

Nei sistemi multi-programmati o multi-thread, la \textbf{sincronizzazione} è cruciale per evitare "\textbf{Race Condition}", dove il risultato delle operazioni su dati condivisi dipende dall'ordine di esecuzione, portando a risultati imprevedibili. La "\textbf{Sezione Critica}" è una porzione di codice in cui un processo accede a risorse condivise, e l'obiettivo è garantire la "\textbf{Mutua Esclusione}" (solo un processo alla volta nella sezione critica), il "\textbf{Progresso}" (decisione non ritardata indefinitamente) e l'"\textbf{Attesa Limitata}" (nessuna starvation).
\textbf{Meccanismi di sincronizzazione} includono:
\begin{itemize}
    \item \textbf{Lock}: Meccanismi semplici per acquisire e rilasciare un blocco su una risorsa.
    \item \textbf{Semafori}: Variabili intere gestite da operazioni atomiche `wait()` e `signal()`, usate per mutua esclusione (semafori binari/mutex) o controllo di risorse (semafori contatori).
    \item \textbf{Monitor}: Costrutti di alto livello che incapsulano dati condivisi e procedure, garantendo l'accesso esclusivo tramite variabili di condizione.
\end{itemize}
Il "\textbf{Deadlock}" (interblocco) si verifica quando due o più processi sono bloccati indefinitamente in attesa di risorse detenute da altri processi bloccati. Le \textbf{quattro condizioni necessarie} per il deadlock sono:
\begin{enumerate}
    \item \textbf{Mutua Esclusione} (risorse non condivisibili).
    \item \textbf{Attesa e Mantenimento} (un processo detiene risorse e ne attende altre).
    \item \textbf{Non-Preemption} (le risorse non possono essere sottratte forzatamente).
    \item \textbf{Attesa Circolare} (catena circolare di dipendenze).
\end{enumerate}
Le \textbf{strategie per gestire il deadlock} includono:
\begin{itemize}
    \item \textbf{Prevenzione}: Negare una o più delle condizioni necessarie (es. richiedere tutte le risorse all'inizio).
    \item \textbf{Evitamento}: Utilizzare informazioni a priori per decidere se uno stato è "sicuro" prima di allocare risorse (es. Algoritmo del Banchiere).
    \item \textbf{Rilevamento e Ripristino}: Permettere il deadlock, rilevarlo tramite algoritmi e poi ripristinare il sistema (es. terminando processi o sottraendo risorse).
    \item \textbf{Ignorare il Problema}: Assumere che il deadlock sia raro e gestirlo manualmente (es. riavvio del sistema).
\end{itemize}
% Questo file contiene la domanda e risposta sul funzionamento e algoritmi di scheduling.
% Sarà incluso da domande_ed_esercizi_so.tex.

\subsection*{Domanda: Come Funziona lo Scheduling della CPU e Quali Sono gli Algoritmi Più Comuni?}

\textbf{Domanda}: Come funziona lo scheduling della CPU e quali sono gli algoritmi più comuni?

\textbf{Risposta}:

Lo \textbf{scheduling della CPU} è il processo di selezione del prossimo processo da assegnare alla CPU tra quelli pronti, con l'obiettivo di ottimizzare throughput, tempo di risposta ed equità. Le problematiche includono l'overhead del context switching, la starvation (processi a bassa priorità mai eseguiti) e la dipendenza dall'I/O. Esempi di algoritmi di scheduling includono:
\begin{itemize}
    \item \textbf{First-Come, First-Served (FCFS)}: Non-preemptive, i processi vengono eseguiti nell'ordine di arrivo. Semplice ma può soffrire dell'"Effetto Convoglio".
    \item \textbf{Shortest-Job-First (SJF)}: Assegna la CPU al processo con il tempo di esecuzione stimato più breve (può essere preemptive come SRTF). Ottimale per minimizzare il tempo medio di attesa ma difficile da stimare e può causare starvation.
    \item \textbf{Priority Scheduling}: Assegna la CPU al processo con priorità più alta. Vantaggioso per lavori critici ma soggetto a starvation, mitigabile con l'Aging.
    \item \textbf{Round Robin (RR)}: Preemptive, ogni processo ottiene un piccolo "quantum" di CPU. Garantisce equità e buon tempo di risposta per processi interattivi, ma l'overhead del context switching aumenta con quantum piccoli.
    \item \textbf{Multilevel Queue Scheduling} e \textbf{Multilevel Feedback Queue Scheduling}: Dividono i processi in code con algoritmi e priorità diverse, con l'ultimo che permette ai processi di spostarsi tra le code per prevenire starvation e ottimizzare la risposta.
\end{itemize}

% --- Inserirai qui i futuri esercizi per Sistemi Operativi da file separati ---
% \subsection*{Esercizi}
% \addcontentsline{toc}{subsection}{Esercizi}
% \input{capitoli/sistemi_operativi/esercizi/so_esercizio_sincronizzazione.tex}

% Capitolo 2: Basi di Dati
\chapter{Basi di Dati}

Le \textbf{Basi di Dati (Database)} sono collezioni organizzate di dati che permettono un'efficiente memorizzazione, recupero e gestione delle informazioni. Sono fondamentali per la maggior parte delle applicazioni software moderne.

\section{Modello Concettuale Entità-Relazione (ER)}
Il \textbf{Modello Entità-Relazione (ER)} è uno strumento concettuale di alto livello utilizzato nella fase iniziale della progettazione di database. Permette di rappresentare il mondo reale in termini di "entità" (oggetti o concetti di interesse) e "relazioni" (associazioni tra le entità). L'obiettivo è fornire una rappresentazione intuitiva e facilmente comprensibile della struttura dei dati prima di tradurla in un modello logico.

\subsection{Componenti Principali del Modello ER}
\begin{itemize}
    \item \textbf{Entità}: Rappresentano "cose" o "oggetti" del mondo reale su cui si vogliono memorizzare informazioni. Possono essere concrete (es. Persona, Prodotto) o astratte (es. Corso, Ordine). Nel diagramma ER, le entità sono generalmente rappresentate con un rettangolo.
    \item \textbf{Attributi}: Sono le proprietà o caratteristiche che descrivono un'entità o una relazione. Ad esempio, un'entità "Studente" può avere attributi come "Nome", "Cognome", "Matricola", "DataNascita". Nel diagramma ER, gli attributi sono spesso rappresentati con un ovale.
    \begin{itemize}
        \item \textbf{Attributi Semplici/Composti}: Un attributo semplice non può essere scomposto (es. "Età"), mentre uno composto è formato da più attributi (es. "Indirizzo" composto da "Via", "Civico", "Città").
        \item \textbf{Attributi Mono-valore/Multi-valore}: Mono-valore ha un singolo valore per istanza (es. "DataNascita"), multi-valore può avere più valori (es. "NumeriDiTelefono").
        \item \textbf{Attributi Derivati}: Il loro valore può essere calcolato da altri attributi (es. "Età" derivata da "DataNascita").
        \item \textbf{Chiave (Key Attribute)}: Un attributo (o un insieme di attributi) che identifica in modo univoco ogni istanza di un'entità. Viene tipicamente sottolineato nel diagramma ER.
    \end{itemize}
    \item \textbf{Relazioni}: Rappresentano associazioni logiche tra due o più entità. Ad esempio, un "Docente" "insegna" a un "Corso". Nel diagramma ER, le relazioni sono rappresentate con un rombo.
    \begin{itemize}
        \item \textbf{Cardinalità delle Relazioni}: Definisce il numero di istanze di un'entità che possono essere associate a un'istanza dell'altra entità nella relazione. Le cardinalità più comuni sono:
        \begin{itemize}
            \item \textbf{Uno a Uno (1:1)}: Una istanza di entità A è associata a una e una sola istanza di entità B, e viceversa (es. "Persona" - "ha" - "Patente").
            \item \textbf{Uno a Molti (1:N)}: Una istanza di entità A è associata a zero o molte istanze di entità B, ma un'istanza di B è associata a una e una sola istanza di A (es. "Dipartimento" - "comprende" - "Docente").
            \item \textbf{Molti a Molti (N:M)}: Una istanza di entità A è associata a zero o molte istanze di entità B, e viceversa (es. "Studente" - "frequenta" - "Corso").
        \end{itemize}
        \item \textbf{Partecipazione (o Dipendenza)}: Indica se l'esistenza di un'istanza di un'entità dipende dalla sua partecipazione a una relazione.
        \begin{itemize}
            \item \textbf{Totale (o Obbligatoria)}: Ogni istanza dell'entità deve partecipare alla relazione (indicata da una doppia linea).
            \item \textbf{Parziale (o Opzionale)}: Un'istanza dell'entità può partecipare o meno alla relazione (indicata da una singola linea).
        \end{itemize}
    \end{itemize}
\end{itemize}

\subsection{Rappresentazione Grafica del Diagramma ER}
La rappresentazione visuale del modello Entità-Relazione utilizza simboli standardizzati per facilitare la comprensione della struttura del database.
\begin{itemize}
    \item \textbf{Entità}: Rettangolo. Se è un'entità debole, il rettangolo è doppio.
    \item \textbf{Attributi}: Ovale.
    \begin{itemize}
        \item \textbf{Chiave Primaria}: Ovale con il nome dell'attributo sottolineato.
        \item \textbf{Attributo Composto}: Ovale collegato ad altri ovali più piccoli.
        \item \textbf{Attributo Multi-valore}: Doppio ovale.
        \item \textbf{Attributo Derivato}: Ovale tratteggiato.
    \end{itemize}
    \item \textbf{Relazioni}: Rombo. Se è una relazione identificativa (per entità deboli), il rombo è doppio.
    \item \textbf{Connessioni}: Linee che collegano entità e relazioni.
    \item \textbf{Cardinalità e Partecipazione}: Le cardinalità sono indicate con numeri o simboli sulle linee di connessione (es. 1, N, M). La partecipazione (minima, massima) è indicata con notazioni come $(min, max)$ o da linee:
    \begin{itemize}
        \item \textbf{Linea singola}: Partecipazione parziale (0 o 1).
        \item \textbf{Doppia linea}: Partecipazione totale (almeno 1).
        \item \textbf{Notazione (min, max)}: Es. $(0, N)$ per zero a molti, $(1, N)$ per uno a molti.
    \end{itemize}
\end{itemize}
\begin{figure}[h!]
    \centering
    % Inserirai qui l'immagine di un Esempio di Diagramma ER
    % \includegraphics[width=0.8\textwidth]{immagini/diagramma_er_esempio.png}
    \caption{Esempio di Diagramma Entità-Relazione che mostra entità, attributi e relazioni con cardinalità e partecipazione.}
    \label{fig:diagramma_er_uml}
\end{figure}

\section{Progettazione Logica: Normalizzazione e Forme Normali}
La \textbf{normalizzazione} è un processo sistematico di organizzazione dei dati in un database relazionale. Il suo scopo è ridurre la ridondanza dei dati, eliminare le anomalie di aggiornamento (inserimento, cancellazione, modifica) e migliorare l'integrità e la coerenza dei dati. La normalizzazione si basa su una serie di regole chiamate "forme normali".

\subsection{Obiettivi della Normalizzazione}
\begin{itemize}
    \item \textbf{Riduzione della Ridondanza}: Evitare la duplicazione inutile dei dati, che spreca spazio e può portare a incongruenze.
    \item \textbf{Miglioramento dell'Integrità dei Dati}: Assicurare che i dati siano accurati e consistenti.
    \item \textbf{Prevenzione delle Anomalie}:
    \begin{itemize}
        \item \textbf{Anomalia di Inserimento}: Impossibilità di inserire un'informazione a meno che non si inseriscano anche altre informazioni non correlate.
        \item \textbf{Anomalia di Cancellazione}: La cancellazione di un dato comporta la perdita accidentale di altre informazioni non desiderate.
        \item \textbf{Anomalia di Aggiornamento}: La modifica di un dato ripetuto richiede l'aggiornamento di più occorrenze, con rischio di inconsistenza se non tutte vengono aggiornate.
    \end{itemize}
    \item \textbf{Flessibilità e Manutenibilità}: Rendere il database più facile da modificare ed estendere.
\end{itemize}

\subsection{Principali Forme Normali}
Le forme normali sono una serie di regole progressive; per essere in una forma normale N, una relazione deve soddisfare i requisiti della forma normale N-1. Le più comuni e rilevanti per la maggior parte delle applicazioni sono la Prima, Seconda e Terza Forma Normale.

\subsubsection{Prima Forma Normale (1NF)}
Una relazione è in 1NF se e solo se:
\begin{itemize}
    \item Tutti gli attributi sono \textbf{atomici} (indivisibili). Non ci sono attributi con valori multipli o attributi composti che non sono stati scomposti.
    \item Ogni record (riga) nella relazione è \textbf{unico}. Questo implica che deve esistere una chiave primaria.
\end{itemize}
\textbf{Esempio di Violazione}: Una colonna "NumeriDiTelefono" che contiene più numeri per un'unica riga.

\subsubsection{Seconda Forma Normale (2NF)}
Una relazione è in 2NF se e solo se:
\begin{itemize}
    \item È in \textbf{1NF}.
    \item Tutti gli attributi non-chiave dipendono \textbf{completamente} dalla chiave primaria. Non ci sono dipendenze parziali, il che significa che nessun attributo non-chiave dipende solo da una parte di una chiave primaria composta.
\end{itemize}
\textbf{Esempio di Violazione}: In una tabella `(IDCorso, IDStudente, NomeCorso, Voto)`, se `(IDCorso, IDStudente)` è la chiave primaria, e `NomeCorso` dipende solo da `IDCorso` (e non da `IDStudente`), allora `NomeCorso` è parzialmente dipendente e viola la 2NF.

\subsubsection{Terza Forma Normale (3NF)}
Una relazione è in 3NF se e solo se:
\begin{itemize}
    \item È in \textbf{2NF}.
    \item Non contiene \textbf{dipendenze transitive}. Ovvero, nessun attributo non-chiave dipende da un altro attributo non-chiave (anziché dipendere direttamente dalla chiave primaria).
\end{itemize}
\textbf{Esempio di Violazione}: In una tabella `(IDImpiegato, NomeImpiegato, Dipartimento, CapoDipartimento)`, se `IDImpiegato` è la chiave primaria e `CapoDipartimento` dipende da `Dipartimento` (che a sua volta dipende da `IDImpiegato`), si ha una dipendenza transitiva.

\subsection{Linguaggio SQL}
Il \textbf{Structured Query Language (SQL)} è il linguaggio standard per la gestione dei sistemi di gestione di database relazionali (RDBMS). Permette di definire, manipolare e controllare i dati.

\subsubsection{Categorie di Comandi SQL}
\begin{itemize}
    \item \textbf{Data Definition Language (DDL)}: Utilizzato per definire e modificare la struttura del database.
    \begin{itemize}
        \item \textbf{CREATE}: Crea database, tabelle, viste, indici, ecc. (es. `CREATE TABLE Studenti (...)`).
        \item \textbf{ALTER}: Modifica la struttura di oggetti esistenti (es. `ALTER TABLE Studenti ADD COLUMN Età INT`).
        \item \textbf{DROP}: Cancella oggetti dal database (es. `DROP TABLE Studenti`).
    \end{itemize}
    \item \textbf{Data Manipulation Language (DML)}: Utilizzato per manipolare i dati all'interno delle tabelle.
    \begin{itemize}
        \item \textbf{SELECT}: Recupera dati da una o più tabelle. È la query più usata.
        \item \textbf{INSERT}: Aggiunge nuove righe a una tabella.
        \item \textbf{UPDATE}: Modifica righe esistenti in una tabella.
        \item \textbf{DELETE}: Rimuove righe da una tabella.
    \end{itemize}
    \item \textbf{Data Control Language (DCL)}: Utilizzato per gestire i permessi di accesso ai dati.
    \begin{itemize}
        \item \textbf{GRANT}: Concede privilegi agli utenti.
        \item \textbf{REVOKE}: Rimuove privilegi dagli utenti.
    \end{itemize}
    \item \textbf{Transaction Control Language (TCL)}: Utilizzato per gestire le transazioni (gruppi di operazioni che devono essere eseguite atomicamente).
    \begin{itemize}
        \item \textbf{COMMIT}: Salva le modifiche di una transazione.
        \item \textbf{ROLLBACK}: Annulla le modifiche di una transazione.
    \end{itemize}
\end{itemize}

\subsubsection{Elementi Comuni delle Query SQL (SELECT)}
La query \texttt{SELECT} è la più potente e versatile, permettendo di interrogare il database.
\begin{itemize}
    \item \textbf{SELECT}: Specifica le colonne da recuperare.
    \begin{itemize}
        \item \texttt{SELECT colonna1, colonna2}
        \item \texttt{SELECT *} (tutte le colonne)
        \item \texttt{SELECT DISTINCT colonna} (solo valori unici)
    \end{itemize}
    \item \textbf{FROM}: Specifica la tabella (o le tabelle) da cui recuperare i dati.
    \item \textbf{WHERE}: Filtra le righe in base a una condizione specificata.
    \begin{itemize}
        \item \texttt{WHERE condizione} (es. `WHERE Età > 18`)
        \item Operatori: `=`, `>`, `<`, `>=`, `<=`, `<>`, `LIKE` (per pattern matching), `IN`, `BETWEEN`, `IS NULL`.
    \end{itemize}
    \item \textbf{JOIN}: Combina righe da due o più tabelle basandosi su una colonna correlata.
    \begin{itemize}
        \item \textbf{INNER JOIN}: Restituisce solo le righe che hanno corrispondenze in entrambe le tabelle.
        \item \textbf{LEFT (OUTER) JOIN}: Restituisce tutte le righe dalla tabella sinistra e le righe corrispondenti dalla tabella destra (con NULL se non ci sono corrispondenze).
        \item \textbf{RIGHT (OUTER) JOIN}: Simile al LEFT JOIN, ma per la tabella destra.
        \item \textbf{FULL (OUTER) JOIN}: Restituisce tutte le righe quando c'è una corrispondenza in una delle due tabelle.
    \end{itemize}
    \item \textbf{GROUP BY}: Raggruppa le righe che hanno gli stessi valori in una o più colonne, spesso usato con funzioni di aggregazione.
    \item \textbf{HAVING}: Filtra i gruppi creati da `GROUP BY` in base a una condizione. Si usa con le funzioni di aggregazione.
    \item \textbf{ORDER BY}: Ordina il set di risultati in base a una o più colonne (ASC per ascendente, DESC per discendente).
    \item \textbf{Funzioni di Aggregazione}: Calcolano un singolo valore da un insieme di valori (es. `COUNT()`, `SUM()`, `AVG()`, `MAX()`, `MIN()`).
\end{itemize}

\subsubsection{Esempi di Operatori e Funzioni SQL Comuni}
Oltre agli elementi base delle query \lstinline{SELECT}, SQL offre un'ampia gamma di operatori e funzioni per manipolare e filtrare i dati in modo più complesso.
\begin{itemize}
    \item \textbf{Operatori Logici}:
    \begin{itemize}
        \item \lstinline{AND}: Combina due condizioni, entrambe devono essere vere.
        \item \lstinline{OR}: Combina due condizioni, almeno una deve essere vera.
        \item \lstinline{NOT}: Nega una condizione.
    \end{itemize}
    \begin{lstlisting}[language=SQL, caption={Esempio Operatori Logici}]
SELECT FirstName, LastName
FROM Students
WHERE Age > 20 AND City = 'Bologna';
    \end{lstlisting}
    \item \textbf{Operatori di Confronto}:
    \begin{itemize}
        \item \lstinline{=}: Uguale a.
        \item \lstinline{<>} o \lstinline{!=}: Diverso da.
        \item \lstinline{<}, \lstinline{>}, \lstinline{<=}, \lstinline{>=}: Minore, maggiore, minore o uguale, maggiore o uguale.
        \item \lstinline{BETWEEN min AND max}: Valore compreso in un intervallo (inclusi gli estremi).
        \item \lstinline{LIKE pattern}: Ricerca stringhe che corrispondono a un pattern (es. \lstinline{LIKE 'A%'}).
        \item \lstinline{IN (value1, value2, ...)}: Valore presente in una lista di valori.
        \item \lstinline{IS NULL} / \lstinline{IS NOT NULL}: Verifica se un valore è NULL.
    \end{itemize}
    \begin{lstlisting}[language=SQL, caption={Esempio Operatori di Confronto}]
SELECT ProductName, Price
FROM Products
WHERE Price BETWEEN 10.00 AND 50.00
  AND ProductName LIKE 'Book%';

SELECT OrderID
FROM Orders
WHERE DeliveryDate IS NULL;
    \end{lstlisting}
    \item \textbf{Funzioni Stringa}:
    \begin{itemize}
        \item \lstinline{CONCAT(s1, s2, ...)}: Concatena stringhe.
        \item \lstinline{SUBSTRING(string, start, length)}: Estrae una sottostringa.
        \item \lstinline{LENGTH(string)}: Restituisce la lunghezza di una stringa.
        \item \lstinline{UPPER(string)} / \lstinline{LOWER(string)}: Converte in maiuscolo/minuscolo.
    \end{itemize}
    \begin{lstlisting}[language=SQL, caption={Esempio Funzioni Stringa}]
SELECT CONCAT(FirstName, ' ', LastName) AS FullName
FROM Users
WHERE LENGTH(FirstName) > 5;

SELECT UPPER(CategoryName)
FROM Categories;
    \end{lstlisting}
    \item \textbf{Funzioni Numeriche}:
    \begin{itemize}
        \item \lstinline{ROUND(number, decimal_places)}: Arrotonda un numero.
        \item \lstinline{ABS(number)}: Valore assoluto.
    \end{itemize}
    \begin{lstlisting}[language=SQL, caption={Esempio Funzioni Numeriche}]
SELECT ROUND(UnitPrice * Quantity, 2) AS RoundedTotal
FROM OrderDetails;

SELECT ABS(Balance)
FROM BankAccounts;
    \end{lstlisting}
    \item \textbf{Funzioni Data/Ora}:
    \begin{itemize}
        \item \lstinline{NOW()}: Data e ora correnti.
        \item \lstinline{CURDATE()}: Data corrente.
        \item \lstinline{DATE_ADD(date, INTERVAL value unit)}: Aggiunge un intervallo a una data.
    \end{itemize}
    \begin{lstlisting}[language=SQL, caption={Esempio Funzioni Data/Ora}]
SELECT EventName, EventDate
FROM Events
WHERE EventDate > CURDATE();

SELECT DATE_ADD(EventDate, INTERVAL 7 DAY) AS ExpectedEventDate
FROM Events;
    \end{lstlisting}
\end{itemize}
\begin{figure}[h!]
    \centering
    % Inserirai qui l'immagine di un Esempio di Query SQL complessa
    % \includegraphics[width=0.8\textwidth]{immagini/query_sql_esempio.png}
    \caption{Esempio di una query SQL che utilizza operatori e funzioni avanzate per filtrare e aggregare i dati.}
    \label{fig:query_sql_esempio}
\end{figure}
% Questo file conterrà le domande e gli esercizi per il capitolo "Basi di Dati".
% Sarà incluso nel main.tex subito dopo il riassunto teorico del capitolo.

\section*{Domande e Esercizi} % Usiamo * per non numerare la sezione nell'indice
\addcontentsline{toc}{section}{Domande e Esercizi: Basi di Dati} % Aggiunge la voce all'indice manualmente

\subsection*{Domande d'Esame Principali}
\addcontentsline{toc}{subsection}{Domande d'Esame Principali} % Aggiunge la voce all'indice

% --- Includi qui le singole domande e risposte da file separati ---
% Questo file contiene la domanda e risposta sul Modello ER, Normalizzazione e SQL per Corsi/Docenti/Appelli.
% Sarà incluso da domande_ed_esercizi_bd.tex.

\subsection*{Domanda: Modello ER, Normalizzazione e SQL (Corsi, Docenti, Appelli)}

\textbf{Domanda}: Il candidato descriva il modello concettuale Entità-Relazione (ER) e il concetto di forma normale nella progettazione logica. Produca un diagramma ER che rappresenti il contesto seguente: "Un corso è associato a un docente titolare e può includere più appelli d'esame durante l'anno accademico. Ogni appello d'esame può avere uno o più studenti iscritti." Successivamente il candidato fornisca una progettazione in forma tabellare del modello, normalizzando le relazioni, fornendo infine una query SQL per ottenere l'elenco di tutti gli iscritti a un determinato appello d'esame.

\paragraph{Risposta}:

\textbf{Descrizione del Modello Concettuale Entità-Relazione (ER)}
Il \textbf{Modello Entità-Relazione (ER)} è uno strumento di alto livello per la progettazione concettuale di database. Permette di rappresentare il mondo reale tramite entità (oggetti o concetti di interesse) e relazioni (associazioni tra entità). I suoi componenti principali sono:
\begin{itemize}
    \item \textbf{Entità}: Cose o oggetti reali (es. Studente, Docente). Rappresentate con rettangoli.
    \item \textbf{Attributi}: Proprietà delle entità o relazioni (es. Nome, CodiceFiscale). Rappresentati con ovali. Possono essere semplici, composti, multi-valore, derivati o chiavi (sottolineate).
    \item \textbf{Relazioni}: Associazioni logiche tra entità (es. Insegnare, Iscrizione). Rappresentate con rombi. Hanno cardinalità (min:max) e partecipazione.
\end{itemize}

\paragraph{Concetto di Forma Normale nella Progettazione Logica}
La \textbf{normalizzazione} è un processo di organizzazione dei dati in un database relazionale per ridurre la ridondanza e migliorare l'integrità dei dati. Si basa sulle forme normali:
\begin{itemize}
    \item \textbf{Prima Forma Normale (1NF)}: Tutti gli attributi sono atomici e ogni record è unico.
    \item \textbf{Seconda Forma Normale (2NF)}: È in 1NF e tutti gli attributi non-chiave dipendono completamente dalla chiave primaria (no dipendenze parziali).
    \item \textbf{Terza Forma Normale (3NF)}: È in 2NF e non contiene dipendenze transitive (nessun attributo non-chiave dipende da un altro attributo non-chiave).
\end{itemize}

\paragraph{Diagramma ER per il Contesto Specifico}
Il contesto da modellare è: "Un corso è associato a un docente titolare e può includere più appelli d'esame durante l'anno accademico. Ogni appello d'esame può avere uno o più studenti iscritti."

\begin{figure}[h!]
    \centering
    \includegraphics[width=0.9\textwidth]{capitoli/basi_di_dati/domande_teoriche/immagini/er_esame_stato_corsi_docenti_appelli.png}
    \caption{Diagramma Entità-Relazione per la gestione di corsi, docenti, appelli d'esame e studenti.}
    \label{fig:er_esame_stato_corsi_docenti_appelli}
\end{figure}

\textbf{Spiegazione del Diagramma:}
\begin{itemize}
    \item \textbf{Entità}: `Studente`, `Appello`, `Docente`, `Corso`, `Anno Accademico`.
    \item \textbf{Attributi impliciti/necessari}: Per completezza, assumiamo per `Studente` (Matricola PK, Nome, Cognome), per `Appello` (IDAppello PK, Data, Ora, Luogo), per `Docente` (IDDocente PK, Nome, Cognome), per `Corso` (IDCorso PK, NomeCorso) e per `Anno Accademico` (Anno PK).
    \item \textbf{Relazione "Iscrizione" (Studente - Appello)}: È una relazione N:M. Un `Studente` può iscriversi a zero o più `Appelli` (`(0,N)` lato Studente). Un `Appello` deve avere uno o più `Studenti` iscritti (`(1,N)` lato Appello). Potrebbe avere attributi sulla relazione come `Voto`, `DataIscrizione`.
    \item \textbf{Relazione "Insegnare" (Docente - Corso)}: È una relazione 1:N. Un `Docente` può insegnare a zero o più `Corsi` (`(0,N)` lato Docente). Un `Corso` è insegnato da uno e un solo `Docente` (`(1,1)` lato Corso), indicando il docente titolare.
    \item \textbf{Relazione "Includere" (Corso - Appello - Anno Accademico)}: Questa è modellata come una relazione ternaria. Un `Corso` può avere zero o più `Appelli` in un dato `Anno Accademico` (`(0,N)` lato Corso). Un `Appello` è sempre legato a uno e un solo `Corso` e a un solo `Anno Accademico` specifico (`(1,1)` lato Appello). Un `Anno Accademico` può includere zero o molti `Appelli` di `Corsi` (`(0,N)` lato Anno Accademico). Questa modellazione ternaria cattura il vincolo che un appello di un corso si tiene in un determinato anno accademico.
\end{itemize}

\paragraph{Progettazione in Forma Tabellare (Modello Relazionale Normalizzato)}
Traducendo lo schema ER in uno schema relazionale e applicando la normalizzazione (fino alla 3NF):

\begin{itemize}
    \item \textbf{DOCENTI} (\underline{IDDocente}, Nome, Cognome)
    \item \textbf{CORSI} (\underline{IDCorso}, NomeCorso, IDDocenteFK)
    \item \textbf{ANNI\_ACCADEMICI} (\underline{Anno})
    \item \textbf{APPELLI} (\underline{IDAppello}, Data, Ora, Luogo, IDCorsoFK, AnnoAccademicoFK)
    \item \textbf{STUDENTI} (\underline{Matricola}, Nome, Cognome, DataNascita)
    \item \textbf{ISCRIZIONI} (\underline{MatricolaFK, IDAppelloFK}, DataIscrizione, Voto)
\end{itemize}

\textbf{Motivazione della Normalizzazione}:
\begin{itemize}
    \item Tutte le tabelle sono in 1NF (attributi atomici, PK unica).
    \item Poiché non ci sono chiavi primarie composte in \lstinline{DOCENTI}, \lstinline{CORSI}, \lstinline{ANNI_ACCADEMICI}, \lstinline{APPELLI}, \lstinline{STUDENTI}, queste sono automaticamente in 2NF. Non ci sono dipendenze transitive, quindi sono in 3NF.
    \item Per \lstinline{ISCRIZIONI} (chiave composta: \lstinline{MatricolaFK}, \lstinline{IDAppelloFK}), gli attributi \lstinline{DataIscrizione} e \lstinline{Voto} dipendono da \textit{entrambe} le parti della chiave, quindi è in 2NF. Non ci sono dipendenze transitive, quindi è in 3NF.
    \item Questa scomposizione riduce la ridondanza (es. nomi di corsi/docenti non ripetuti in \lstinline{APPELLI}) e previene anomalie di aggiornamento, inserimento e cancellazione.
\end{itemize}

\paragraph{Query SQL per l'Elenco degli Iscritti a un Determinato Appello d'Esame}
Supponiamo di voler ottenere l'elenco di tutti gli studenti iscritti all'appello d'esame con `IDAppello = 'APP2025-001'`.

\begin{lstlisting}[language=SQL, caption={Query per l'elenco degli iscritti a un appello d'esame specifico}]
SELECT
    S.Nome AS StudentFirstName,
    S.Cognome AS StudentLastName,
    S.Matricola AS StudentMatricola,
    A.Data AS ExamDate,
    A.Ora AS ExamTime,
    C.NomeCorso AS CourseName
FROM
    STUDENTI S
JOIN
    ISCRIZIONI I ON S.Matricola = I.MatricolaFK
JOIN
    APPELLI A ON I.IDAppelloFK = A.IDAppello
JOIN
    CORSI C ON A.IDCorsoFK = C.IDCorso
WHERE
    A.IDAppello = 'APP2025-001';
\end{lstlisting}
Questa query connette le tabelle `STUDENTI`, `ISCRIZIONI`, `APPELLI` e `CORSI` per recuperare i dettagli degli studenti e dell'appello/corso corrispondente, filtrando per un ID di appello specifico.
% Questo file contiene la domanda e risposta sul Modello ER, Normalizzazione e SQL per Commissione/Candidati.
% Sarà incluso da domande_ed_esercizi_bd.tex.

\subsection*{Domanda: Modello ER, Normalizzazione e SQL (Commissione d'Esame di Stato)}

\textbf{Domanda}: Il candidato descriva il modello concettuale Entità-Relazione (ER) e il concetto di forma normale nella progettazione logica. Produca un diagramma ER per una base di dati che tenga conto per ogni sessione dell’esame di stato, i membri effettivi della commissione e i candidati, e per ogni esame memorizzi la data e i nomi dei candidati che lo hanno superato. Successivamente il candidato fornisca una progettazione in forma tabellare del modello, normalizzando le relazioni, fornendo infine una query SQL per ottenere l’elenco di tutti gli candidati che hanno superato l’esame.

\paragraph{Risposta}:

\textbf{Descrizione del Modello Concettuale Entità-Relazione (ER)}
Il \textbf{Modello Entità-Relazione (ER)} è uno strumento di alto livello per la progettazione concettuale di database. Permette di rappresentare il mondo reale tramite entità (oggetti o concetti di interesse) e relazioni (associazioni tra entità). I suoi componenti principali sono:
\begin{itemize}
    \item \textbf{Entità}: Rappresentano "cose" o "oggetti" reali (es. Candidato, Esame di stato, Membro Commissione). Rappresentate con rettangoli.
    \item \textbf{Attributi}: Proprietà delle entità o relazioni (es. Nome, Cognome, Data, Esito). Rappresentati con ovali.
    \item \textbf{Relazioni}: Associazioni logiche tra entità (es. Sostiene, Scrutina). Rappresentate con rombi. Hanno cardinalità (min:max) e partecipazione.
\end{itemize}

\paragraph{Concetto di Forma Normale nella Progettazione Logica}
La \textbf{normalizzazione} è un processo di organizzazione dei dati in un database relazionale per ridurre la ridondanza e migliorare l'integrità dei dati. Si basa sulle forme normali:
\begin{itemize}
    \item \textbf{Prima Forma Normale (1NF)}: Tutti gli attributi sono atomici e ogni record è unico.
    \item \textbf{Seconda Forma Normale (2NF)}: È in 1NF e tutti gli attributi non-chiave dipendono completamente dalla chiave primaria (no dipendenze parziali).
    \item \textbf{Terza Forma Normale (3NF)}: È in 2NF e non contiene dipendenze transitive (nessun attributo non-chiave dipende da un altro attributo non-chiave).
\end{itemize}

\paragraph{Diagramma ER per il Contesto Specifico}
Il contesto da modellare è la gestione delle sessioni d'esame di stato, dei membri della commissione, dei candidati, e il tracciamento dei candidati che hanno superato l'esame.

\begin{figure}[h!]
    \centering
    \includegraphics[width=0.9\textwidth]{capitoli/basi_di_dati/domande_teoriche/immagini/er_esame_stato_commissione_candidati.png}
    \caption{Diagramma Entità-Relazione per la gestione delle sessioni dell'Esame di Stato, Commissione e Candidati.}
    \label{fig:er_esame_stato_commissione_candidati}
\end{figure}

\textbf{Spiegazione del Diagramma:}
\begin{itemize}
    \item \textbf{Entità `Candidato`}: Rappresenta i partecipanti all'esame. Attributi: \lstinline{IdCandidato} (PK), \lstinline{Nome}, \lstinline{Cognome}.
    \item \textbf{Entità `Esame di stato`}: Rappresenta le singole sessioni d'esame. Attributi: \lstinline{Data} (PK).
    \item \textbf{Entità `Membro Commissione`}: Rappresenta i membri che compongono la commissione d'esame. Attributi: \lstinline{IdMembro} (PK), \lstinline{Nome}, \lstinline{Cognome}.
    \item \textbf{Relazione "Sostiene" (Candidato - Esame di stato)}: Relazione N:M.
    \begin{itemize}
        \item Cardinalità: Un \lstinline{Candidato} può sostenere uno o più \lstinline{Esami di stato} (\lstinline{(1,N)} lato Candidato). Un \lstinline{Esame di stato} è sostenuto da uno o più \lstinline{Candidati} (\lstinline{(1,N)} lato Esame di stato).
        \item Attributo: \lstinline{Esito} (indica se l'esame è stato superato, non superato, ecc.).
    \end{itemize}
    \item \textbf{Relazione "Scrutina" (Esame di stato - Membro Commissione)}: Relazione N:M.
    \begin{itemize}
        \item Cardinalità: Un \lstinline{Esame di stato} è scrutinato da uno o più \lstinline{Membri Commissione} (\lstinline{(1,N)} lato Esame di stato). Un \lstinline{Membro Commissione} può scrutinare uno o più \lstinline{Esami di stato} (\lstinline{(1,N)} lato Membro Commissione).
    \end{itemize}
\end{itemize}

\paragraph{Progettazione in Forma Tabellare (Modello Relazionale Normalizzato)}
Traducendo lo schema ER in uno schema relazionale e applicando la normalizzazione (fino alla 3NF):

\begin{itemize}
    \item \textbf{CANDIDATI} (\underline{IdCandidato}, Nome, Cognome)
    \item \textbf{ESAMI\_DI\_STATO} (\underline{Data})
    \item \textbf{MEMBRI\_COMMISSIONE} (\underline{IdMembro}, Nome, Cognome)
    \item \textbf{SOSTIENE} (\underline{IdCandidatoFK, DataEsameFK}, Esito)
    \item \textbf{SCRUTINA} (\underline{DataEsameFK, IdMembroFK})
\end{itemize}

\textbf{Motivazione della Normalizzazione}:
\begin{itemize}
    \item Tutte le tabelle sono in 1NF (attributi atomici, chiavi primarie uniche).
    \item Poiché non ci sono chiavi primarie composte in \textsf{CANDIDATI}, \textsf{ESAMI\_DI\_STATO}, \textsf{MEMBRI\_COMMISSIONE}, queste sono automaticamente in 2NF e 3NF in quanto hanno chiavi primarie singole e non presentano dipendenze transitive.
    \item Le tabelle \textsf{SOSTIENE} e \textsf{SCRUTINA} sono tabelle di associazione generate da relazioni N:M. I loro attributi (es. \textsf{Esito} per \textsf{SOSTIENE}) dipendono interamente dalla chiave composta (\textsf{IdCandidatoFK}, \textsf{DataEsameFK} per \textsf{SOSTIENE}), garantendo la 2NF. Non presentano dipendenze transitive, quindi sono in 3NF.
    \item Questa struttura riduce la ridondanza (es. nomi dei candidati non ripetuti nella relazione \textsf{SOSTIENE}) e previene anomalie di aggiornamento, inserimento e cancellazione.
\end{itemize}

\paragraph{Query SQL per ottenere l'elenco di tutti gli candidati che hanno superato l'esame}
Supponiamo di voler l'elenco di tutti i candidati che hanno superato un esame in una data specifica (es. il 21 luglio 2025).

\begin{lstlisting}[language=SQL, caption={Query per l'elenco dei candidati che hanno superato l'esame in una data specifica}]
SELECT
    C.Nome AS CandidateFirstName,
    C.Cognome AS CandidateLastName,
    E.Data AS ExamDate
FROM
    CANDIDATI C
JOIN
    SOSTIENE S ON C.IdCandidato = S.IdCandidatoFK
JOIN
    ESAMI_DI_STATO E ON S.DataEsameFK = E.Data
WHERE
    S.Esito = 'Superato' AND E.Data = '2025-07-21';
\end{lstlisting}
Questa query unisce le tabelle \lstinline{CANDIDATI}, \lstinline{SOSTIENE} e \lstinline{ESAMI_DI_STATO} per recuperare i nomi dei candidati e la data dell'esame, filtrando per l'esito 'Superato' e la data dell'esame desiderata.

% --- Inserirai qui le future domande/esercizi per Basi di Dati da file separati ---
% % Questo file contiene la domanda e risposta sul Modello ER, Normalizzazione e SQL per Commissione/Candidati.
% Sarà incluso da domande_ed_esercizi_bd.tex.

\subsection*{Domanda: Modello ER, Normalizzazione e SQL (Commissione d'Esame di Stato)}

\textbf{Domanda}: Il candidato descriva il modello concettuale Entità-Relazione (ER) e il concetto di forma normale nella progettazione logica. Produca un diagramma ER per una base di dati che tenga conto per ogni sessione dell’esame di stato, i membri effettivi della commissione e i candidati, e per ogni esame memorizzi la data e i nomi dei candidati che lo hanno superato. Successivamente il candidato fornisca una progettazione in forma tabellare del modello, normalizzando le relazioni, fornendo infine una query SQL per ottenere l’elenco di tutti gli candidati che hanno superato l’esame.

\paragraph{Risposta}:

\textbf{Descrizione del Modello Concettuale Entità-Relazione (ER)}
Il \textbf{Modello Entità-Relazione (ER)} è uno strumento di alto livello per la progettazione concettuale di database. Permette di rappresentare il mondo reale tramite entità (oggetti o concetti di interesse) e relazioni (associazioni tra entità). I suoi componenti principali sono:
\begin{itemize}
    \item \textbf{Entità}: Rappresentano "cose" o "oggetti" reali (es. Candidato, Esame di stato, Membro Commissione). Rappresentate con rettangoli.
    \item \textbf{Attributi}: Proprietà delle entità o relazioni (es. Nome, Cognome, Data, Esito). Rappresentati con ovali.
    \item \textbf{Relazioni}: Associazioni logiche tra entità (es. Sostiene, Scrutina). Rappresentate con rombi. Hanno cardinalità (min:max) e partecipazione.
\end{itemize}

\paragraph{Concetto di Forma Normale nella Progettazione Logica}
La \textbf{normalizzazione} è un processo di organizzazione dei dati in un database relazionale per ridurre la ridondanza e migliorare l'integrità dei dati. Si basa sulle forme normali:
\begin{itemize}
    \item \textbf{Prima Forma Normale (1NF)}: Tutti gli attributi sono atomici e ogni record è unico.
    \item \textbf{Seconda Forma Normale (2NF)}: È in 1NF e tutti gli attributi non-chiave dipendono completamente dalla chiave primaria (no dipendenze parziali).
    \item \textbf{Terza Forma Normale (3NF)}: È in 2NF e non contiene dipendenze transitive (nessun attributo non-chiave dipende da un altro attributo non-chiave).
\end{itemize}

\paragraph{Diagramma ER per il Contesto Specifico}
Il contesto da modellare è la gestione delle sessioni d'esame di stato, dei membri della commissione, dei candidati, e il tracciamento dei candidati che hanno superato l'esame.

\begin{figure}[h!]
    \centering
    \includegraphics[width=0.9\textwidth]{capitoli/basi_di_dati/domande_teoriche/immagini/er_esame_stato_commissione_candidati.png}
    \caption{Diagramma Entità-Relazione per la gestione delle sessioni dell'Esame di Stato, Commissione e Candidati.}
    \label{fig:er_esame_stato_commissione_candidati}
\end{figure}

\textbf{Spiegazione del Diagramma:}
\begin{itemize}
    \item \textbf{Entità `Candidato`}: Rappresenta i partecipanti all'esame. Attributi: \lstinline{IdCandidato} (PK), \lstinline{Nome}, \lstinline{Cognome}.
    \item \textbf{Entità `Esame di stato`}: Rappresenta le singole sessioni d'esame. Attributi: \lstinline{Data} (PK).
    \item \textbf{Entità `Membro Commissione`}: Rappresenta i membri che compongono la commissione d'esame. Attributi: \lstinline{IdMembro} (PK), \lstinline{Nome}, \lstinline{Cognome}.
    \item \textbf{Relazione "Sostiene" (Candidato - Esame di stato)}: Relazione N:M.
    \begin{itemize}
        \item Cardinalità: Un \lstinline{Candidato} può sostenere uno o più \lstinline{Esami di stato} (\lstinline{(1,N)} lato Candidato). Un \lstinline{Esame di stato} è sostenuto da uno o più \lstinline{Candidati} (\lstinline{(1,N)} lato Esame di stato).
        \item Attributo: \lstinline{Esito} (indica se l'esame è stato superato, non superato, ecc.).
    \end{itemize}
    \item \textbf{Relazione "Scrutina" (Esame di stato - Membro Commissione)}: Relazione N:M.
    \begin{itemize}
        \item Cardinalità: Un \lstinline{Esame di stato} è scrutinato da uno o più \lstinline{Membri Commissione} (\lstinline{(1,N)} lato Esame di stato). Un \lstinline{Membro Commissione} può scrutinare uno o più \lstinline{Esami di stato} (\lstinline{(1,N)} lato Membro Commissione).
    \end{itemize}
\end{itemize}

\paragraph{Progettazione in Forma Tabellare (Modello Relazionale Normalizzato)}
Traducendo lo schema ER in uno schema relazionale e applicando la normalizzazione (fino alla 3NF):

\begin{itemize}
    \item \textbf{CANDIDATI} (\underline{IdCandidato}, Nome, Cognome)
    \item \textbf{ESAMI\_DI\_STATO} (\underline{Data})
    \item \textbf{MEMBRI\_COMMISSIONE} (\underline{IdMembro}, Nome, Cognome)
    \item \textbf{SOSTIENE} (\underline{IdCandidatoFK, DataEsameFK}, Esito)
    \item \textbf{SCRUTINA} (\underline{DataEsameFK, IdMembroFK})
\end{itemize}

\textbf{Motivazione della Normalizzazione}:
\begin{itemize}
    \item Tutte le tabelle sono in 1NF (attributi atomici, chiavi primarie uniche).
    \item Poiché non ci sono chiavi primarie composte in \textsf{CANDIDATI}, \textsf{ESAMI\_DI\_STATO}, \textsf{MEMBRI\_COMMISSIONE}, queste sono automaticamente in 2NF e 3NF in quanto hanno chiavi primarie singole e non presentano dipendenze transitive.
    \item Le tabelle \textsf{SOSTIENE} e \textsf{SCRUTINA} sono tabelle di associazione generate da relazioni N:M. I loro attributi (es. \textsf{Esito} per \textsf{SOSTIENE}) dipendono interamente dalla chiave composta (\textsf{IdCandidatoFK}, \textsf{DataEsameFK} per \textsf{SOSTIENE}), garantendo la 2NF. Non presentano dipendenze transitive, quindi sono in 3NF.
    \item Questa struttura riduce la ridondanza (es. nomi dei candidati non ripetuti nella relazione \textsf{SOSTIENE}) e previene anomalie di aggiornamento, inserimento e cancellazione.
\end{itemize}

\paragraph{Query SQL per ottenere l'elenco di tutti gli candidati che hanno superato l'esame}
Supponiamo di voler l'elenco di tutti i candidati che hanno superato un esame in una data specifica (es. il 21 luglio 2025).

\begin{lstlisting}[language=SQL, caption={Query per l'elenco dei candidati che hanno superato l'esame in una data specifica}]
SELECT
    C.Nome AS CandidateFirstName,
    C.Cognome AS CandidateLastName,
    E.Data AS ExamDate
FROM
    CANDIDATI C
JOIN
    SOSTIENE S ON C.IdCandidato = S.IdCandidatoFK
JOIN
    ESAMI_DI_STATO E ON S.DataEsameFK = E.Data
WHERE
    S.Esito = 'Superato' AND E.Data = '2025-07-21';
\end{lstlisting}
Questa query unisce le tabelle \lstinline{CANDIDATI}, \lstinline{SOSTIENE} e \lstinline{ESAMI_DI_STATO} per recuperare i nomi dei candidati e la data dell'esame, filtrando per l'esito 'Superato' e la data dell'esame desiderata.
% \input{capitoli/basi_di_dati/esercizi/bd_esercizio_normalizzazione_complessa.tex} % Abilitare quando avrai le domande/esercizi per Basi di Dati

% Capitolo 3: Reti di Calcolatori
\chapter{Reti di Calcolatori}

Le \textbf{reti di calcolatori} sono sistemi che permettono a dispositivi interconnessi di scambiare dati e condividere risorse. Sono la base di quasi ogni infrastruttura informatica moderna, dal World Wide Web alle reti aziendali locali.

\section{Modello di Comunicazione Client/Server}
Il \textbf{modello Client/Server} è un'architettura di rete distribuita in cui i client (richiedenti servizi) e i server (fornitori di servizi) sono entità separate che comunicano su una rete. È il modello dominante per la maggior parte delle applicazioni web e molte applicazioni enterprise.

\subsection{Funzionamento del Modello Client/Server}
\begin{itemize}
    \item \textbf{Client}: Invia richieste di servizi al server, riceve le risposte e le presenta all'utente. Tipicamente è l'applicazione utente (es. browser web, app mobile).
    \item \textbf{Server}: Ascolta le richieste dei client, le elabora (es. recupera dati, esegue calcoli), e invia le risposte al client. Gestisce le risorse condivise (database, file, stampanti).
    \item \textbf{Comunicazione}: Avviene tramite protocolli di rete (es. TCP/IP) su porte specifiche. Il client avvia la connessione e la richiesta.
\end{itemize}

\subsection{Vantaggi e Svantaggi del Modello Client/Server}
\begin{itemize}
    \item \textbf{Vantaggi}:
    \begin{itemize}
        \item \textbf{Centralizzazione}: Gestione centralizzata di dati e risorse, facilitando la sicurezza e la manutenzione.
        \item \textbf{Scalabilità}: Possibilità di scalare il server per gestire più richieste o aggiungere più client alla rete.
        \item \textbf{Sicurezza}: Controllo più agevole degli accessi e dei permessi sui dati centralizzati.
        \item \textbf{Manutenzione Facilitata}: Aggiornamenti e backup possono essere eseguiti sul server senza influenzare i client.
    \end{itemize}
    \item \textbf{Svantaggi}:
    \begin{itemize}
        \item \textbf{Single Point of Failure (Punto Singolo di Fallimento)}: Se il server si blocca, tutti i client perdono l'accesso ai servizi.
        \item \textbf{Collo di Bottiglia del Server}: Un server sovraccarico può rallentare l'intera rete.
        \item \textbf{Costo}: I server e la loro manutenzione possono essere costosi.
    \end{itemize}
\end{itemize}

\section{Protocollo HTTP}
L'\textbf{Hypertext Transfer Protocol (HTTP)} è il protocollo applicativo fondamentale per il World Wide Web. Opera nel modello client/server ed è stateless.

\subsection{Funzionamento e Applicazione in HTTP}
\begin{itemize}
    \item Un \textbf{client} (tipicamente un browser web) invia una \textbf{richiesta HTTP} (es. GET, POST, PUT, DELETE) a un \textbf{server web}. La richiesta include URL, metodo, header e, opzionalmente, un body.
    \item Il \textbf{server} riceve la richiesta, la elabora (es. recupera una pagina web, esegue uno script), e invia una \textbf{risposta HTTP}. La risposta include uno stato (es. 200 OK, 404 Not Found), header e il body (es. il contenuto HTML della pagina richiesta).
    \item La comunicazione avviene tipicamente su TCP/IP (porta 80 per HTTP, 443 per HTTPS).
\end{itemize}
\textbf{Esempio}: Quando un utente digita un URL nel browser, il browser è il client che invia una richiesta HTTP al server. Il server risponde con la pagina HTML e le risorse associate, che il browser poi renderizza.

\subsection{Connessioni Stateless e Stateful}

\subsubsection{Stateless Connection (Connessione Senza Stato)}
\begin{itemize}
    \item \textbf{Definizione}: Ogni richiesta inviata dal client al server è completamente indipendente e autocontenuta. Il server non mantiene alcuna informazione (stato) sulle richieste precedenti del client. Ogni richiesta include tutte le informazioni necessarie per essere elaborata.
    \item \textbf{Vantaggi}: Scalabilità elevata (il server non deve allocare memoria per lo stato di ogni client), resilienza (se un server si blocca, un altro può prendere il suo posto senza perdere lo stato), semplicità di progettazione lato server.
    \item \textbf{Svantaggi}: Richiede che ogni richiesta contenga potenzialmente informazioni ridondanti (es. credenziali di autenticazione), e può essere meno efficiente per operazioni che richiedono una sequenza di passaggi.
    \item \textbf{HTTP come Stateless}: HTTP è intrinsecamente stateless. Ogni richiesta HTTP è trattata come se fosse la prima e unica richiesta tra il client e il server.

\end{itemize}

\subsubsection{Stateful Connection (Connessione Con Stato)}
\begin{itemize}
    \item \textbf{Definizione}: Il server mantiene e ricorda lo stato delle interazioni passate con un client per un certo periodo di tempo. Le richieste successive possono fare riferimento a questo stato.
    \textbf{Vantaggi}: Minore ridondanza di informazioni nelle richieste successive, può semplificare la logica client per sequenze di operazioni complesse.
    \item \textbf{Svantaggi}: Minore scalabilità (il server deve mantenere lo stato per ogni client attivo, consumando risorse), minore resilienza (se il server che detiene lo stato si blocca, la sessione del client viene persa), complessità maggiore.
    \item \textbf{Esempi}: Una connessione TCP (a un livello più basso) è stateful; una sessione di login a un database.
\end{itemize}

\subsection{Metodi per Gestire la Persistenza dello Stato in HTTP}
Dato che HTTP è stateless, per costruire applicazioni web interattive che richiedono il mantenimento dello stato (es. carrelli della spesa, sessioni utente), sono stati sviluppati diversi meccanismi:
\begin{itemize}
    \item \textbf{Cookies}: Piccoli frammenti di dati che il server invia al browser del client e che il browser memorizza. Ad ogni richiesta successiva verso lo stesso server, il browser invia nuovamente i cookie al server. Usati per ID di sessione, preferenze utente, tracciamento.
    \item \textbf{Session IDs (ID di Sessione)}: Il server crea un ID univoco per ogni sessione utente e lo invia al client (spesso tramite cookie). Il server memorizza i dati della sessione sul proprio lato (nel database o in memoria cache) associati a quell'ID. Usati per mantenere lo stato di login, carrelli della spesa.
    \item \textbf{URL Rewriting (Parametri URL)}: Lo stato viene incorporato direttamente nell'URL come parametri. Usato quando i cookie non sono disponibili.
    \item \textbf{Hidden Form Fields}: Dati nascosti all'interno di moduli HTML che vengono inviati con ogni richiesta POST. Usati per mantenere lo stato tra le pagine di un modulo multi-step.
    \item \textbf{Web Storage (Local Storage, Session Storage)}: API JavaScript che permettono alle applicazioni web di memorizzare dati nel browser del client (Local Storage persistente, Session Storage per la durata della sessione del browser). Usati per memorizzare dati client-side, cache di dati.
\end{itemize}

\section{Standard ISO/OSI}
Il \textbf{modello ISO/OSI (Open Systems Interconnection)} è un modello concettuale che descrive come i sistemi di comunicazione in una rete interagiscono e cooperano. È diviso in sette strati (layer), ciascuno con responsabilità specifiche, che operano sopra lo strato precedente e forniscono servizi a quello successivo.

\subsection{Struttura a Sette Strati del Modello OSI}
\begin{enumerate}
    \item \textbf{Strato Fisico (Physical Layer)}: Gestisce la trasmissione e ricezione di flussi di bit non strutturati e grezzi su un mezzo fisico. Definisce le specifiche elettriche, meccaniche, procedurali e funzionali. (Es. cavi Ethernet, Wi-Fi, connettori).
    \item \textbf{Strato di Collegamento Dati (Data Link Layer)}: Fornisce la trasmissione di dati da nodo a nodo, rilevando e correggendo potenzialmente gli errori che possono verificarsi a livello fisico. Gestisce l'indirizzamento MAC e il controllo del flusso. (Es. Ethernet, PPP).
    \item \textbf{Strato di Rete (Network Layer)}: Gestisce l'instradamento dei pacchetti attraverso la rete (routing). È responsabile dell'indirizzamento logico (IP) e della selezione del percorso migliore. (Es. IP, ICMP).
    \item \textbf{Strato di Trasporto (Transport Layer)}: Fornisce la comunicazione end-to-end tra processi su host diversi. Assicura la consegna affidabile dei dati, il controllo di flusso e il controllo della congestione. (Es. TCP, UDP).
    \item \textbf{Strato di Sessione (Session Layer)}: Stabilisce, gestisce e termina le sessioni di comunicazione tra applicazioni. Gestisce la sincronizzazione e il dialogo.
    \item \textbf{Strato di Presentazione (Presentation Layer)}: Si occupa della sintassi e della semantica dei dati scambiati. Traduce i dati tra il formato dell'applicazione e il formato di rete, e gestisce la crittografia/decrittografia e la compressione.
    \item \textbf{Strato di Applicazione (Application Layer)}: Fornisce servizi di rete direttamente alle applicazioni dell'utente finale. (Es. HTTP, FTP, SMTP, DNS).
\end{enumerate}

\section{Livello di Trasporto (TCP vs UDP)}
Il \textbf{Livello di Trasporto} è il quarto strato del modello OSI e fornisce servizi di comunicazione end-to-end tra applicazioni in esecuzione su host diversi. I due protocolli principali a questo livello sono TCP e UDP.

\subsection{TCP (Transmission Control Protocol)}
\textbf{TCP} è un protocollo orientato alla connessione, affidabile e con controllo di flusso e congestione.
\begin{itemize}
    \item \textbf{Orientato alla Connessione}: Stabilisce una connessione (handshake a tre vie) prima di iniziare la trasmissione dei dati e la termina esplicitamente.
    \item \textbf{Affidabile}: Garantisce la consegna dei dati, senza perdite o duplicazioni, e nell'ordine corretto.
    \begin{itemize}
        \item \textbf{Numerazione e Riconoscimenti (ACK)}: Ogni segmento inviato è numerato e il mittente si aspetta un riconoscimento (ACK) dal destinatario. Se un ACK non arriva entro un certo tempo, il segmento viene ritrasmesso.
        \item \textbf{Checksum}: Utilizza un checksum per rilevare errori nei dati.
    \end{itemize}
    \item \textbf{Controllo di Flusso}: Impedisce a un mittente veloce di sovraccaricare un destinatario lento. Il destinatario comunica al mittente quanto spazio buffer è disponibile (finestra di ricezione).
    \item \textbf{Controllo di Congestione}: Evita che un mittente invii troppi dati in una rete congestionata, riducendo la velocità di trasmissione se rileva congestione (es. tramite perdite di pacchetti o ritardi).
    \item \textbf{Segmentazione e Riasssemblaggio}: Spezza i dati dell'applicazione in segmenti più piccoli per la trasmissione e li riassembla alla destinazione.
    \item \textbf{Applicazioni Tipiche}: Web Browse (HTTP), trasferimento file (FTP), email (SMTP), connessioni sicure (SSH).
\end{itemize}

\subsection{UDP (User Datagram Protocol)}
\textbf{UDP} è un protocollo senza connessione, inaffidabile e che non implementa controllo di flusso o congestione.
\begin{itemize}
    \item \textbf{Senza Connessione}: Non stabilisce una connessione preliminare. Ogni datagramma viene inviato indipendentemente.
    \item \textbf{Inaffidabile (Best-Effort)}: Non garantisce la consegna dei dati, l'ordine di arrivo, né che non ci siano duplicazioni. Non ci sono ACK né ritrasmissioni automatiche.
    \item \textbf{Nessun Controllo di Flusso/Congestione}: Trasmette i dati alla massima velocità possibile senza preoccuparsi della capacità del destinatario o della rete.
    \item \textbf{Overhead Minimo}: Ha un header molto piccolo, il che lo rende molto efficiente in termini di overhead.
    \item \textbf{Applicazioni Tipiche}: Streaming multimediale (audio/video), VoIP, DNS, giochi online, dove la velocità è più importante dell'affidabilità perfetta (piccole perdite possono essere accettabili).
\end{itemize}
% Questo file conterrà le domande e gli esercizi per il capitolo "Reti di Calcolatori".
% Sarà incluso nel main.tex subito dopo il riassunto teorico del capitolo.

\section*{Domande e Esercizi} % Usiamo * per non numerare la sezione nell'indice
\addcontentsline{toc}{section}{Domande e Esercizi: Reti di Calcolatori} % Aggiunge la voce all'indice manualmente

\subsection*{Domande d'Esame Principali}
\addcontentsline{toc}{subsection}{Domande d'Esame Principali} % Aggiunge la voce all'indice

% --- Includi qui le singole domande e risposte da file separati ---
% Questo file contiene la domanda e risposta sullo Standard ISO/OSI e Livello di Trasporto.
% Sarà incluso da domande_ed_esercizi_reti.tex.

\subsection*{Domanda: Standard ISO/OSI e Livello di Trasporto}

\textbf{Domanda}: Il candidato dia una panoramica della standard architetturale per reti di calcolatori interoperabili ISO/OSI, approfondendo in particolare il livello di trasporto motivando e spiegando le scelte alla base di alcune tecniche di consegna affidabile del dato.

\paragraph{Risposta}:

\textbf{Panoramica sullo Standard Architetturale ISO/OSI}
Il \textbf{modello ISO/OSI (Open Systems Interconnection)} è un modello concettuale a sette strati (layer) che descrive come i sistemi di comunicazione in una rete interagiscono e cooperano. Ogni strato ha responsabilità specifiche, opera sopra lo strato precedente e fornisce servizi a quello successivo, facilitando l'interoperabilità tra sistemi eterogenei.
La struttura a sette strati è la seguente:
\begin{enumerate}
    \item \textbf{Strato Fisico (Physical Layer)}: Gestisce la trasmissione e ricezione di flussi di bit non strutturati su un mezzo fisico (es. cavi Ethernet, Wi-Fi). Definisce specifiche elettriche e meccaniche.
    \item \textbf{Strato di Collegamento Dati (Data Link Layer)}: Fornisce la trasmissione di dati da nodo a nodo, rilevando e potenzialmente correggendo errori. Gestisce l'indirizzamento MAC e il controllo del flusso (es. Ethernet, PPP).
    \item \textbf{Strato di Rete (Network Layer)}: Gestisce l'instradamento dei pacchetti attraverso la rete (routing) ed è responsabile dell'indirizzamento logico (IP) e della selezione del percorso migliore (es. IP, ICMP).
    \item \textbf{Strato di Trasporto (Transport Layer)}: Fornisce la comunicazione end-to-end tra processi su host diversi. Assicura la consegna affidabile dei dati, il controllo di flusso e il controllo della congestione (es. TCP, UDP).
    \item \textbf{Strato di Sessione (Session Layer)}: Stabilisce, gestisce e termina le sessioni di comunicazione tra applicazioni, gestendo sincronizzazione e dialogo.
    \item \textbf{Strato di Presentazione (Presentation Layer)}: Si occupa della sintassi e della semantica dei dati scambiati, traducendo i dati e gestendo crittografia/decrittografia e compressione.
    \item \textbf{Strato di Applicazione (Application Layer)}: Fornisce servizi di rete direttamente alle applicazioni dell'utente finale (es. HTTP, FTP, SMTP, DNS).
\end{enumerate}

\paragraph{Approfondimento sul Livello di Trasporto e Tecniche di Consegna Affidabile del Dato}
Il Livello di Trasporto è cruciale per stabilire una comunicazione logica tra applicazioni che risiedono su host diversi. I due protocolli principali a questo livello sono TCP (Transmission Control Protocol) e UDP (User Datagram Protocol), con TCP che si distingue per la sua capacità di fornire una consegna affidabile del dato.

\textbf{TCP (Transmission Control Protocol)}:
TCP è un protocollo orientato alla connessione, affidabile e con controllo di flusso e congestione. È la scelta preferita per applicazioni che richiedono garanzie di consegna dei dati.
Le \textbf{scelte alla base della consegna affidabile del dato} in TCP includono:
\begin{itemize}
    \item \textbf{Numerazione dei Segmenti e Riconoscimenti (ACK - Acknowledgement)}: Ogni segmento di dati inviato da TCP è numerato. Il mittente si aspetta un ACK dal destinatario per ogni segmento ricevuto. Se un ACK non viene ricevuto entro un determinato timeout, il segmento viene considerato perso e ritrasmesso. Questo meccanismo garantisce che tutti i dati arrivino a destinazione e nell'ordine corretto, gestendo perdite e duplicazioni.
    \item \textbf{Retrasmissione Selettiva o Go-Back-N}: Se vengono rilevate perdite (es. tramite timeout o ACK duplicati), TCP può ritrasmettere solo i segmenti persi (selettiva) o ritrasmettere dal primo segmento non riconosciuto (Go-Back-N), assicurando la completezza dei dati.
    \item \textbf{Checksum}: TCP calcola un checksum per ogni segmento e lo include nell'header. Il destinatario ricalcola il checksum e lo confronta: se non corrispondono, il segmento viene scartato, garantendo l'integrità dei dati.
    \item \textbf{Controllo di Flusso (Flow Control)}: Impedisce che un mittente veloce sovraccarichi il buffer di un destinatario lento. Il destinatario comunica al mittente la dimensione della sua "finestra di ricezione" (receive window), cioè la quantità di spazio buffer disponibile. Il mittente non invierà più dati di quanto il destinatario possa gestire, prevenendo la perdita di dati dovuta a buffer pieni.
    \item \textbf{Controllo di Congestione (Congestion Control)}: Evita che un mittente invii troppi dati in una rete congestionata, il che potrebbe portare a un collasso della rete. TCP rileva la congestione (es. tramite perdite di pacchetti o ritardi) e riduce dinamicamente la sua velocità di trasmissione, rallentando fino a quando la congestione non diminuisce. Implementa algoritmi come Slow Start, Congestion Avoidance, Fast Retransmit, e Fast Recovery.
\end{itemize}

\textbf{UDP (User Datagram Protocol)}:
In contrasto, UDP è un protocollo senza connessione e inaffidabile ("best-effort"). Non garantisce la consegna, l'ordine o l'assenza di duplicazioni. Ha un overhead minimo ed è utilizzato per applicazioni dove la velocità è più critica dell'affidabilità perfetta (es. streaming multimediale, VoIP, DNS, giochi online), e dove l'applicazione stessa può gestire eventuali perdite o ritrasmissioni.

In sintesi, il Livello di Trasporto offre una scelta tra un servizio affidabile e robusto (TCP) e uno veloce e a basso overhead (UDP), permettendo alle applicazioni di scegliere il protocollo più adatto alle proprie esigenze.

% --- Inserirai qui le future domande/esercizi per Reti di Calcolatori da file separati ---
\subsection*{Altre Possibili Domande}
\addcontentsline{toc}{subsection}{Altre Possibili Domande}
% Questo file contiene la domanda e risposta sul Modello Client/Server e HTTP.
% Sarà incluso da domande_ed_esercizi_reti.tex.

\subsection*{Domanda: Modello Client/Server e Applicazione in HTTP}

\textbf{Domanda}: Come funziona il Modello Client/Server, quali sono i suoi vantaggi e svantaggi, e come si applica nel contesto HTTP?

\textbf{Risposta}:

Il \textbf{modello Client/Server} è un'architettura di rete distribuita in cui i client (richiedenti servizi) e i server (fornitori di servizi) sono entità separate che comunicano su una rete.
\paragraph{Funzionamento}: Il Client invia richieste di servizi al Server, che le elabora e invia risposte. La comunicazione avviene tramite protocolli di rete (es. TCP/IP) su porte specifiche.
\paragraph{Vantaggi}:
\begin{itemize}
    \item \textbf{Centralizzazione}: Gestione centralizzata di dati e risorse (sicurezza, manutenzione).
    \item \textbf{Scalabilità}: Possibilità di scalare il server per gestire più richieste.
    \item \textbf{Manutenzione Facilitata}: Aggiornamenti sul server senza influenzare i client.
\end{itemize}
\paragraph{Svantaggi}:
\begin{itemize}
    \item \textbf{Single Point of Failure}: Se il server si blocca, tutti i client perdono l'accesso.
    \item \textbf{Collo di Bottiglia}: Un server sovraccarico rallenta l'intera rete.
    \item \textbf{Costo}: Server e manutenzione possono essere costosi.
\end{itemize}
\paragraph{Applicazione in HTTP}: L'\textbf{Hypertext Transfer Protocol (HTTP)} è il protocollo applicativo fondamentale per il World Wide Web, che opera nel modello client/server ed è intrinsecamente stateless. Il client (browser) invia richieste HTTP (GET, POST) a un server web, che risponde con lo stato e il contenuto richiesto (es. pagina HTML).
% Questo file contiene la domanda e risposta su Connessioni Stateless/Stateful in HTTP.
% Sarà incluso da domande_ed_esercizi_reti.tex.

\subsection*{Domanda: Connessioni Stateless e Stateful in HTTP}

\textbf{Domanda}: Qual è la differenza tra connessioni stateless e stateful nel contesto HTTP, e quali metodi sono utilizzati per gestire la persistenza dello stato in HTTP?

\textbf{Risposta}:

\paragraph{Connessioni Stateless e Stateful}
\begin{itemize}
    \item \textbf{Stateless Connection (Senza Stato)}: Ogni richiesta dal client al server è indipendente e autocontenuta; il server non ricorda le interazioni passate.
    \begin{itemize}
        \item \textbf{Vantaggi}: Alta scalabilità, resilienza, semplicità lato server.
        \item \textbf{Svantaggi}: Richiede informazioni ridondanti per ogni richiesta.
        \item \textbf{HTTP}: È intrinsecamente stateless.
    \end{itemize}
    \item \textbf{Stateful Connection (Con Stato)}: Il server mantiene e ricorda lo stato delle interazioni passate con un client per un periodo.
    \begin{itemize}
        \item \textbf{Vantaggi}: Meno ridondanza, semplifica logica per sequenze complesse.
        \item \textbf{Svantaggi}: Minore scalabilità, minore resilienza, maggiore complessità.
    \end{itemize}
\end{itemize}
\paragraph{Metodi per Gestire la Persistenza dello Stato in HTTP (per simulare Stateful)}:
Dato che HTTP è stateless, si utilizzano meccanismi come:
\begin{itemize}
    \item \textbf{Cookies}: Piccoli dati inviati dal server al browser, che li memorizza e li invia con richieste successive (per ID sessione, preferenze utente).
    \item \textbf{Session IDs}: Il server crea un ID univoco e lo invia al client (spesso tramite cookie); il server memorizza i dati della sessione.
    \item \textbf{URL Rewriting}: Lo stato incorporato direttamente nell'URL come parametri.
    \item \textbf{Hidden Form Fields}: Dati nascosti in moduli HTML inviati con richieste POST.
    \item \textbf{Web Storage} (Local Storage, Session Storage): API JavaScript per memorizzare dati nel browser del client.
\end{itemize}
% Questo file contiene la domanda e risposta sulla struttura del Modello ISO/OSI.
% Sarà incluso da domande_ed_esercizi_reti.tex.

\subsection*{Domanda: Struttura a Sette Strati del Modello ISO/OSI}

\textbf{Domanda}: Qual è la struttura a sette strati del Modello ISO/OSI e quali sono le responsabilità di ciascun strato?

\textbf{Risposta}:

Il \textbf{modello ISO/OSI (Open Systems Interconnection)} è un modello concettuale che descrive come i sistemi di comunicazione in una rete interagiscono. È diviso in sette strati (layer), ciascuno con responsabilità specifiche:
\begin{enumerate}
    \item \textbf{Strato Fisico (Physical Layer)}: Gestisce la trasmissione di bit grezzi su un mezzo fisico (es. cavi Ethernet).
    \item \textbf{Strato di Collegamento Dati (Data Link Layer)}: Fornisce la trasmissione dati da nodo a nodo, rilevando errori e gestendo l'indirizzamento MAC (es. Ethernet).
    \item \textbf{Strato di Rete (Network Layer)}: Gestisce l'instradamento dei pacchetti (routing) e l'indirizzamento logico (IP).
    \item \textbf{Strato di Trasporto (Transport Layer)}: Fornisce la comunicazione end-to-end tra processi, assicurando la consegna affidabile dei dati, controllo di flusso e congestione (es. TCP, UDP).
    \item \textbf{Strato di Sessione (Session Layer)}: Stabilisce, gestisce e termina le sessioni di comunicazione tra applicazioni.
    \item \textbf{Strato di Presentazione (Presentation Layer)}: Si occupa della sintassi e semantica dei dati, inclusa crittografia/decrittografia e compressione.
    \item \textbf{Strato di Applicazione (Application Layer)}: Fornisce servizi di rete direttamente alle applicazioni dell'utente finale (es. HTTP, FTP).
\end{enumerate}
% Questo file contiene la domanda e risposta sul confronto TCP vs UDP.
% Sarà incluso da domande_ed_esercizi_reti.tex.

\subsection*{Domanda: Confronto tra TCP e UDP}

\textbf{Domanda}: Confronta il protocollo TCP con il protocollo UDP, evidenziando le loro caratteristiche principali, vantaggi, svantaggi e applicazioni tipiche.

\textbf{Risposta}:

Il \textbf{Livello di Trasporto} del modello OSI gestisce la comunicazione end-to-end tra processi. I protocolli principali sono TCP e UDP.
\paragraph{TCP (Transmission Control Protocol)}:
\begin{itemize}
    \item \textbf{Orientato alla Connessione}: Stabilisce una connessione (handshake a tre vie).
    \item \textbf{Affidabile}: Garantisce la consegna ordinata e senza perdite (numerazione, ACK, ritrasmissioni).
    \item \textbf{Controllo di Flusso}: Previene il sovraccarico del destinatario (finestra di ricezione).
    \item \textbf{Controllo di Congestione}: Riduce la velocità in reti congestionate.
    \item \textbf{Applicazioni Tipiche}: Web (HTTP), trasferimento file (FTP), email (SMTP), SSH.
\end{itemize}
\paragraph{UDP (User Datagram Protocol)}:
\begin{itemize}
    \item \textbf{Senza Connessione}: Invia datagrammi indipendenti senza handshake.
    \item \textbf{Inaffidabile (Best-Effort)}: Non garantisce consegna, ordine o assenza di duplicazioni.
    \item \textbf{Nessun Controllo}: Non implementa controllo di flusso o congestione.
    \item \textbf{Overhead Minimo}: Header piccolo, molto efficiente.
    \item \textbf{Applicazioni Tipiche}: Streaming multimediale (VoIP, video), DNS, giochi online (dove la velocità è prioritaria).
\end{itemize}
% Questo file contiene la domanda e risposta sul Subnetting.
% Sarà incluso da domande_ed_esercizi_reti.tex.

\subsection*{Domanda: Concetto e Calcolo del Subnetting}

\textbf{Domanda}: Spiega il concetto di subnetting, i suoi obiettivi e illustra con un esempio pratico come si calcolano le sottoreti per una data rete IP.

\textbf{Risposta}:

Il \textbf{subnetting} è il processo di divisione di una rete IP più grande in sottoreti più piccole e gestibili, per migliorare efficienza, sicurezza e gestione degli indirizzi IP.
\paragraph{Obiettivi}:
\begin{itemize}
    \item \textbf{Efficienza nell'uso degli indirizzi IP}: Sfruttare meglio lo spazio IP.
    \item \textbf{Riduzione del traffico di rete}: Confinare il traffico di broadcast.
    \item \textbf{Miglioramento della Sicurezza}: Isolare segmenti di rete.
    \item \textbf{Facilitazione della Gestione}: Reti più piccole più facili da gestire.
\end{itemize}
Un indirizzo IP (IPv4) ha 32 bit (parte di rete + parte host). La \textbf{maschera di sottorete} (subnet mask) separa queste due parti. La notazione \textbf{CIDR} (\lstinline{IP/prefisso}) indica la lunghezza del prefisso di rete.
\paragraph{Calcolo del Subnetting (Esempio Pratico)}:
\textbf{Scenario}: Una rete \lstinline{192.168.1.0/24} deve essere divisa in 4 sottoreti per ospitare almeno 50 host per sottorete.
\begin{enumerate}
    \item \textbf{Bit per sottoreti}: $2^n \ge 4 \Rightarrow n=2$ bit.
    \item \textbf{Bit per host}: Maschera originale /24 (8 bit host). Usando 2 bit per sottoreti, rimangono $8-2=6$ bit per host. Max host per sottorete: $2^6-2=62$ host (sufficiente).
    \item \textbf{Nuova maschera}: $24+2=26$. Maschera \lstinline{/26} (\lstinline{255.255.255.192}).
    \item \textbf{Sottoreti valide} (intervalli di 64 nell'ultimo ottetto):
    \begin{itemize}
        \item Sottorete 1: \lstinline{192.168.1.0/26} (Host: .1 a .62, Broadcast: .63)
        \item Sottorete 2: \lstinline{192.168.1.64/26} (Host: .65 a .126, Broadcast: .127)
        \item Sottorete 3: \lstinline{192.168.1.128/26} (Host: .129 a .190, Broadcast: .191)
        \item Sottorete 4: \lstinline{192.168.1.192/26} (Host: .193 a .254, Broadcast: .255)
    \end{itemize}
\end{enumerate}
% \subsection*{Esercizi}
% \addcontentsline{toc}{subsection}{Esercizi}
% \input{capitoli/reti_di_calcolatori/esercizi/reti_esercizio_routing.tex} % Abilitare quando avrai le domande/esercizi per Reti

% Capitolo 4: Programmazione Orientata agli Oggetti
\chapter{Programmazione Orientata agli Oggetti (e Fondamenti)}

La \textbf{Programmazione Orientata agli Oggetti (OOP)} è un paradigma di programmazione basato sul concetto di "oggetti", che possono contenere dati e codice. È uno dei paradigmi più diffusi per lo sviluppo di software moderno.

\section{Concetti Base della Programmazione Orientata agli Oggetti (POO)}
La POO si fonda su alcuni pilastri fondamentali che ne definiscono la struttura e il funzionamento:
\begin{itemize}
    \item \textbf{Classe}: Una blueprint o un modello per creare oggetti. Definisce le proprietà (attributi/campi) e i comportamenti (metodi/funzioni) che gli oggetti di quel tipo avranno. Non è un'entità fisica, ma una definizione logica.
    \item \textbf{Oggetto}: Un'istanza di una classe. È un'entità concreta che ha uno stato (valori specifici degli attributi) e un comportamento (i metodi che può eseguire).
    \item \textbf{Incapsulamento (Encapsulation)}: Il principio di raggruppare i dati (attributi) e le funzioni (metodi) che operano su quei dati all'interno di un'unica unità (la classe). Protegge i dati interni dall'accesso diretto esterno, permettendone la manipolazione solo tramite metodi pubblici della classe. Questo migliora la sicurezza e la manutenibilità del codice.
    \item \textbf{Ereditarietà (Inheritance)}: Un meccanismo che permette a una classe (sottoclasse o classe derivata) di ereditare proprietà e comportamenti da un'altra classe (superclasse o classe base). Promuove il riutilizzo del codice e la creazione di gerarchie di classi che riflettono relazioni "è un tipo di".
    \item \textbf{Polimorfismo (Polymorphism)}: Il concetto che un oggetto possa assumere molte forme. In OOP, si riferisce alla capacità di oggetti di classi diverse di rispondere allo stesso messaggio (chiamata di metodo) in modi diversi, o alla capacità di un'interfaccia di riferirsi a oggetti di diverse classi che la implementano. Si manifesta tramite:
    \begin{itemize}
        \item \textbf{Overriding}: Una sottoclasse fornisce un'implementazione specifica di un metodo già definito nella sua superclasse.
        \item \textbf{Overloading}: Definire più metodi con lo stesso nome all'interno della stessa classe, ma con liste di parametri diverse (numero, tipo o ordine).
    \end{itemize}
\end{itemize}

\section{Strutture Dati Astratte (ADT) e Fondamentali di Programmazione}

\subsection{Abstract Data Type (ADT)}
Un \textbf{Abstract Data Type (ADT)} è una definizione matematica di una struttura dati, che specifica un insieme di dati e un insieme di operazioni che possono essere eseguite su quei dati. È "astratto" perché si concentra sul "cosa" la struttura dati fa (il suo comportamento) piuttosto che sul "come" lo fa (la sua implementazione interna). L'ADT separa l'interfaccia (pubblica) dall'implementazione (privata).

\begin{itemize}
    \item \textbf{Caratteristiche}:
    \begin{itemize}
        \item \textbf{Astrazione dei Dati}: Nasconde i dettagli di rappresentazione interna dei dati.
        \item \textbf{Astrazione Funzionale}: Nasconde i dettagli di implementazione delle operazioni.
        \item \textbf{Insieme di Operazioni}: Definisce un insieme ben preciso di funzioni o metodi che possono essere applicati ai dati.
    \end{itemize}
    \item \textbf{Esempi Comuni di ADT}:
    \begin{itemize}
        \item \textbf{ADT List (Lista)}: Un insieme ordinato di elementi. Operazioni tipiche: inserimento (add), rimozione (remove), accesso a un elemento per indice (get), dimensione (size), verifica se vuota (isEmpty). L'implementazione può essere con array, liste concatenate, ecc.
        \item \textbf{ADT Stack (Pila)}: Una collezione di elementi che segue il principio LIFO (Last-In, First-Out). Operazioni tipiche: push (aggiungere un elemento in cima), pop (rimuovere l'elemento in cima), peek (vedere l'elemento in cima senza rimuoverlo), isEmpty.
        \item \textbf{ADT Queue (Coda)}: Una collezione di elementi che segue il principio FIFO (First-In, First-Out). Operazioni tipiche: enqueue (aggiungere un elemento in coda), dequeue (rimuovere l'elemento in testa), peek, isEmpty.
    \end{itemize}
\end{itemize}

\subsection{Passaggio di Parametri nelle Chiamate di Funzione}
Quando si chiama una funzione (o routine, o metodo), i valori o i riferimenti alle variabili vengono passati come argomenti. Esistono due meccanismi principali per il passaggio dei parametri:

\subsubsection{Passaggio per Valore (Call by Value)}
\begin{itemize}
    \item \textbf{Descrizione}: Viene passata una \textbf{copia del valore} dell'argomento alla funzione. La funzione opera su questa copia locale.
    \item \textbf{Effetto}: Qualsiasi modifica apportata alla copia del parametro all'interno della funzione \textbf{non influisce} sulla variabile originale passata dal chiamante.
    \item \textbf{Quando usato}: Per tipi di dati primitivi (interi, booleani, caratteri) nella maggior parte dei linguaggi, e per oggetti complessi quando si desidera che la funzione non alteri l'originale.
\end{itemize}

\subsubsection{Passaggio per Riferimento (Call by Reference / Call by Address)}
\begin{itemize}
    \item \textbf{Descrizione}: Viene passato l'\textbf{indirizzo di memoria} della variabile originale alla funzione. La funzione accede e opera direttamente sulla variabile originale tramite questo indirizzo.
    \item \textbf{Effetto}: Qualsiasi modifica apportata al parametro all'interno della funzione \textbf{influenza direttamente} la variabile originale passata dal chiamante.
    \item \textbf{Quando usato}: Spesso per oggetti complessi (in linguaggi come Java, Python, C\# gli oggetti sono tipicamente passati per riferimento implicito, anche se tecnicamente è un "passaggio per valore del riferimento"), o esplicitamente in linguaggi come C++ (usando `\&` o puntatori).
\item \textbf{Differenze Chiave}: La differenza fondamentale è se la funzione lavora su una copia (per valore) o direttamente sull'originale (per riferimento). Il passaggio per riferimento è più efficiente per oggetti grandi in quanto evita la copia, ma richiede maggiore attenzione per evitare effetti collaterali indesiderati.
\end{itemize}

\subsection{Overloading di Funzioni (o Metodi)}
L'\textbf{Overloading di funzioni} (o metodi, nel contesto OOP) è la capacità di definire più funzioni o metodi con lo \textbf{stesso nome} all'interno dello stesso scope (solitamente la stessa classe), ma che si distinguono per avere \textbf{liste di parametri diverse}. La lista dei parametri può differire per:
\begin{itemize}
    \item Il \textbf{numero} di parametri.
    \item Il \textbf{tipo} dei parametri.
    \item L'\textbf{ordine} dei parametri.
\end{itemize}
Il compilatore (o l'interprete) determina quale versione del metodo chiamare basandosi sul numero e tipo degli argomenti forniti durante la chiamata.
\textbf{Esempio}:
\begin{lstlisting}[language=Pseudocode, caption={Esempio di Function Overloading}]
FUNCTION Add(a: Integer, b: Integer):
    RETURN a + b
END FUNCTION

FUNCTION Add(a: Double, b: Double):
    RETURN a + b
END FUNCTION

FUNCTION Add(a: Integer, b: Integer, c: Integer):
    RETURN a + b + c
END FUNCTION
\end{lstlisting}

\section{Linguaggi Compilati e Linguaggi Interpretati}
La distinzione tra linguaggi compilati e interpretati riguarda il modo in cui il codice sorgente viene eseguito dal computer.

\subsection{Linguaggi Compilati}
\begin{itemize}
    \item \textbf{Descrizione}: Il codice sorgente viene tradotto (compilato) in codice macchina (o bytecode per la JVM) una sola volta da un programma chiamato "compilatore" prima dell'esecuzione. Il file eseguibile risultante può essere eseguito direttamente dal sistema operativo.
    \item \textbf{Processo}: Codice Sorgente $\rightarrow$ Compilatore $\rightarrow$ Codice Macchina Eseguibile.
    \item \textbf{Vantaggi}:
    \begin{itemize}
        \item \textbf{Prestazioni Elevate}: Il codice macchina è ottimizzato per l'hardware specifico, risultando in esecuzioni molto veloci.
        \item \textbf{Esecuzione Indipendente}: Una volta compilato, l'eseguibile non richiede il compilatore per essere eseguito.
        \item \textbf{Rilevamento Errori Precoce}: La maggior parte degli errori di sintassi e alcuni errori logici vengono rilevati in fase di compilazione.
    \end{itemize}
    \item \textbf{Svantaggi}:
    \begin{itemize}
        \item \textbf{Tempo di Compilazione}: Richiede un passaggio aggiuntivo di compilazione prima dell'esecuzione.
        \item \textbf{Dipendenza dalla Piattaforma}: L'eseguibile compilato è specifico per una particolare architettura hardware e sistema operativo (a meno di VM come Java).
    \end{itemize}
    \item \textbf{Esempi}: C, C++, Java (compila in bytecode che viene interpretato/JIT compilato dalla JVM), Go, Rust.
\end{itemize}

\subsection{Linguaggi Interpretati}
\begin{itemize}
    \item \textbf{Descrizione}: Il codice sorgente viene eseguito riga per riga da un programma chiamato "interprete" al momento dell'esecuzione, senza una fase di compilazione preventiva in codice macchina.
    \item \textbf{Processo}: Codice Sorgente $\rightarrow$ Interprete $\rightarrow$ Esecuzione.
    \item \textbf{Vantaggi}:
    \begin{itemize}
        \item \textbf{Flessibilità e Rapidità di Sviluppo}: Non c'è un passo di compilazione, quindi le modifiche possono essere testate immediatamente.
        \item \textbf{Indipendenza dalla Piattaforma}: Lo stesso codice sorgente può essere eseguito su qualsiasi piattaforma che abbia un interprete installato.
    \end{itemize}
    \item \textbf{Svantaggi}:
    \begin{itemize}
        \item \textbf{Prestazioni Generalmente Inferiori}: L'interprete analizza e traduce il codice in tempo reale, il che è più lento dell'esecuzione di codice macchina nativo.
        \item \textbf{Rilevamento Errori Tardo}: Gli errori (anche di sintassi) vengono spesso rilevati solo a runtime, quando l'interprete tenta di eseguire la riga problematica.
    \end{itemize}
    \item \textbf{Esempi}: Python, JavaScript, Ruby, PHP.
\end{itemize}

\section{Algoritmi e Complessità Computazionale}
L'\textbf{analisi della complessità computazionale} è lo studio delle risorse richieste da un algoritmo per risolvere un problema. Le risorse principali sono il tempo di esecuzione e lo spazio di memoria.

\subsection{Complessità Temporale}
Misura il tempo che un algoritmo impiega per completare la sua esecuzione, in funzione della dimensione dell'input. Viene espressa utilizzando la \textbf{notazione O-grande (Big O notation)} ($O(n)$), che descrive il tasso di crescita superiore del tempo di esecuzione dell'algoritmo al crescere della dimensione dell'input.
\begin{itemize}
    \item \textbf{O(1) - Tempo Costante}: Il tempo di esecuzione non dipende dalla dimensione dell'input.
    \item \textbf{O(log n) - Tempo Logaritmico}: Il tempo di esecuzione cresce logaritmicamente con la dimensione dell'input (es. ricerca binaria).
    \item \textbf{O(n) - Tempo Lineare}: Il tempo di esecuzione cresce linearmente con la dimensione dell'input (es. scansione di un array).
    \item \textbf{O(n log n) - Tempo Lineare-Logaritmico}: Tempo di esecuzione per algoritmi di ordinamento efficienti (es. Merge Sort, Quick Sort).
    \item \textbf{O($n^2$) - Tempo Quadratico}: Il tempo di esecuzione è proporzionale al quadrato della dimensione dell'input (es. algoritmi di ordinamento semplici come Bubble Sort, nested loops).
    \item \textbf{O($2^n$) - Tempo Esponenziale}: Il tempo di esecuzione cresce esponenzialmente con la dimensione dell'input (es. problemi NP-completi non ottimizzati, ricerca esaustiva).
\end{itemize}

\subsection{Complessità Spaziale}
Misura la quantità di memoria ausiliaria (oltre all'input stesso) che un algoritmo richiede per completare la sua esecuzione, in funzione della dimensione dell'input. Anche questa è espressa con la notazione O-grande.

\begin{itemize}
    \item \textbf{O(1) - Spazio Costante}: L'algoritmo utilizza una quantità fissa di memoria, indipendentemente dalla dimensione dell'input.
    \item \textbf{O(n) - Spazio Lineare}: La memoria richiesta cresce linearmente con la dimensione dell'input (es. memorizzare una copia dell'input).
    \item \textbf{O(n²) - Spazio Quadratico}: La memoria richiesta cresce quadraticamente con la dimensione dell'input (es. memorizzare una matrice $N \times N$).
\end{itemize}
% \input{capitoli/programmazione_oo/domande_ed_esercizi_poo.tex} % Abilitare quando avrai le domande/esercizi per POO

% Capitolo 5: Progettazione del Software
\chapter{Progettazione del Software}

La \textbf{Progettazione del Software} è il processo di definizione dell'architettura, dei componenti, delle interfacce e di altri attributi di un sistema o di un componente. È una fase cruciale nel ciclo di vita dello sviluppo del software, che traduce i requisiti in un piano dettagliato per la costruzione del sistema.

\section{Analisi dei Requisiti}
L'\textbf{analisi dei requisiti} è il processo di definizione, documentazione e mantenimento dei requisiti software. È la fase iniziale di qualsiasi progetto software e mira a comprendere le esigenze degli stakeholder per il sistema da costruire.

\subsection{Tipi di Requisiti}
I requisiti possono essere classificati in due categorie principali:
\begin{itemize}
    \item \textbf{Requisiti Funzionali}: Descrivono ciò che il sistema \textit{deve fare}. Definiscono le funzioni, i servizi e i comportamenti specifici che il sistema deve fornire agli utenti.
    \begin{itemize}
        \item \textbf{Esempi}: "Il sistema deve consentire la registrazione di nuovi utenti.", "Il sistema deve calcolare la somma delle vendite giornaliere.", "Il sistema deve permettere la visualizzazione del calendario delle partite."
    \end{itemize}
    \item \textbf{Requisiti Non Funzionali (o di Qualità)}: Descrivono come il sistema \textit{deve funzionare}. Definiscono le caratteristiche di qualità, i vincoli e gli attributi del sistema, piuttosto che le sue funzionalità specifiche. Spesso influenzano l'architettura e l'implementazione.
    \begin{itemize}
        \item \textbf{Esempi}:
        \begin{itemize}
            \item \textbf{Prestazioni}: "Il sistema deve rispondere a una query entro 2 secondi."
            \item \textbf{Scalabilità}: "Il sistema deve supportare 1000 utenti concorrenti."
            \item \textbf{Sicurezza}: "Il sistema deve autenticare gli utenti tramite username e password con criptazione."
            \item \textbf{Usabilità}: "L'interfaccia utente deve essere intuitiva e di facile apprendimento."
            \item \textbf{Affidabilità}: "Il sistema deve essere disponibile il 99.9\% del tempo."
            \item \textbf{Manutenibilità}: "Il codice deve essere modulare e ben documentato."
            \item \textbf{Portabilità}: "Il sistema deve funzionare su Windows e Linux."
        \end{itemize}
    \end{itemize}
\end{itemize}

\subsection{Diagrammi di Casi d'Uso UML (Use Case Diagrams)}
I \textbf{Diagrammi di Casi d'Uso} sono uno strumento UML (Unified Modeling Language) utilizzato nell'analisi dei requisiti funzionali. Descrivono le interazioni tra gli utenti (attori) e il sistema, rappresentando le diverse funzionalità che il sistema offre dal punto di vista dell'utente.
\begin{itemize}
    \item \textbf{Componenti Principali}:
    \begin{itemize}
        \item \textbf{Attore (Actor)}: Rappresenta un ruolo esterno che interagisce con il sistema (persona, altro sistema, dispositivo). Disegnato come un omino stilizzato.
        \item \textbf{Caso d'Uso (Use Case)}: Rappresenta una funzionalità specifica del sistema, un servizio che il sistema fornisce all'attore. Disegnato come un ovale.
        \item \textbf{Confine del Sistema (System Boundary)}: Un rettangolo che racchiude i casi d'uso, distinguendo ciò che è all'interno del sistema da ciò che è esterno.
        \item \textbf{Relazioni}:
        \begin{itemize}
            \item \textbf{Associazione}: L'interazione tra un attore e un caso d'uso (linea semplice).
            \item \textbf{Include}: Un caso d'uso include la funzionalità di un altro caso d'uso (freccia tratteggiata da caso d'uso includente a caso d'uso incluso, con etichetta `<<include>>`).
            \item \textbf{Extend}: Un caso d'uso estende il comportamento di un altro caso d'uso in circostanze specifiche (freccia tratteggiata da caso d'uso estendente a caso d'uso esteso, con etichetta `<<extend>>`).
            \item \textbf{Generalizzazione}: Una relazione di ereditarietà tra attori o casi d'uso (freccia con punta triangolare).
        \end{itemize}
    \end{itemize}
    \item \textbf{Scopo}: Fornire una visione ad alto livello dei requisiti funzionali, facilitare la comunicazione tra stakeholder e sviluppatori, e servire da base per la progettazione successiva.
\end{itemize}

\section{Progettazione dell'Architettura del Sistema Software}
La \textbf{progettazione dell'architettura del software} definisce la struttura di alto livello di un sistema software, delineando come i suoi componenti interagiscono e sono organizzati. Include la scelta di pattern architetturali e di design.

\subsection{Design Patterns (Pattern Architetturali e di Progettazione)}
I \textbf{Design Patterns} sono soluzioni riutilizzabili a problemi comuni che si presentano nella progettazione del software. Non sono soluzioni pronte all'uso, ma modelli da adattare al contesto specifico.

\begin{itemize}
    \item \textbf{Vantaggi}:
    \begin{itemize}
        \item \textbf{Riutilizzo di Soluzioni Comprovate}: Utilizzano approcci che hanno dimostrato di funzionare.
        \item \textbf{Vocabolario Condiviso}: Facilitano la comunicazione tra gli sviluppatori.
        \item \textbf{Miglioramento della Qualità del Codice}: Portano a codice più manutenibile, scalabile e flessibile.
    \end{itemize}
    \item \textbf{Esempi Comuni}:
    \begin{itemize}
        \item \textbf{MVC (Model-View-Controller)}: Un pattern architetturale che separa l'applicazione in tre componenti interconnessi per gestire meglio l'interfaccia utente.
        \begin{itemize}
            \item \textbf{Model}: Gestisce i dati e la logica di business.
            \item \textbf{View}: Si occupa della presentazione dei dati all'utente.
            \item \textbf{Controller}: Gestisce l'input dell'utente e coordina Model e View.
        \end{itemize}
        \item \textbf{Singleton}: Garantisce che una classe abbia una sola istanza e fornisce un punto di accesso globale a essa. Utile per risorse uniche come gestori di configurazione o log.
        \item \textbf{Factory Method}: Definisce un'interfaccia per creare un oggetto, ma lascia alle sottoclassi la decisione di quale classe istanziare. Permette di creare oggetti senza specificare la classe esatta che verrà creata.
        \item \textbf{Observer}: Definisce una dipendenza uno-a-molti tra oggetti in modo che quando un oggetto cambia stato, tutti i suoi dipendenti vengono notificati e aggiornati automaticamente. Utile per eventi e notifiche.
        \item \textbf{Strategy}: Definisce una famiglia di algoritmi, incapsula ciascuno di essi e li rende intercambiabili. Permette all'algoritmo di variare indipendentemente dai client che lo usano.
    \end{itemize}
\end{itemize}

\subsection{Diagrammi UML (Unified Modeling Language)}
Il \textbf{Unified Modeling Language (UML)} è un linguaggio di modellazione standardizzato, ampiamente utilizzato nell'ingegneria del software per specificare, visualizzare, costruire e documentare gli artefatti di un sistema software. Offre una notazione grafica per rappresentare vari aspetti di un sistema, dalla sua struttura statica al suo comportamento dinamico.

\subsubsection{Diagrammi dei Casi d'Uso (Use Case Diagrams)}
I \textbf{Diagrammi dei Casi d'Uso} sono diagrammi comportamentali UML utilizzati nella fase di analisi dei requisiti. Descrivono le interazioni tra gli utenti (attori) e il sistema, rappresentando le diverse funzionalità che il sistema offre dal punto di vista dell'utente. Si concentrano sul "cosa" il sistema fa per i suoi attori, piuttosto che sul "come" lo fa.
\begin{itemize}
    \item \textbf{Scopo}: Identificare e documentare i requisiti funzionali del sistema. Forniscono una visione ad alto livello delle funzionalità, facilitano la comunicazione tra stakeholder e sviluppatori.
    \item \textbf{Componenti Principali}:
    \begin{itemize}
        \item \textbf{Attore (Actor)}: Un ruolo esterno (persona, altro sistema, dispositivo hardware) che interagisce con il sistema. Rappresentato da un omino stilizzato.
        \item \textbf{Caso d'Uso (Use Case)}: Una funzionalità specifica del sistema, un servizio che il sistema fornisce all'attore. Rappresentato da un ovale.
        \item \textbf{Confine del Sistema (System Boundary)}: Un rettangolo che racchiude i casi d'uso, distinguendo ciò che è all'interno del sistema da ciò che è esterno.
        \item \textbf{Relazioni}:
        \begin{itemize}
            \item \textbf{Associazione}: L'interazione tra un attore e un caso d'uso (linea semplice).
            \item \textbf{Include (`<<include>>`)}: Indica che un caso d'uso include la funzionalità di un altro caso d'uso. La freccia è tratteggiata, va dal caso d'uso includente a quello incluso. Usato per riutilizzare comportamenti comuni.
            \item \textbf{Extend (`<<extend>>`)}: Indica che un caso d'uso estende il comportamento di un altro caso d'uso in circostanze specifiche. La freccia è tratteggiata, va dal caso d'uso estendente a quello esteso. Usato per funzionalità opzionali o eccezionali.
            \item \textbf{Generalizzazione}: Una relazione di ereditarietà tra attori o casi d'uso (freccia con punta triangolare non piena).
        \end{itemize}
    \end{itemize}
\end{itemize}

\paragraph{Esempio: Relazioni tra Casi d'Uso}
La comprensione delle relazioni `Include`, `Extend` e `Generalizzazione` è cruciale per modellare accuratamente il comportamento del sistema e i requisiti opzionali o riutilizzabili.
\begin{figure}[h!]
    \centering
    \includegraphics[width=0.8\textwidth]{immagini/uml_use_case_relazioni.png} % Immagine da Pagina 2 del tuo UML.pdf
    \caption{Rappresentazione grafica delle relazioni comuni tra Casi d'Uso in UML (Inclusione, Estensione e Generalizzazione).}
    \label{fig:uml_use_case_relazioni}
\end{figure}

\paragraph{Esempio Completo: Sistema di Ristorazione}
Un esempio pratico aiuta a consolidare la comprensione di come i casi d'uso e le loro relazioni si applichino in un contesto reale. Il diagramma seguente mostra le funzionalità di un sistema di gestione per un ristorante, illustrando le interazioni tra clienti, personale e il sistema stesso.
\begin{figure}[h!]
    \centering
    \includegraphics[width=0.9\textwidth]{immagini/diagramma_casi_uso_ristorante.jpg} % Immagine da Pagina 1 del tuo UML.pdf
    \caption{Esempio completo di Diagramma dei Casi d'Uso per un sistema di Ristorazione, che illustra le interazioni tra attori (Cameriere, Cliente, Cuoco, Cassiere) e le funzionalità del sistema.}
    \label{fig:diagramma_casi_uso_ristorante_completo}
\end{figure}

\subsubsection{Diagrammi delle Classi (Class Diagrams)}
I \textbf{Diagrammi delle Classi} sono diagrammi strutturali UML che mostrano la struttura statica di un sistema, le classi, i loro attributi (dati), i loro metodi (operazioni) e le relazioni tra le classi. Sono fondamentali per la progettazione orientata agli oggetti e per modellare il design logico del database.

\begin{itemize}
    \item \textbf{Scopo}: Modellare il design logico del sistema, la struttura del codice, le relazioni tra le classi e le dipendenze. Utile per visualizzare l'architettura del software a livello di classi.
    \item \textbf{Componenti Principali}:
    \begin{itemize}
        \item \textbf{Classe}: Rappresentata da un rettangolo diviso in tre sezioni: nome della classe, attributi (con visibilità e tipo), e metodi (con visibilità, parametri e tipo di ritorno).
        \item \textbf{Visibilità (Visibility)}: Indica l'accessibilità degli attributi e metodi, rappresentata da simboli specifici (\lstinline{+}, \lstinline{-}, \lstinline{\#}, \lstinline{\textasciitilde}).
        \item \textbf{Relazioni}: Specificano le associazioni tra le classi.
    \end{itemize}
\end{itemize}

\paragraph{Struttura Base e Attributi}
Un diagramma delle classi inizia con la rappresentazione della singola classe, evidenziando il suo nome, i suoi attributi (le proprietà) e le sue operazioni (i metodi).
\begin{figure}[h!]
    \centering
    \subcaptionbox{Struttura di una Classe (Automobile)\label{fig:class_base_automobile}}{%
        \includegraphics[width=0.45\textwidth]{immagini/uml_class_base_automobile.png}%
    }\quad
    \subcaptionbox{Classe con Metodi e Attributi (Scheda Anagrafica)\label{fig:class_scheda_anagrafica}}{%
        \includegraphics[width=0.45\textwidth]{immagini/uml_class_scheda_anagrafica_struttura.png}%
    }
    \caption{Esempi di rappresentazione base di una classe UML, con attributi e metodi, e come si relaziona al codice.}
    \label{fig:struttura_base_classi}
\end{figure}

L'uso di modificatori di visibilità definisce l'accessibilità di questi attributi e metodi, come mostrato nell'esempio dei Clienti e Fornitori.
\begin{figure}[h!]
    \centering
    \includegraphics[width=0.7\textwidth]{immagini/uml_class_modificatori_visibilita.png}
    \caption{Esempio di Diagramma delle Classi che illustra l'uso dei modificatori di visibilità (public, private, protected) per attributi e metodi.}
    \label{fig:class_modificatori}
\end{figure}

\paragraph{Relazioni di Generalizzazione (Ereditarietà)}
La generalizzazione indica che una classe (sottoclasse) eredita proprietà e comportamenti da un'altra classe (superclasse). Questo promuove il riutilizzo del codice.
\begin{figure}[h!]
    \centering
    \includegraphics[width=0.7\textwidth]{immagini/uml_class_ereditarieta_generica.png}
    \caption{Esempio di Diagramma delle Classi che mostra la relazione di ereditarietà tra le classi Persona, Cliente e Potenziale Cliente.}
    \label{fig:class_ereditarieta}
\end{figure}
In contesti come Java, dove l'ereditarietà multipla non è ammessa (a differenza di C++), le interfacce vengono utilizzate per ovviare a questa limitazione, permettendo a una classe di implementare più interfacce.
\begin{figure}[h!]
    \centering
    \includegraphics[width=0.7\textwidth]{immagini/uml_class_ereditarieta_multipla_interfacce.png}
    \caption{Esempio di Diagramma delle Classi che illustra come le interfacce consentono di simulare l'ereditarietà multipla in linguaggi che non la supportano nativamente.}
    \label{fig:class_ereditarieta_multipla}
\end{figure}

\paragraph{Relazioni Strutturali: Aggregazione e Composizione}
Aggregazione e composizione sono forme specifiche di associazione che esprimono relazioni "parte di" tra classi, distinguendosi per la dipendenza esistenziale della parte rispetto al tutto.
\begin{itemize}
    \item \textbf{Aggregazione}: Rappresenta una relazione uno a molti in cui l'oggetto "parte" può esistere indipendentemente dall'oggetto "tutto" (es. un libro può esistere senza una mensola specifica). Viene rappresentata con una freccia con la punta a diamante vuota all'estremità del "tutto".
    \item \textbf{Composizione}: Una forma più forte di aggregazione, che implica una esclusività. La "parte" non può esistere da sola senza il "tutto", e la distruzione del "tutto" comporta la distruzione delle "parti" (es. una pagina non può esistere senza il suo libro). Il diamante si disegna pieno.
\end{itemize}
\begin{figure}[h!]
    \centering
    \subcaptionbox{Esempio di Aggregazione (Mensola e Libro)\label{fig:class_aggregazione_esempio}}{%
        \includegraphics[width=0.45\textwidth]{immagini/uml_class_aggregazione_esempio.png}%
    }\quad
    \subcaptionbox{Esempio di Composizione (Libro e Pagina)\label{fig:class_composizione_esempio}}{%
        \includegraphics[width=0.45\textwidth]{immagini/uml_class_composizione_esempio.png}%
    }
    \caption{Confronto visivo tra Aggregazione e Composizione nei Diagrammi delle Classi UML.}
    \label{fig:class_aggregazione_composizione}
\end{figure}

\paragraph{Esempi Complessi di Diagrammi delle Classi}
Per illustrare l'applicazione di queste relazioni in contesti più ampi, consideriamo sistemi con diverse classi interconnesse, come un sistema e-commerce o un'applicazione di chat. Questi esempi mostrano come le classi interagiscono per formare un sistema coerente.
\begin{figure}[h!]
    \centering
    \includegraphics[width=0.9\textwidth]{immagini/uml_class_sistema_e_commerce.png} % Assicurati che l'immagine sia salvata con questo nome e percorso
    \caption{Diagramma delle Classi per un sistema di gestione prodotti/catalogo/carrello, che illustra varie classi e le loro relazioni (es. Prodotto, Carrello, DVD, Libro).}
    \label{fig:class_sistema_e_commerce}
\end{figure}
\begin{figure}[h!]
    \centering
    \includegraphics[width=0.9\textwidth]{immagini/uml_class_sistema_chat_corretto.png} % Assicurati che l'immagine sia salvata con questo nome e percorso
    \caption{Diagramma delle Classi per un sistema di Chat, che mostra le interazioni tra classi come Utente, Chat e Messaggio, e le loro molteplicità.}
    \label{fig:class_sistema_chat}
\end{figure}

\subsubsection{Diagrammi di Sequenza (Sequence Diagrams)}
I \textbf{Diagrammi di Sequenza} sono diagrammi comportamentali (di interazione) UML che mostrano l'ordine cronologico dei messaggi scambiati tra oggetti o attori in un'interazione specifica. Sono usati per modellare la logica di una funzionalità o di un algoritmo, illustrando come diversi oggetti collaborano per raggiungere un obiettivo.
\begin{itemize}
    \item \textbf{Scopo}: Dettagliare il flusso di controllo e di dati per funzionalità specifiche, spesso come implementazione di casi d'uso. Utili per visualizzare l'interazione dinamica e l'ordine delle chiamate.
    \item \textbf{Componenti Principali}:
    \begin{itemize}
        \item \textbf{Lifeline (Linea di Vita)}: Rappresenta la partecipazione di un oggetto o attore all'interazione, mostrata come una linea verticale tratteggiata. In cima alla lifeline c'è un rettangolo (per gli oggetti) o un omino (per gli attori).
        \item \textbf{Attore}: Utente o sistema esterno che avvia o partecipa all'interazione.
        \item \textbf{Messaggio}: Una comunicazione o chiamata di metodo tra lifeline, rappresentata da una freccia orizzontale. Possono essere:
        \begin{itemize}
            \item \textbf{Sincroni}: Freccia piena, indica una chiamata bloccante in attesa di risposta.
            \item \textbf{Asincroni}: Freccia con punta aperta, indica una chiamata non bloccante.
            \item \textbf{Risposta}: Freccia tratteggiata, indica il ritorno di un valore o la fine di una chiamata sincrona.
        \end{itemize}
        \item \textbf{Barra di Attivazione (Activation Bar/Execution Occurrence)}: Un rettangolo sottile posizionato verticalmente sulla lifeline, che indica il periodo di tempo durante il quale un oggetto è attivo e sta eseguendo un'operazione o aspettando una risposta.
        \item \textbf{Frammenti Combinati (Combined Fragments)}: Costrutti per mostrare strutture di controllo logico:
        \begin{itemize}
            \item \textbf{alt (alternative)}: Per mostrare blocchi if-else.
            \item \textbf{opt (optional)}: Per mostrare un blocco if.
            \item \textbf{loop (loop)}: Per mostrare iterazioni.
            \item \textbf{par (parallel)}: Per mostrare esecuzioni parallele.
        \end{itemize}
    \end{itemize}
\end{itemize}
\begin{figure}[h!]
    \centering
    % Placeholder per l'immagine del Diagramma di Sequenza UML
    % \includegraphics[width=0.8\textwidth]{immagini/diagramma_sequenza_esempio.png}
    \caption{Esempio di Diagramma di Sequenza UML che mostra l'interazione temporale degli oggetti per una funzionalità.}
    \label{fig:diagramma_sequenza_uml_specifico}
\end{figure}

\subsubsection{Diagrammi di Deployment (Deployment Diagrams)}
I \textbf{Diagrammi di Deployment} sono diagrammi strutturali UML che mostrano la configurazione fisica dei nodi hardware (computer, server, dispositivi) e come i componenti software sono distribuiti su questi nodi.
\begin{itemize}
    \item \textbf{Scopo}: Visualizzare la topologia fisica del sistema, la distribuzione dei componenti software sull'hardware e le interazioni fisiche tra i nodi. Utile per architetture distribuite e pianificazione dell'infrastruttura.
    \item \textbf{Componenti Principali}:
    \begin{itemize}
        \item \textbf{Nodo (Node)}: Una risorsa computazionale fisica o logica (hardware come un server, un PC, un dispositivo mobile; o un ambiente di esecuzione software come un Docker container, una JVM). Rappresentato da un cubo 3D.
        \item \textbf{Artefatto (Artifact)}: Un prodotto fisico risultante dal processo di sviluppo del software (file eseguibile, libreria, file JAR, file di configurazione, script). Rappresentato da un documento con l'icona di un artefatto.
        \item \textbf{Componente (Component)}: Una parte modulare e sostituibile del sistema con interfacce ben definite. Può essere distribuito su un nodo.
        \item \textbf{Comunicazione (Communication Path)}: Una linea che connette i nodi, indicando un percorso di comunicazione tra di essi (es. TCP/IP, HTTP).
    \end{itemize}
\end{itemize}
\begin{figure}[h!]
    \centering
    % Placeholder per l'immagine del Diagramma di Deployment UML
    % \includegraphics[width=0.8\textwidth]{immagini/diagramma_deployment_esempio.png}
    \caption{Esempio di Diagramma di Deployment UML che illustra la distribuzione dei componenti software sull'infrastruttura hardware.}
    \label{fig:diagramma_deployment_uml}
\end{figure}
% \input{capitoli/progettazione_software/domande_ed_esercizi_ps.tex} % Abilitare quando avrai le domande/esercizi per Progettazione Software

% Capitolo 6: Elettronica
\chapter{Elettronica}

L'\textbf{Elettronica} è la branca dell'ingegneria e della fisica che si occupa del controllo del flusso di elettroni, tipicamente attraverso dispositivi semiconduttori, per costruire circuiti e sistemi che elaborano informazioni o controllano energia. È alla base di tutti i dispositivi digitali e di molteplici sistemi analogici moderni.

\section{Transistore MOS (Metal-Oxide-Semiconductor Field-Effect Transistor)}
Il \textbf{transistore MOS} è il dispositivo fondamentale della microelettronica moderna e la spina dorsale di microprocessori, memorie e altri circuiti integrati. È un transistore a effetto di campo (FET), dove la corrente tra due terminali (Source e Drain) è controllata da un campo elettrico generato da una tensione applicata a un terzo terminale (Gate).

\subsection{Struttura del Transistore MOS (nMOS)}
Per comprendere il funzionamento, si consideri la realizzazione più comune: l'nMOS (n-channel Metal-Oxide-Semiconductor).
\begin{itemize}
    \item \textbf{Substrato (Bulk/Body)}: Generalmente di tipo P per un nMOS. È il materiale semiconduttore di base su cui viene costruito il dispositivo.
    \item \textbf{Source (S) e Drain (D)}: Due regioni altamente drogate (N+ per nMOS) impiantate nel substrato P. Sono simmetriche e collegate ai terminali esterni. Il Source è tipicamente la sorgente di portatori di carica, il Drain è dove fluiscono.
    \item \textbf{Canale}: La regione del substrato (P) tra Source e Drain. È qui che si formerà il canale di conduzione per il flusso di corrente.
    \item \textbf{Ossido di Gate ($SiO_2$)}: Uno strato isolante molto sottile (diossido di silicio) deposto sopra il canale. Agisce come dielettrico, isolando elettricamente il Gate dal canale.
    \item \textbf{Gate (G)}: Uno strato conduttivo (metallo o polisilicio) deposto sopra l'ossido di gate. È il terminale di controllo attraverso il quale si applica la tensione per creare il campo elettrico.
    \item \textbf{Terminali}: Gate (G), Source (S), Drain (D), Bulk/Substrate (B). Spesso il Bulk è collegato al Source o a una tensione fissa (es. massa per nMOS, $V_{DD}$ per pMOS).
\end{itemize}

\subsection{Principio di Funzionamento (nMOS)}
Il funzionamento del transistore MOS dipende dalla tensione applicata tra Gate e Source ($V_{GS}$) e tra Drain e Source ($V_{DS}$). Si basa sulla modulazione della conduttività del canale tramite un campo elettrico.
\begin{itemize}
    \item \textbf{Interdizione (Cut-off Region)}:
    \begin{itemize}
        \item \textbf{Condizione}: $V_{GS} < V_{TH}$ (Tensione di Soglia).
        \item \textbf{Descrizione}: Non c'è un campo elettrico sufficiente per attrarre elettroni al canale. Non si forma un canale di conduzione tra Source e Drain. La corrente $I_{DS}$ è praticamente nulla (solo una piccola corrente di leakage). Il transistore agisce come un interruttore aperto.
    \end{itemize}
    \item \textbf{Conduzione (quando $V_{GS} > V_{TH}$)}:
    \begin{itemize}
        \item Quando una tensione positiva $V_{GS}$ sufficiente (maggiore di $V_{TH}$) viene applicata al Gate, il campo elettrico attrae gli elettroni (portatori minoritari nel substrato P) verso la superficie del semiconduttore, sotto l'ossido.
        \item Si forma uno strato sottile di elettroni, creando un \textbf{canale di conduzione} a bassa resistenza tra Source e Drain.
        \item Se si applica una tensione $V_{DS} > 0$, una corrente $I_{DS}$ fluirà attraverso questo canale.
    \end{itemize}
\end{itemize}

\subsection{Regimi di Funzionamento (per nMOS con $V_{GS} > V_{TH}$)}
Una volta che il transistore è in conduzione, il suo comportamento (e la corrente $I_{DS}$) dipende dalla relazione tra $V_{GS}$ e $V_{DS}$.
\begin{itemize}
    \item \textbf{Regione di Triodo (o Lineare/Ohmica)}:
    \begin{itemize}
        \item \textbf{Condizioni}: $V_{GS} > V_{TH}$ e $V_{DS} < (V_{GS} - V_{TH})$.
        \item \textbf{Descrizione}: Il canale è completamente formato e la sua profondità è relativamente uniforme. Il transistore si comporta come una resistenza controllata in tensione, e la corrente $I_{DS}$ è approssimativamente lineare rispetto a $V_{DS}$.
    \item \textbf{Applicazione}: Usata come interruttore chiuso (ON) in circuiti digitali (con bassa resistenza) o come resistenza variabile in circuiti analogici.
    \end{itemize}
    \item \textbf{Regione di Saturazione}:
    \begin{itemize}
        \item \textbf{Condizioni}: $V_{GS} > V_{TH}$ e $V_{DS} \geq (V_{GS} - V_{TH})$.
        \item \textbf{Descrizione}: Aumentando $V_{DS}$, la tensione canale-gate verso il lato del Drain si riduce. Quando $V_{DS}$ raggiunge $V_{GS} - V_{TH}$, il canale si "pizzica" (pinch-off) vicino al Drain. Oltre questo punto, ulteriori aumenti di $V_{DS}$ non aumentano significativamente la corrente $I_{DS}$, che diventa quasi costante.
    \item \textbf{Applicazione}: Questa è la regione di funzionamento preferita per gli amplificatori (circuiti analogici) in quanto si comporta come una sorgente di corrente controllata in tensione, e per gli stati ON (logico '1') in circuiti digitali per un'elevata impedenza di uscita.
    \end{itemize}
\end{itemize}

\subsection{Vantaggi del MOS rispetto al Transistore Bipolare (BJT)}
Il transistore MOS ha soppiantato in gran parte il BJT nella microelettronica digitale e in molte applicazioni analogiche grazie a diversi vantaggi chiave:
\begin{itemize}
    \item \textbf{Alta Impedenza di Ingresso}: Il gate del MOS è isolato dall'ossido, il che lo rende quasi idealmente un circuito aperto con una corrente di gate praticamente nulla ($I_G \approx 0$). Questo riduce il carico sui circuiti precedenti e semplifica il design degli stadi di ingresso. Il BJT, invece, è controllato in corrente (corrente di base) e presenta un'impedenza di ingresso inferiore.
    \item \textbf{Minore Dissipazione di Potenza Statica (in CMOS)}: Nelle configurazioni Complementary MOS (CMOS), quando il circuito è in stato stabile (non commuta), uno dei transistori è sempre spento, eliminando un percorso di corrente diretto tra alimentazione e massa. Questo porta a una dissipazione di potenza statica estremamente bassa, fondamentale per dispositivi a batteria e circuiti ad alta integrazione. I BJT, anche quando "spenti", possono avere correnti di leakage maggiori e configurazioni logiche basate su BJT tendono a dissipare più potenza.
    \item \textbf{Scalabilità e Densità di Integrazione}: I MOS possono essere miniaturizzati molto più facilmente rispetto ai BJT, permettendo la realizzazione di circuiti integrati con miliardi di transistori su un singolo chip. La loro struttura planare si presta bene alla fabbricazione su larga scala.
    \item \textbf{Costo di Fabbricazione}: Il processo di fabbricazione MOS è generalmente più semplice e meno costoso rispetto a quello BJT, che richiede più passaggi di drogaggio e diffusione.
    \item \textbf{Immunità al Rumore (in CMOS)}: I circuiti CMOS hanno buoni margini di rumore, rendendoli robusti contro le fluttuazioni di tensione indesiderate.
\end{itemize}
Nonostante ciò, i BJT mantengono vantaggi in alcune applicazioni specifiche (es. alta frequenza, alta potenza) grazie a una maggiore transconduttanza e velocità in particolari contesti.

\section{Circuiti CMOS (Complementary MOS)}
La tecnologia \textbf{CMOS} è il design più diffuso per la realizzazione di circuiti integrati digitali, caratterizzata dall'uso complementare di transistori nMOS e pMOS.

\subsection{Invertitore CMOS}
L'\textbf{invertitore CMOS} è la porta logica fondamentale in tecnologia CMOS, che realizza la funzione NOT (negazione logica). Produce un'uscita opposta all'ingresso.
\begin{itemize}
    \item \textbf{Realizzazione}: È composto da due transistori MOS collegati in serie tra $V_{DD}$ e GND, con i gate collegati insieme a formare l'ingresso (IN) e i drain collegati a formare l'uscita (OUT).
    \begin{itemize}
        \item Un \textbf{pMOS} (pull-up network) è collegato tra $V_{DD}$ e OUT. Il pMOS conduce quando il suo gate è basso.
        \item Un \textbf{nMOS} (pull-down network) è collegato tra OUT e GND. L'nMOS conduce quando il suo gate è alto.
    \end{itemize}
    \item \textbf{Funzionamento}:
    \begin{itemize}
        \item \textbf{Ingresso Alto (Logico '1', $V_{IN} = V_{DD}$)}: L'nMOS è ON (conduce), creando un percorso a bassa resistenza tra OUT e GND. Il pMOS è OFF (interdetto). L'uscita $V_{OUT}$ è tirata a GND (Logico '0').
        \item \textbf{Ingresso Basso (Logico '0', $V_{IN} = GND$)}: Il pMOS è ON (conduce), creando un percorso a bassa resistenza tra OUT e $V_{DD}$. L'nMOS è OFF (interdetto). L'uscita $V_{OUT}$ è tirata a $V_{DD}$ (Logico '1').
    \end{itemize}
\end{itemize}

\subsection{Vantaggi dell'Invertitore CMOS rispetto agli Invertitori con Carico Resistivo}
Gli invertitori CMOS offrono vantaggi significativi rispetto alle implementazioni più vecchie che usavano un transistore (BJT o MOS) con una resistenza di carico:
\begin{itemize}
    \item \textbf{Minore Dissipazione di Potenza Statica (Vantaggio Primario)}: Nello stato stabile (ingresso alto o basso), uno dei due transistori (nMOS o pMOS) è sempre spento. Non c'è un percorso di corrente diretto da $V_{DD}$ a GND. La corrente statica è limitata a correnti di leakage quasi nulle, risultando in un consumo di potenza estremamente basso a riposo. Gli invertitori con carico resistivo, invece, dissipano continuamente potenza quando l'uscita è nello stato "basso", poiché la corrente fluisce attraverso il resistore di carico e il transistore acceso.
    \item \textbf{Migliori Margini di Rumore}: Le curve di trasferimento tensione-tensione degli invertitori CMOS sono quasi ideali, con una transizione molto ripida tra i due stati logici. Questo fornisce ampi margini di rumore, rendendo i circuiti più robusti alle fluttuazioni di tensione indesiderate.
    \item \textbf{Prestazioni Simmetriche (Potenziale)}: Con un corretto dimensionamento dei transistori (rapporto $W/L$), i tempi di salita (charge time) e di discesa (discharge time) dell'uscita possono essere resi quasi simmetrici. Negli invertitori con resistore, il tempo di salita (carica del condensatore di carico attraverso il resistore) è tipicamente più lento del tempo di discesa.
    \item \textbf{Densità di Integrazione}: I transistori MOS occupano meno area rispetto ai resistori su chip, consentendo una maggiore densità di circuiti.
\item \textbf{Ampio Range di Tensione Operativa}: I circuiti CMOS possono operare su un'ampia gamma di tensioni di alimentazione mantenendo buone prestazioni.
\end{itemize}

\subsection{Applicazioni Notevoli dell'Invertitore CMOS}
L'invertitore CMOS è un blocco costruttivo fondamentale per una vasta gamma di applicazioni:
\begin{itemize}
    \item \textbf{Porta Logica Fondamentale}: È il componente di base da cui vengono costruite tutte le altre porte logiche digitali (NAND, NOR, XOR, latch, flip-flop) in tecnologia CMOS.
    \item \textbf{Buffer e Driver}: Utilizzando cascate di invertitori, si possono creare buffer (per aumentare la capacità di pilotaggio di un segnale) o driver per linee lunghe o carichi capacitivi elevati (es. clock network nei microprocessori).
    \item \textbf{Celle di Memoria}: Due invertitori CMOS collegati in back-to-back formano un latch, che è la cella di memoria fondamentale nelle SRAM (Static RAM), capace di memorizzare un bit.
    \item \textbf{Oscillatori ad Anello (Ring Oscillators)}: Una catena di un numero dispari di invertitori, con l'uscita dell'ultimo collegata all'ingresso del primo, crea un oscillatore che produce un'onda quadra. Usati per testare le prestazioni di velocità e per generare segnali di clock.
    \item \textbf{Livelli di Tensione}: Possono essere usati per convertire livelli di tensione logici tra diversi standard o blocchi di circuiti.
\end{itemize}

% \input{capitoli/elettronica/domande_ed_esercizi_elettronica.tex} % Abilitare quando avrai le domande/esercizi per Elettronica

% Capitolo 7: Sistemi Numerici
\chapter{Sistemi Numerici}

I \textbf{sistemi numerici} sono metodi per rappresentare i numeri utilizzando simboli specifici e regole ben definite. In informatica, la rappresentazione binaria è fondamentale, ma è cruciale anche capire come i numeri negativi vengono gestiti, in particolare tramite il complemento a due.

\section{Rappresentazione per Numeri Interi}
I numeri interi possono essere rappresentati in diversi modi all'interno di un sistema digitale. Le rappresentazioni più comuni per i numeri con segno includono segno e modulo, complemento a uno, e complemento a due. Il complemento a due è la rappresentazione più utilizzata nei sistemi digitali moderni per la sua efficienza nelle operazioni aritmetiche.

\subsection{Complemento a Due}
La rappresentazione in \textbf{complemento a due} è il metodo più diffuso per rappresentare numeri interi con segno nei sistemi digitali. Offre il vantaggio di semplificare le operazioni di somma e sottrazione, poiché la sottrazione può essere implementata come una somma con il complemento a due del sottraendo, eliminando la necessità di circuiti dedicati alla sottrazione.

\subsubsection{Principio di Funzionamento}
\begin{itemize}
    \item Un numero positivo in complemento a due è rappresentato esattamente come nella notazione binaria pura (senza segno), con il bit più significativo (MSB, most significant byte) pari a $0$.
    \item Un numero negativo in complemento a due è ottenuto complementando (invertendo) tutti i bit del suo valore assoluto (passando da $0$ a $1$ e viceversa) e poi sommando $1$ al risultato. Il bit più significativo (MSB) sarà sempre $1$ per i numeri negativi.
    \item Il bit più significativo (MSB) indica il segno: $0$ per i positivi, $1$ per i negativi.
    \item Il range di valori rappresentabile con $N$ bit in complemento a due va da $-2^{N-1}$ a $2^{N-1}-1$. Ad esempio, con $8$ bit, si possono rappresentare numeri da $-128$ a $127$.
\end{itemize}

\subsubsection{Derivazione della Rappresentazione in Complemento a Due}
Per convertire un numero decimale negativo in complemento a due (con $N$ bit):
\begin{enumerate}
    \item Prendi il valore assoluto del numero decimale (positivo).
    \item Converti il valore assoluto in binario su $N$ bit.
    \item Inverti tutti i bit (complemento a uno, $0$ diventa $1$, $1$ diventa $0$).
    \item Somma $1$ al risultato binario.
\end{enumerate}
Per convertire un numero binario in complemento a due a decimale:
\begin{itemize}
    \item Se il MSB è $0$: il numero è positivo. Convertilo come un normale binario senza segno.
    \item Se il MSB è $1$: il numero è negativo.
    \begin{enumerate}
        \item Inverti tutti i bit del numero binario.
        \item Somma $1$ al risultato binario.
        \item Converti questo risultato binario in decimale e anteponi il segno meno.
    \end{enumerate}
\end{itemize}

\subsubsection{Esempio: Rappresentazione in Binario a 16 bit del numero decimale -15}
Per rappresentare il numero decimale -15 in complemento a due su 16 bit:
\begin{enumerate}
    \item \textbf{Valore assoluto}: $|-15| = 15$.
    \item \textbf{15 in binario su 16 bit}: $0000\,0000\,0000\,1111_2$.
    \item \textbf{Inverti tutti i bit (complemento a uno)}: $1111\,1111\,1111\,0000_2$.
    \item \textbf{Somma 1}: $1111\,1111\,1111\,0000_2 + 1_2 = 1111\,1111\,1111\,0001_2$.
\end{enumerate}
Quindi, la rappresentazione in complemento a due di -15 su 16 bit è $1111\,1111\,1111\,0001_2$.

\section{Operazioni Aritmetiche con Complemento a Due}
Il vantaggio principale del complemento a due è che le operazioni di addizione e sottrazione possono essere eseguite utilizzando lo stesso circuito per l'addizione.

\subsection{Addizione}
L'addizione di due numeri (positivi o negativi) in complemento a due viene eseguita come una normale addizione binaria. Qualsiasi bit di riporto che esce dal bit più significativo (MSB) viene semplicemente ignorato.
\begin{itemize}
    \item \textbf{Esempio}: $5 + (-2)$ su 4 bit ($N=4$, range da -8 a 7)
    \begin{itemize}
        \item $5_{10} = 0101_2$
        \item $-2_{10}$:
        \begin{itemize}
            \item $2_{10} = 0010_2$
            \item Complemento a uno: $1101_2$
            \item Somma 1: $1101_2 + 1_2 = 1110_2$ (che è $-2_{10}$ in complemento a due)
        \end{itemize}
        \item \textbf{Somma}:
        \begin{lstlisting}[language=Pseudocode, numbers=none]
  0101  (5)
+ 1110  (-2)
-----
 10011
        \end{lstlisting}
        Il bit di riporto (il primo $1$ a sinistra) viene ignorato. Il risultato è $0011_2 = 3_{10}$, che è corretto.
    \end{itemize}
\end{itemize}

\subsection{Sottrazione}
La sottrazione $A - B$ è implementata come la somma $A + (-B)$. Per calcolare $-B$, si determina il complemento a due di $B$.
\begin{itemize}
    \item \textbf{Esempio}: $5 - 2$ su 4 bit ($N=4$, range da -8 a 7)
    \begin{itemize}
        \item $5_{10} = 0101_2$
        \item $-2_{10} = 1110_2$ (già calcolato sopra)
        \item \textbf{Sottrazione come somma}: $5 + (-2)$
        \begin{lstlisting}[language=Pseudocode, numbers=none]
  0101  (5)
+ 1110  (-2)
-----
 10011
        \end{lstlisting}
        Ignorando il riporto, il risultato è $0011_2 = 3_{10}$, che è corretto.
    \end{itemize}
\end{itemize}

\subsection{Overflow}
L'\textbf{overflow} si verifica quando il risultato di un'operazione aritmetica supera il range di valori rappresentabili con il numero di bit a disposizione.
\begin{itemize}
    \item \textbf{Rilevamento}: Si verifica un overflow se:
    \begin{itemize}
        \item Si sommano due numeri positivi e il risultato è negativo.
        \item Si sommano due numeri negativi e il risultato è positivo.
        \item Un modo più formale per rilevarlo è controllare se il riporto nel bit del segno (MSB) è diverso dal riporto fuori dal bit del segno.
    \end{itemize}
\end{itemize}
% \input{capitoli/sistemi_numerici/domande_ed_esercizi_sn.tex} % Abilitare quando avrai le domande/esercizi per Sistemi Numerici

% --- La sezione \appendix finale non è più necessaria con questa organizzazione ---
% Se in futuro vuoi comunque un'appendice per materiali generici non legati a un capitolo specifico, puoi riabilitarla.

\backmatter % Sezione finale del libro (ad es. bibliografia, indici analitici)

\printbibliography % Stampa la bibliografia. Questo comando è fornito da biblatex.

\end{document}